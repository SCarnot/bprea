\documentclass{article}
 
\usepackage[utf8]{inputenc}
\usepackage[francais]{babel}
\usepackage[T1]{fontenc}      
\usepackage[top=3.5cm, bottom=3cm, left=3.0cm, right=3.0cm]{geometry}
\usepackage{graphicx}
\usepackage{eurosym}
\graphicspath{{figures/}{../figures}}
\usepackage[activate={true,nocompatibility},final,tracking=true,kerning=true,spacing=true,factor=1100,stretch=10,shrink=10]{microtype}
\title{Culture céréalière pour la transformation boulangère : exemple d'un paysan-boulanger dans le pays Voironnais}
\date{Octobre 2019}

\begin{document}

\section{Présentation du projet}

\subsection*{Finalité}

Mon projet est de \textbf{créer, à terme, une ferme diversifiée, viable, centrée sur la culture de céréales et la transformation, en agroécologie}. Ce projet s'est construit d'abord sur une volonté personnelle de reconversion dans le monde agricole, et aussi sur mon attrait pour la culture céréalière et la boulangerie, avec toute la symbolique nourricière et traditionnelle associée. 

\subsection*{Démarche}

Que ce soit d'un point de vue personnel ou agronomique, je compte \textbf{adopter systématiquement une approche pragmatique} : c'est-à-dire partir d'une situation connue et maîtrisée, avant d'apporter des modification progressives dans les techniques culturales ou de production. L'idée de cette démarche est de ne pas compromettre la viabilité économique de la ferme et psychologique de ses associés, tout en tendant vers mes idéaux.

\subsubsection*{Production}

\textbf{La production sera centrée dans un premier temps sur la production céréalière (blé, seigle, petit épeautre) et la transformation en pain}, pour maîtriser la conduite céréalière et la fabrication du pain, assurer les canaux de commercialisation et une viabilité économique. Dans un second temps, je souhaiterais diversifier la production en ajoutant des légumineuses dans la rotation. A long terme, dans l'idéal en association, j'envisage aussi d'autres productions (bétail, maraîchage, fruitiers). Cette diversification correspond à ma vision idéale d'une ferme, permet une sécurité économique et s'inscrit dans une démarche agroécologique avec un sens agronomique.

\subsubsection*{Valeurs agroécologique}

La démarche agroécologique est une valeur essentielle à mes yeux pour la durabilité du système agricole. Il s'agit d'assurer la fertilité des sols, réduire au maximum les intrants extérieurs et les émissions de gaz à effet de serre, tout en assurant la viabilité économique de la ferme. 

\subsubsection*{Temps de travail et revenu}

Les céréales sont relativement commodes à conduire et se stockent facilement et la boulangerie permet de très bien valoriser la production. Le métier de paysan boulanger, sous réserve d'une maîtrise de l'ensemble du processus, a l'avantage de lisser et de maîtriser le temps de travail durant l'année, pour un niveau de revenu élevé en regard des métiers agricoles. Ces deux aspects (temps de travail et revenu) sont importants dans le cadre de ma reconversion : venant d'un monde salarié à horaire fixe et à salaire conséquent, la transition sera moins brutale et diminue le risque d'échec. 

Mon objectif est de viser à terme le salaire médian Français ($\sim$1700\euro{} en 2018 en France) avec un temps de travail hebdomadaire compris entre 40 et 50h.

\subsection{•}

\subsubsection*{Points forts et points faibles}

\end{document}
\documentclass{article}
 
\usepackage[utf8]{inputenc}
\usepackage[francais]{babel}
\usepackage[T1]{fontenc}      
\usepackage[top=3.5cm, bottom=3cm, left=3.0cm, right=3.0cm]{geometry}
\usepackage{graphicx}
\usepackage{eurosym}
\graphicspath{{figures/}{../figures}}
\usepackage[activate={true,nocompatibility},final,tracking=true,kerning=true,spacing=true,factor=1100,stretch=10,shrink=10]{microtype}
\title{Culture céréalière pour la transformation boulangère : exemple d'un paysan-boulanger dans le pays Voironnais}
\date{Octobre 2019}

\begin{document}

\section{Présentation du projet}

\subsection{Finalité}

Mon projet est de \textbf{créer, à terme, une ferme diversifiée, viable, centrée sur la culture de céréales et la transformation, en agroécologie}. Ce projet s'est construit d'abord sur une volonté personnelle de reconversion dans le monde agricole, et aussi sur mon attrait pour la culture céréalière et la boulangerie, avec toute la symbolique nourricière et traditionnelle associée. 

Ce projet se matérialiserait à travers une ferme disposant d'une vingtaine d'hectares, comprenant des terres cultivables et des parcelles de bois et taillis, d'un bâti agricole et d'un lieu de transformation pour la boulangerie. Dans l'idéal, avec un lieu habitable attenant ou très proche, le tout situé à une distance respectable de la métropole Grenobloise. 

\subsection{Démarche}

Que ce soit d'un point de vue personnel ou agronomique, je compte \textbf{adopter systématiquement une approche pragmatique} : c'est-à-dire partir d'une situation connue et maîtrisée, avant d'apporter des modifications progressives dans la production ou l'organisation. L'idée de cette démarche est de ne pas compromettre la viabilité économique de la ferme et psychologique de ses associés, tout en tendant vers mes idéaux.

\subsubsection*{Agroécologique}

La démarche agroécologique est une valeur essentielle à mes yeux pour la durabilité du système agricole. Il s'agit d'assurer la fertilité des sols, réduire au maximum les intrants extérieurs et les émissions de gaz à effet de serre, tout en assurant la viabilité économique et l'autonomie de la ferme. 

Au-delà de ces aspects techniques, je suis intimement convaincu qu'un retour à une société massivement agricole est la seule solution aux crises environnementales, économiques et sociétales en cours et à venir. Plus qu'un choix de vie, c'est aussi une volonté de montrer qu'une telle reconversion est non seulement possible mais aussi souhaitable. Et qu'elle puisse pousser des urbains à franchir le pas, surtout ceux qui ne se reconnaissent pas dans le milieu néo-paysan.

\subsubsection*{Production \& association}

\textbf{La production sera centrée dans un premier temps sur la production céréalière (blé, seigle) et la transformation en pain}, pour maîtriser la conduite céréalière et la fabrication du pain, assurer les canaux de commercialisation et une viabilité économique. Dans un second temps, je souhaiterais diversifier la production en ajoutant des légumineuses dans la rotation. A long terme, \textbf{dans l'idéal en association, j'envisage aussi d'autres productions} (bétail, maraîchage, fruitiers). Cette diversification correspond à ma vision idéale d'une ferme, permet une sécurité économique et s'inscrit dans une démarche agroécologique avec un sens agronomique.

\subsubsection*{Temps de travail et revenu}

Les céréales sont relativement commodes à conduire et se stockent facilement et la boulangerie permet de très bien valoriser la production. Le métier de paysan boulanger, sous réserve d'une maîtrise de l'ensemble du processus, a l'avantage de lisser et de maîtriser le temps de travail durant l'année, pour un niveau de revenu élevé en regard des métiers agricoles. Ces deux aspects (temps de travail et revenu) sont importants dans le cadre de ma reconversion : venant d'un monde salarié à horaire fixe et à salaire conséquent, la transition sera moins brutale et diminue le risque d'échec. 

Mon objectif est de viser à terme le salaire médian Français ($\sim$1700\euro{} en 2018 en France) avec un temps de travail hebdomadaire compris entre 40 et 50h.

\subsection{Déroulement}

Je décris les grandes étapes de mon projet, mais celles-ci ne sont pas rigides : elles pourront se chevaucher, précéder ou suivre une autre, ou ne pas se réaliser. La figure \ref{fig:deroulement} détaille les grandes lignes de ce déroulement.

\begin{figure}[h!]
\centering
		\includegraphics[scale=0.6]{20200122_BPREA_schema_installation.pdf}
		\caption{Déroulement du projet en grandes étapes}
		\label{fig:deroulement}
\end{figure}

\subsubsection*{Formation}

Le BPREA me permet d'acquérir les compétences de base nécessaires pour se lancer dans le monde agricole, créer un réseau et préparer mon projet. En parallèle, je vais suivre une formation à l'École internationale de boulangerie pour m'initier à la fabrication du pain au levain. Je souhaite passer le CAP boulangerie l'année suivante (juin 2021) pour m'assurer de pouvoir exercer une activité boulangère sans pour autant avoir une production de céréales. Je pourrai ainsi investir dans du matériel de boulangerie tout en commençant à l'amortir.

\subsubsection*{Salariat}

A l'issue du BPREA, je souhaiterai me mettre en salariat au moins une saison pour me former au métier, dans la boulange notamment, chez un paysan boulanger. Cette année me permet d'acquérir de l'expérience tout en recherchant activement du foncier ou en répondant à des appels d'offre.  

\subsubsection*{Lancement}

Ayant trouvé un local de transformation, je peux démarrer l'activité boulangère en parallèle du salariat, en réalisant des investissements progressivement (silo, meule, four), sans pour autant avoir suffisamment avoir de terre agricole pour pérenniser l'activité\footnote{Cela est possible uniquement si je suis diplômé du CAP boulangerie}. Je recherche activement du foncier et des terrains pour atteindre une vingtaine d'hectare qui m'assureront la viabilité économique voulue, à savoir : m'assurer un revenu, financer mes outils agricoles et le foncier. Une vingtaine d'hectare sont nécessaires. 

\subsubsection*{Pérennisation}

Une fois atteinte une SAU d'une vingtaine d'hectares, et fort d'une expérience en boulangerie, avec des canaux de vente rodés, je réalise les investissements nécessaires pour pouvoir conduire de manière autonome l'ensemble du processus de production. Il s'agit principalement des outils agricoles (tracteur, outils) et du financement du foncier. La production et le chiffre d'affaire augmentent, une organisation rigoureuse se met en place et les techniques agronome et boulangères sont acquises. La ferme atteint son rythme de croisière et permet d'atteindre les revenus souhaités et le volume de travail. 

\subsubsection*{Diversification, expérimentation}

La ferme ayant atteint son plein potentiel en terme de culture céréalière et de boulangerie, j'envisage désormais une diversification des productions de la ferme. Une association avec un éleveur de bovins permettrait d'explorer les possibilités agronomiques de la production céréalière et bovine, pour diversifier au maximum les productions sur une surface données. De la même manière, est-il possible d'ajouter du maraîchage, un verger ? De tester l'agroforesterie de bois d'œuvre ? De cultiver des céréales en traction animale ? De la possibilité de l'agriculture biologique de conservation ? Autant de propositions qui seront expérimentées.

\subsection{Points forts et points faibles}

\begin{figure}[h!]
\begin{center}
\begin{tabular}{|p{2cm}|p{5cm}|p{5cm}|}
\hline
 & Points forts & Points faibles \\
\hline
Projet & 
$\ast$ Débouchés prometteurs

$\ast$ Valorisation des céréales

$\ast$ Temps de travail régulé sur une saison

$\ast$ Possibilité d'acheter du grain en cas de catastrophe
  & 
$\ast$ Investissements conséquents

$\ast$ SAU importante

$\ast$ Mécanisation lourde indispensable
 \\
\hline
Personnels & 
$\diamond$ Capacité d'adaptation

$\diamond$ Goût à l'effort

$\diamond$ Capital personnel
& 
$\diamond$ Peu d'expérience en agriculture, en meunerie et en boulangerie

$\diamond$ Pas de foncier

$\diamond$ Peur de la commercialisation
\\
\hline
\end{tabular}
\end{center}
\label{sejour_semaine}
\end{figure}

\section{Environnement de la ferme}

Je n'ai pas de foncier pour le moment, néanmoins, \textbf{je vise une installation dans le département de l'Isère} : bas-Dauphiné, pré-Alpes (Trièves), Grésivaudan, vallée de l'Isère. La seule contrainte géographique est de ne pas être en environnement de montagne ou haute montagne, peu propices aux grandes cultures. Je choisis de m'installer dans le département isérois car je connais le territoire et j'ai un tissu social implanté dans la région.

\subsection{Environnement technico-économique}

\subsubsection*{Vente}

La partie commerciale est celle à laquelle je voudrais passer le moins de temps, je me prédispose donc à viser plutôt les magasins de producteurs ou bio et les AMAP plutôt que faire des marchés. Sachant que le marché du pain traditionnel au levain semble d'être loin d'être saturé en terme de débouchés commerciaux\footnote{Information informelle issue du bouche-à-oreille}. 

Pour évaluer les potentiels débouchés commerciaux, on peut s'intéresser aux carte de la figure \ref{fig:test}, qui donnent la densité de population et le revenu en Isère.

\begin{figure}[h!]
\centering
\begin{minipage}{.5\textwidth}
  \centering
  \includegraphics[width=0.84\textwidth]{densite38.jpeg}
\end{minipage}%
\begin{minipage}{.5\textwidth}
  \centering
  \includegraphics[width=0.9\textwidth]{revenu38.jpeg}
\end{minipage}
\caption{\textit{Gauche} : carte de densité de population en Isère. \textit{Droite} : carte du niveau de revenu en Isère}
\label{fig:test}
\end{figure}

A un endroit donné, la carte de densité de population permet d'estimer le volume de vente et la carte des revenus le prix de vente. Sachant que les magasins ou les AMAP se trouvent dans les zones de densité élevée, \textbf{il peut être intéressant de viser comme bassin de vente l'axe Grenoble-Lyon et ces métropoles elles-mêmes.} Le prix de vente peut s'estimer dans une fourchette de 4-5\euro{}/kg (au vu des prix pratiqués dans les magasins bio).

En terme de distance de trajet livraison, le maximum à parcourir est lorsqu'on se situe à mi chemin entre Grenoble et Lyon et qu'on livre à l'une des deux villes, soit 50km aller. Dans ce cas, l'aller-retour coûte environ 12\euro{} de carburant et nécessite environ 4h (chargement et livraison)pour une livraison de 200kg de pain, soit 1000\euro. \textbf{La livraison jusqu'à 50km est donc intéressante}, que la ferme se trouve dans le Trièves, le Grésivaudan ou le Nord-isère.

\subsubsection*{Entraide, partage de matériel}

Sans localisation exacte, je ne peux pas connaître la facilité à intégrer un réseau CUMA. Néanmoins, comme il est possible que je réalise les investissements agricoles en second temps, je vais avoir besoin de faire appel à une ETA ou une CUMA les premières années. 

Pour l'entraide, mon réseau amical se situe à Grenoble. J'espère pouvoir en faire appel lors de l'installation.

\subsubsection*{Approvisionnement}

En tant que paysan boulanger, les principaux besoins en terme d'approvisionnement sont principalement du carburant pour les engins agricoles, des semences (quelques centaines de kg/an), du matériel d'emballage et divers de ferme. Potentiellement du bois de chauffe si les terrains ne permettent pas d'en avoir. 

\subsection{Environnement social}
 
Ma famille et mon réseau sont de Grenoble, mais je n'ai pas de contraintes familiales actuellement. Néanmoins, je souhaiterai pouvoir continuer à enseigner au lycée Champollion à Grenoble, de l'ordre de quelques heures par semaine pour avoir une source de financement supplémentaire (de l'ordre de 50\euro{}/h, pour 2-3h hebdomadaires). 

\section{Les moyens de production}

La particularité d'un système paysan-boulanger est que les investissements sont assez conséquents aussi bien pour la partie boulangerie que pour la partie agricole. Les tarifs ont été estimés à partir de l'analyse des ventes d'occasion ou du témoignages de paysans boulangers installés. Le montant des investissements, de l'ordre de 60 000\euro{} implique un assujettissement à la TVA.

\subsection{Matériel de boulangerie}

Dans le déroulement de l'installation envisagé, il est envisagé de commencer par les investissements en boulangerie pour démarrer la production boulangère sans pour autant jouir de terrains agricoles.

\begin{figure}[h!]
\begin{center}
\begin{tabular}{|p{4cm}|p{3.5cm}|p{3.5cm}|}
\hline
 Matériel & Estimation basse (\euro{}) & Estimation haute (\euro{})\\
\hline
Silo & \hfill 4 000 & \hfill 6 000 \\
Meule & \hfill 8 000 & \hfill 10 000 \\
Labo. boulangerie & \hfill 8 000 & \hfill 20 000 \\ 
Matériel boulangerie & \hfill 500 & \hfill 3 000 \\
Four & \hfill 3 000 & \hfill 10 000 \\
\hline
\textbf{Total} & \hfill \textbf{23500} & \hfill \textbf{49 000}
\\
\hline
\end{tabular}
\end{center}
\label{sejour_semaine}
\end{figure}

\subsection{Matériel de fonctionnement et trésorerie}

Il s'agit du prix du matériel nécessaire à l'activité, même au démarrage. 

\begin{figure}[h!]
\begin{center}
\begin{tabular}{|p{4cm}|p{3.5cm}|p{3.5cm}|}
\hline
 Matériel & Estimation basse (\euro{}) & Estimation haute (\euro{})\\
\hline
Camionnette & \hfill 6 000 & \hfill 12 000 \\
Caisse homologuée & \hfill 500 & \hfill 500 \\
Frais administratifs & \hfill 500 & \hfill 1 000 \\
Trésorerie & \hfill 3 000 & \hfill 5 000 \\
\hline
\textbf{Total} & \hfill \textbf{10 000} & \hfill \textbf{17 500}
\\
\hline
\end{tabular}
\end{center}
\label{sejour_semaine}
\end{figure}

\subsection{Matériel agricole}

Le matériel agricole peut s'acheter dans un second temps. Les prix sont estimés à partir de sites de revente de matériel d'occasion. Ce matériel peut néanmoins être utilisé en CUMA.

\begin{figure}[h!]
\begin{center}
\begin{tabular}{|p{4cm}|p{3.5cm}|p{3.5cm}|}
\hline
 Matériel & Estimation basse (\euro{}) & Estimation haute (\euro{})\\
\hline
Tracteur &\hfill 25 000 & \hfill 32 000 \\
Semoir &  \hfill 300 &  \hfill 500 \\
Remorque & \hfill 2 000 & \hfill 4 000 \\
Charrue & \hfill 1 000 & \hfill 3 000 \\
Herse étrille & \hfill 1 000 & \hfill 2 000 \\
Déchaumeuse & \hfill 1 000 & \hfill 2 000 \\
Bineuse & \hfill 1 000 & \hfill 2 000 \\
\hline
\textbf{Total} & \hfill \textbf{31 300} & \hfill \textbf{45 500}
\\
\hline
\end{tabular}
\end{center}
\label{sejour_semaine}
\end{figure}

\end{document}
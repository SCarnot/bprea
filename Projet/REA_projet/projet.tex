\documentclass{book}
 
\usepackage[utf8]{inputenc}
\usepackage[francais]{babel}
\usepackage[T1]{fontenc}      
\usepackage{amsmath}
\usepackage{amssymb}
\usepackage[top=3.5cm, bottom=3cm, left=3.0cm, right=3.0cm]{geometry}
\usepackage{graphicx}
\usepackage{eurosym}
\usepackage{etoolbox} % Pour faire des citations dans le texte
\AtBeginEnvironment{quote}{\singlespacing\small} % Pour faire des citations dans le texte
\usepackage{setspace} % for \onehalfspacing and \singlespacing macros
\usepackage{xcolor}
\onehalfspacing 
\graphicspath{{figures/}{../figures}}
\usepackage[activate={true,nocompatibility},final,tracking=true,kerning=true,spacing=true,factor=1100,stretch=10,shrink=10]{microtype}
\title{Projet d'installation dans un système paysan-boulanger}
\date{Mai 2020}

\makeindex

\begin{document}

\maketitle

\chapter{Démarche, finalités et objectifs du projet}

\section{Démarche personnelle : des technologies de pointes à la terre}

Le projet de reconversion en paysan-boulanger présenté ici a été le fruit de longues réflexions, qui l'ont construit laborieusement mais consciensieusement. Depuis mon choix de me reconvertir dans l'agriculture, j'ai souvent changé de cap pour en arriver ici aujourd'hui ; et ce ne seront certainement pas les derniers amendements à ce projet. Néanmois, la direction à suivre semble être désormais la bonne : celle-ci s'est fixée autant à partir de mes réflexions que de mon vécu. Je vais ainsi commencer par présenter quelques éléments personnels, qui permettent de comprendre la plupart de mes choix et des orientations de mon projet.\textbf{ Je m'excuse auprès du lecteur pour ces digressions personnelles ; si celui-ci souhaite directement connaître le projet, il peut débuter sa lecture à la partie XX.}

\subsection{Pourquoi cette reconversion agricole ?}

D'où je viens : normale sup, thèse physqiue, IA. Ce que ça m'a apporté 
Mode de vie : ce que j'en ai retenu
Ce que j'ai appris : catastrophe climatique et pénurie des ressources. Comme bcp, j'ai lu les limties de la croissances.
Disonnance cognotive

Comment vivre avec cela avec mon mode de vie parisien ? Des changements certes, mais cosmétiques. Le militantisme ne suffit pas 

Conséquence : je suis intimement convaincu de la nécessité d'un retour massif à la vie de camapgne et nécessairement à l'agriculture. 

\subsection{Pourquoi être paysan-boulanger ?}

\subsubsection{Le maraîchage et ses limites}

De l'impératif à entamer une nécessaire et massive reconversion dans le monde agricole, il nous vient souvent à l'esprit l'image caricaturale du jeune maraîcher subversif au chapeau de paille, cultivant sur une petite surface sans mécanisation, en permaculture, vendant en circuit-court. A la question "que vous inspire l'agroécologie ?", on pense souvent à cette image-là, popularisée et portée par des personnes comme Jean-Martin Fortier, Pascal Poot ou des projets la ferme du Bec Hellouin. 

\begin{figure}[h!]
\centering
\begin{minipage}{.5\textwidth}
  \centering
  \includegraphics[width=0.9\textwidth]{pascal_poot.jpg}
  \label{fig:test1}
\end{minipage}%
\begin{minipage}{.5\textwidth}
  \centering
  \includegraphics[width=0.98\textwidth]{ferme_bec_hellouin.jpg}
  \label{fig:test2}
\end{minipage}
\caption{\textit{(Gauche)} Pascal Poot, maraîcher et semencier, réputé pour les qualités d'adaptation de ses semences et de ses plants. \textit{(Droite)} Charles et Perrine Hervé-Gruyer, de la ferme du Bec Hellouin, popularisée grâce au film \textit{Demain}. La ferme est réputée pour son approche permacultrice du maraîchage.}
\label{fig:test}
\end{figure}

Que cette représentation soit réaliste ou non, j'ai remarqué que c'est celle là qui imprègne l'imaginaire commun. Modèle pour certains, repoussoir pour d'autre, c'est  cette image en tête avec laquelle j'ai démarré ma reconversion. Je me suis dirigé vers la voie maraîchère, que j'ai pu rejoindre grâce à un heureux concours de circonstances rejoindre la formation TPH en janvier 2019 au CFPPA de Saint-Ismier. 

A travers les stages dans divers exploitations maraîchères, j'ai pu affirmer mon choix de reconversion dans le monde agricole. Cependant, j'ai été interpellé des besoins en intrants extérieurs d'une exploitation maraîchère, principalement en amendement (fumier, engrais organiques ou minéraux) et en eau, dont la consommation nécessite impérativement un système conséquent d'irrigation. Et malgré l'exemple de la ferme du bec Hellouin, il semblait très difficile de se passer de tracteur ou de plastique, sans se tuer à la tâche. Même les maraîchers les plus "écolos" se sont résignés à des pratiques qu'ils voulaient éviter.

\textbf{Finalement, le maraîchage est-il généralisable à l'image de ces micro-fermes à l'ensemble de l'agriculture ? De part le besoin important en intrant, la réponse semble être négative.} Exigeant en eau et en fumure, les légumes nécessitent beaucoup plus d'eau que leur surface de production. Si la culture des fruits et légumes est ancienne, elle semblait plus se limiter à des grands potagers proches des fermes, en auto-consommation. De fait, historiquement le maraîchage pur semble s'être développé au début de la révolution industrielle, dans les zones marécageuses\footnote{Le temre "maraîchage" est dérivé de "marais".} de la périphérie des grandes villes pour subvenir au besoins des citadins, toujours plus nombreux. Les maraichers bénéficiaient des éfluents oragniques de la ville pour leur besoin en fumure. 

\textbf{Malgré son image liée à la permaculture et à l'écologie, le maraîchage pur ne serait-il pas paradoxalement apparu conjointement à la société thermo-industrielle ?}

Pourquoi un tel engouement pour le maraîchage alors ? D'un point de vue économique, le maraîchage bénéficie d'un avantage majeur : le rendement à l'hectare étant très élevé, de l'ordre de 30 000\euro{}/ha, il est donc possible de s'installer avec peu de moyens sur une très petite surface. 

\subsubsection{Des céréales...}

A l'inverse, comment cultivaient nos ancètres avant l'utilisation massive des ressources fossiles ? Le sujet est évidemment complexe ; néanmoins il est clair que les principaux moyens de subsistance étaient issus de la culture des céréales et de l'élevage, à la base de la révolution agricole néolithique.

Les céréales, et notamment le blé, sont en effet parfaitement adaptées aux conditions pédoclimatiques de la plupart des terres cultivables. Elles nécessitent beaucoup moins d'eau en comparaison des légumes\footnote{Beaucoup de légumes, notamment les solanacées et les cucurbitacées, proviennent de régions tropicales, chaudes et humides.} et beaucoup moins de fumure. Leur cycle est parfaitement adapté au climat local : un hiver froid et un été sec. De plus, la capacité de concurrence des céréales vis-à-vis de leurs concurrentes permet de limiter grandement le temps passer à désherber. D'un point de vue agronomique, les céréales sont parfaitement adaptées à l'élevage, où la paille sert sert de litière et le fumier d'amendement. Au moyen-âge était pratiquée la rotation "triennale", où se succédaient une culture de céréales, une culture de légumineuse et une jachère, sur laquelle on pouvait faire paître des animaux.

Finalement, est-ce que ce système de polyculture-élevage ne serait pas la clé de la transition écologique ? Parfaitement autonome en intrants, peu gourmant en eau, permettant de produire beaucoup de calories à l'hectare, c'est ce système qu'ont adopté progressivement nos ancètres. Si les pratiques adoptées pour la culture des céréales et l'élevage depuis l'après-guerre sont clairement problématiques, je suis persuadé que ce sont les clés de l'agroécologie à grande échelle. 

\subsubsection{A la boulangerie}

Cependant, la faible rentabilité de la culture céréalière, de l'ordre du millier d'euros par hectare, nécessite une grande surface agricole et une importante mécanisation pour pouvoir en vivre. L'investissement très important pour démarrer une activité agricole peut être rédhibitoire pour un néo-rural. 

Lors d'un stage en juin 2019, j'ai pu découvrir le métier de paysan-boulanger à la Poule Aux Fruits d'Or, une ferme en polyculture dans le pays Voironnais. A travers la transformation en farine puis en pain, les associés valorisent très bien la culture des céréales, en atteignant un chiffre d'affaire de 100 000\euro{} pour 24ha cultivés. Au-delà de l'aspect économique, j'ai trouvé la fabrication du pain au levain selon des méthodes traditionnelles passionnante et le pain, délicieux. Le rythme de travail, certes intense, correspond aussi beaucoup plus à mes attentes : la boulangerie nécessite beaucoup de temps, mais celui-ci est planifiable et régulier. La possibilité de stocker le grain facilement permet de lisser le temps de travail tout au long de l'année, contrairement au maraîchage. 

A l'issue de mes diverses expériences, j'ai choisi d'orienter mon projet agricole sur un système paysan-boulanger, adapté à mes moyens, à ma façon de travailler et à ma vision de l'groécologie. Cette année en BPREA m'a permis d'affiner mes réflexions sur la façon de conduire ce projet et de le concrétiser. 

\section{Objectifs et finalités}

\subsection{Démarche}

Pas peur de lâcher de la thune à l'investissement pour être le plus efficacee et produire le plus par UTH

\section{Présentation globale du projet}

\subsection{Conditions pédo-climatiques}

\chapter{Commercialisation}

\section{Produits et gamme}

La gamme proposée se construit sur les produits issus des céréales et de leur transformation : \textbf{le pain et la farine}. Il s'agit là de denrées de base, dont les débouchés sont certains car la demande est forte, mais à construire. Il est à noter que la crise sanitaire du COVID-19 a montré un engouement surprenant pour la farine, qui, s'il ne sera pas aussi intense après cette crise, a toute les chances d'être durable. Enfin, si le label AB augmente le prix des produits, c'est aussi un solide argument de vente auprès d'un public de plus en plus sensible aux enjeux écologiques. 

\subsection{Pains}
\label{part:pains}

\subsubsection{Gamme principale}

L'offre de pain doit se baser en grande partie sur des pains "classiques", à la farine de blé, avec une approche traditionnelle qui permet de se démarquer de la boulangerie artisanale classique. Il y a une ouverture sur les pains sans ou avec peu de gluten (farine de sarrasin et de petit épeautre) :

\begin{itemize}

	\item[$\triangle$] \textbf{pain de campagne} : 60\% de la gamme, farine de blé T80 , farine de seigle
	\item[$\triangle$] \textbf{pain complet} : 20\% de la gamme, farine T130, son
	\item[$\triangle$] \textbf{pain au sarrasin} : 10\% de la gamme, farine de sarrasin (60\%), farine de blé T80
	\item[$\triangle$] \textbf{pain de petit épeautre} : 10\% de la gamme, farine de petit épeautre

\end{itemize}

Ces différents pains seront proposés dans différents formats : 500g, 1kg, moulé, façonnés. Les pains « classiques », pains complets et de campagne, forment 80 \% de la production. Les pains faibles (sarrasin) ou sans (petit épeautre) gluten constituent les 20 \% restants et s’adressent aux clients sensibles au gluten ou voulant des pains différents. 

\subsubsection{Gamme secondaire}

La gamme secondaire sera constituée de pains dits « spéciaux », à graines, faits à partir du pain de campagne. Les prix restent à définir, en fonction du prix des graines utilisées. 
\begin{itemize}

	\item[$\triangle$] \textbf{pain au noix}
	\item[$\triangle$] \textbf{pain au lin}
	\item[$\triangle$] \textbf{pain au tournesol}

\end{itemize}
      
Il est possible d’imaginer des pains plus complexes, qui pourraientt être réalisés une fois le processus de production des produits précédents bien maitrisé et rodé. Il pourrait y avoir par exemple :
\begin{itemize}

	\item[$\triangle$] \textbf{brioche}
	\item[$\triangle$] \textbf{pains aux olives}
	\item[$\triangle$] \textbf{pain à l’ail}
	\item[$\triangle$] \textbf{fougasse/bruschetta (huile d’olive ou tomate)}

\end{itemize}
Ces produits nécessitent une préparation particulière, plus de temps et des produits pas encore envisagés dans la production. Il faudrait par ailleurs savoir adapter l’organisation de la ferme pour ses produits, en boulangerie notamment. Mais il pourrait être tout à fait envisageable de garder une partie des terrains pour la production de tomates, aromatiques ou ail dédiée à cet élargissement de la gamme.

\subsection{Farines}

La vente de farine me permet de proposer une gamme cohérente, à base de produits céréaliers. Surtout, elle me permet de valoriser le surplus de grain issu de ma production. Je m'attends qu'avec 20ha, il soit possible de produire jusqu'à 60kg de farine par semaine, si la production de pain est limitée par le temps de travail à 300kg. Les farines sont vendues en paquet de 1kg.
\begin{itemize}

	\item[$\triangle$] \textbf{Farine de blé T80}  ; 1,5\euro{}/kg
	\item[$\triangle$] \textbf{Farine de blé T130} ; 1,8\euro{}/kg
	\item[$\triangle$] \textbf{Farine de seigle T130} ; 2,0\euro{}/kg
	\item[$\triangle$] \textbf{Farine de sarrasin} ; 3,5\euro{}/kg
	\item[$\triangle$] \textbf{Farine de petit épeautre} ; 4,0\euro{}/kg

\end{itemize}

\section{Analyse de l'environnement socio-économique}

Au-delà de la gamme de produits, le mode de commercialisation dépend de la capacité à les écouler dans un bassin de population, et de fixer les prix en fonction de ses revenus. Nous analysons l'environnement socio-économique autour de Grenoble, région dans laquelle sera implantée la ferme.

\subsection{Zone de chalandise}

La zone de chalandise pour mon projet est définie à partir de la zone accessible à 1h10 en voiture, en partant de la ferme. Celle-ci étant \textit{a priori} basée à Mens, j’ai pu définir cette zone à partir d’une isochrone de 1h10, que l’on peut voir sur la carte sur la figure \ref{fig:chalandise}. 

\begin{figure}[h!]
\begin{center}
	\includegraphics[scale=0.2]{zone_chalandise.jpeg}
	\caption{Carte isochrone de la région grenobloise, en partant de Mens (Trièves) avec 1h10 de trajet en voiture.}
	\label{fig:chalandise}
\end{center}
\end{figure}

On remarque l’effet « autoroute » : la voie rapide à Monestier de Clermont permet d’allonger les zones accessibles vers le nord, notamment toute la vallée du Grésivaudan, jusqu’aux portes de Chambéry, et tout le pays Voironnais. Alors qu’au sud, on parvient jusqu’à Gap ou Die maximum, pourtant plus proches par route ou vol d’oiseau. 

La valeur de 1h10 parcourue en voiture était une valeur qui me semblait raisonnable pour une demie-journée de livraison. J’ai commencé à me baser sur une heure de route, mais ajouter 10 min allongeait notablement la zone couverte. Evidemment, pour minimiser le temps de trajet et le carburant, je pense me focaliser d’abord sur des ventes au plus près, en visant les zones dans cet ordre : Trièves, sud de Grenoble, Grenoble, Grésivaudan. Cette isochrone permet néamoins d'avoir une idée de la zone accessible dans le cas où mes débouchés sont restreints. 

\subsection{Clientèle : pouvoir d'achat et densité de population}

Les produits vendus - pain, farine – s’adressent à une clientèle très large, donc ne ciblent pas une catégorie en particulier. Néanmoins, le monde de production et l’argumentation – agriculture biologique, agroécologie, local – s’adresse clairement à une clientèle sensible aux enjeux environnementaux, qui correspond à une clientèle globalement urbaine et aisée (que l’on le regrette ou non) et qui a les moyens de dépenser plus pour son alimentation. C’est une opportunité de valoriser ces produits auprès de cette clientèle. D’autre part, ma gamme comporte des produits à faible teneur en gluten (pain petit épeautre, sarrasin), qui concernent une proportion grandissante de personnes.

\begin{figure}[h!]
\begin{center}
	\includegraphics[scale=0.3]{thune38.png}
	\caption{Carte réprésentant, commune par commune, la densité de revenus par habitant et par unité de surface, sur les départements de l'Isère, de la Drôme et des Hautes-Alpes. Plus la couleur tire vers le jeune, plus le marché est important, avec à la fois une forte densité de population et/ou des revenus importants.}
	\label{fig:thune38}
\end{center}
\end{figure}

Pour déterminer plus précisément le lieu de vente, il peut être intéressant de connaître les zones de forte concentration de population, pour s’assurer des débouchés, mais aussi les zones correspondants à des populations à revenus assez élevés, généralement sensibles à l’argumentaire de vente : bio et local. 

Les variables densités de population et revenus ne sont pas forcément corrélées : on peut avoir des villes avec peu d’habitants plus ou moins fortunés. Pour affiner l'analyse, j’ai regardé le « volume de revenus par hectare», c’est à dire pour chaque ville, le revenu moyen de cette ville multiplié par sa population à l’hectare. Cette grandeur correspond plus ou moins à la quantité "d'argent" par hectare : plus le volume de revenus et élevé, plus le marché est important et il est intéressant de s'y rendre. Pour cela, je me suis basé sur les données fournies par l’INSEE. Ce "volume d’argent par hectare" est montré sur la figure \ref{fig:thune38}, sur les départements de l’Isère, la Drôme et des Hautes Alpes. Sans surprise, on voit que les zones intéressantes sont Grenoble et la vallée du Grésivaudan. Le Trièves et le Champsaur présentent beaucoup moins de possibilités commerciales. Gap a un potentiel intéressant. Au vu de ma zone de chalandise, je pense viser la clientèle urbaine et sensible aux enjeux écologiques de l’agglomération Grenobloise.

\subsection{Analyse de la concurrence}

La concurrence principale dans mon projet est le paysan-boulanger en bio présent autours de l’agglomération iséroise. Ce sont les mêmes produits, réalisés avec une démarche similaire, s’adressant à une population similaire et vendus sur l’agglomération grenobloise. J’en compte actuellement 4 à Grenoble.

J’ai répertorié dans la carte de la figure \ref{fig:concurrence} les différents acteurs que je connais, à date. Il doit certainement y en avoir d’autres dont je n’ai pas connaissance. 

\begin{figure}[h!]
\begin{center}
	\includegraphics[scale=0.18]{etude_concurrence.jpeg}
	\caption{Carte réprésentant la concurrence potentielle dans l'agglomération grenobloise.}
	\label{fig:concurrence}
\end{center}
\end{figure}

La concurrence indirecte est plus diverse, car elle est constituée des acteurs ayant une production de pain ou de farine mais avec une démarche différente, sur laquelle j’ai le potentiel de me distinguer. Elle se compose (par ordre de concurrence décroissant) de paysans-boulangers en conventionnel, des boulangeries AB, des boulangeries classiques et enfin des minoteries (pour la farine). 

Je n’ai pas inscrit les boulangeries sur la cartes de part leur nombre. A noter que parmi les boulangeries, il y a deux acteurs connus qui se distinguent dans l’agglomération : le pain Belledonne et le pain des cairns. Ils ne cultivent pas mais les produits vendus et la démarche sont très proches.

Mes concurrents directs ont tous des productions assez importantes et diversifiées (jusqu’à une tonne hebdomadaire de pâtes, pain et brioche pour Allicoud) et ont une implantation marquée, centrée plutôt sur les marchés qui sont leur principal lieu de vente. 

Comment pallier à ces faiblesses ? Tout d’abord, elles sont en partie dues au fait que je ne suis pas implanté, donc je ne peux pas encore diffuser ma communication, ni imaginer étendre ma gamme une fois ma production principale établie. Si je me positionne sur une stratégie commerciale sans marché, il faudra une communication très performante, qui reste à définir. 

Il pourrait être intéressant aussi de s’attaquer au marché de la boulangerie "semi-industrielle" façon Marie-Blachère qui propose sous couvert d’une certaine image "tradi" des produits industriels. Ces enseignes sont implantées dans des zones commerciales (Comboire, Seyssins). Pourrait-on imaginer une collaboration avec d’autres structures pour venir les concurrencer, comme le "Vercors-lait" ? C’est une piste que je souhaiterais étudier. 

\section{Stratégie commerciale}

\subsection{Argumentaire}

L'argumentaire commercial peut se baser sur différentes approches en lien avec mon projet : sur la qualité des pains paysan, sur la tradition qu'ils représentent et sur la démarche écologique de leur production. Voici quelques exemples d'argumentaires, à adapter en fonction du public et du contexte.

\subsubsection{Sur la qualité des pains paysans}

\begin{quote}

\textit{"La concurrence avec les boulangers classique n’est pas directe, car ils ne proposent pas principalement des pains au levains traditionnels et je ne propose pas de pain blanc. Malheureusement, la boulangerie dans son ensemble cède petit à petit à la tentation du productivisme industriel, avec des produits uniformisés et moins nutritifs. \textbf{Quant au pain paysan au levain traditionnel, cultivé en agroécologie, il est le fruit d'une fabrication délicate et de recettes traditionnelles. Les variétés anciennes de blé utilisées sont moulues sur pierre et donnent un pain au goût exceptionnel, d'une grande richesse nutritive limitant les intolérences au gluten.}"}

\end{quote}

\subsubsection{Sur la tradition et l'histoire du pain}

\begin{quote}
\textit{"L'expression « long comme un jour sans pain », traduit bien qu'au delà d'un aliment de base, il est encore essentiel dans un repas. Et \textbf{plus qu’un aliment, le pain est une tradition, un symbole de l’agriculture Française, qui caractérise notre nation peut être plus que tout}. Nul part ailleurs dans le monde, il existe un pain digne de ce nom que l’on peut trouver ici dans une simple boulangerie ici : c’est une exception française qui nous caractérise tant. A travers le pain, c’est 2000 ans d’histoire que l’on porte à nos lèvres; ce sont les générations de paysans qui ont laborieusement façonné nos paysages à travers la culture céréalière ; c’est la révolution française, déclenchée par le manque de pain, portée par l’espoir de la liberté. Mais le pain, c'est avant tout un art subtil pour sa fabrication, qui est à peine décrypté par la chimie moderne, transmis à travers le dur labeur des paysans, meuniers et boulangers."}
\end{quote}


\subsubsection{Sur la nécessité du pain et des céréales en agroécologie}

\begin{quote}

\textit{"Dans un monde bientôt bouleversé par le changement climatique, un profond désir de changement s’empare de la société, pour un monde plus écologique. Si nous avons tous l’image du petit maraîcher comme typique du jeune néo-rural, partant en première ligne du combat écologiste, le maraichage pur est une activité historiquement récente, qui nécessite un apport important d’intrants de toute sorte. De fait, sa généralisation à l’ensemble de l’agriculture n’est possible, ni souhaitable. Nous ne mangeons pas que des légumes, loin s’en faut ! Si la culture céréalière était la culture la plus répandue, c’est parce qu’elle nécessite beaucoup moins de fertilisation et d’eau. \textbf{Le pain paysan est une réponse aussi indispensable au défi écologique qui nous fait face. Le promouvoir, c'est lutter pour la révolution agroécologique.}"}

\end{quote}

\subsection{Vente en magasin}

Pour mes produits, je choisis préférentiellement la vente en magasin bio, de producteur ou AMAP. L’activité commerciale n’est ni mon point fort ni ma passion, je souhaite donc réduire le temps dédié à la vente. 

Les magasins bio et de producteurs "sous-traitent" cette partie contre une marge, mais le temps dédié est seulement celui de la livraison. La vente de pain et de farine est courante, c’est même un produit demandé, où la demande est plus forte que l’offre. L’avantage de ce canal de distribution est le gain de temps, il n’y a pas à tenir un stand une demie journée durant, il n’y a que la livraison à faire. 
Avantage : le gain de temps. Il n’y a que la livraison à faire. La demande est forte, et il y a possibilité de ne faire que quelques tournées par semaine. Il est facile d’écouler la farine pour bien la valoriser. 
Inconvénients : Le principal inconvénient est que ces structurent prennent une marge, qui peut être plus ou moins grande (20 \% magasin de producteur, 28 \% magasins bio). D’autre part, les magasins ne prennent qu’une quantité limitée, il faut donc prévoir potentiellement une grande tournée pour écouler la marchandise.

Globalement, je préfère perdre une partie de la valeur ajoutée de mes produits plutôt que consacrer beaucoup de temps à la vente. Je préfère par ailleurs garder le temps économiser à la production ou pour élargir ma gamme. 

Les AMAP peuvent être intéressantes pour la distribution de pain : il est beaucoup plus facile d’ajouter un pain à un panier que de composer avec tous les légumes. D’autre part, il y a aussi une forte demande de la part des AMAP pour ajouter du pain dans leur panier hebdomadaire. C’est une source de revenu stable, avec une bonne visibilité sur la production. 

\subsection{Politique de prix}
\label{part:prix}

Dans un système paysan-boulanger, la valeur ajoutée provient principalement de la transformation de la farine en pain et peu des matières premières (farine ou chaleur), c’est-à-dire que le coût est essentiellement un coût de main d’oeuvre. Dans une moindre mesure, il est aussi indirect à travers les investissements importants nécessaires pour démarrer l’activité. Etant à mon compte, il n’y a pas de limite basse à mon salaire (hélas) contrairement à un salarié. Le cout de revient est donc à priori faible, mais il me reste à le déterminer précisément. 

\begin{figure}[h!]
\center
\footnotesize
\begin{tabular}{ | p{3cm} | p{3cm}| p{3cm}|}
\hline
	 Produit & Prix de vente (\euro{}/kg) & Prix de vente effectif en magasin (\euro{}/kg) \\ \hline
	 \hline
	Pain complet & 4 & 5 \\ \hline
	Pain campagne & 4 & 5 \\ \hline
	Pain sarrasin & 6 & 7.5 \\ \hline
	Pain petit épeautre & 7 & 8.75 \\ \hline
	\hline
	Farine blé & 1.5 & 1.88 \\ \hline
	Farine seigle & 2 & 2.5 \\ \hline
	Farine petit épeautre & 4 & 5 \\ \hline
	Farine sarrasin & 3.5 & 4.38 \\ \hline
\end{tabular}
\caption{Récapitulatif des prix pour la gamme principale. Le prix de vente correspond aux prix encaissé par la ferme, le prix de vente effectif au prix de vente en magasin avec une commission moyenne de 25\%.}
\label{tab:prix}
\end{figure}

Je compte de toute façon déterminer mes prix sur ceux de la concurrence et aussi en fonction du chiffre d’affaire à l’hectare de mes différentes productions. Mes estimations de prix de vente par produit sont données dans le tableau \ref{tab:prix}, pour caler les prix de vente effectif (avec une marge de 25\%) en magasin. Je fixe le prix de vente de sorte à ce que le prix effectif soit aligné avec ceux de la concurrence. Le prix de base est celui du pain complet et du pain de campagne, à 5\euro{}/kg, qui représente bien l’essentiel des ventes.

Les autres produits, à base de farine d’autres céréales que le blé, sont bien moins productifs à l’hectare. En particulier le petit épeautre, qui pour un kg de grain donne 440g de farine (alors que le blé donne presque 800 grammes de farine !). Leur prix est nécessairement plus élevé. 

\subsection{Communication}

Ne visant \textit{a priori} pas une vente directe sur les marchés, ma structure aura moins de visibilité auprès du public que d'autres acteurs présents sur ce mode de commercialisation. Pour y remédier, il faut développer une communication sur les produits et les activités de la ferme.

\begin{figure}[h!]
\begin{center}
	\includegraphics[scale=0.35]{logo_ferme.pdf}
	\caption{Proposition de logo pour la ferme.}
	\label{fig:logo}
\end{center}
\end{figure}

Les réseaux sociaux et le web opèrent un rôle majeur dans la communication, que ce soit celle des individus ou des entreprises. L’agriculture, dont l’image n’est pourtant pas associée à la sophistication technologique ou en communication, s’est pourtant largement emparé de ces canaux pour partager son quotidien, ses expérimentation ou ses produits. Dans le cadre de ma communication, il serait envisageable de créer :
\begin{itemize}

	\item[$\circ$] Un site internet, dûment référencé par des moteurs de recherches\footnote{Je pense naturellement à Google, et en particulier Google Maps.}, contenant une description de la ferme et de ses activités, régulièrement mis à jour pour les montrer sous un angle positif. La réalisation d'un tel site web coûte aux alentours de 500\euro{}, mais que je peux réaliser moi-même, y compris les graphismes, comme cette proposition de logo en figure \ref{fig:logo}.
	
	\item[$\circ$] Une compte facebook, extrêmment simple et rapide à mettre en place. Il faut néanmoins régulièrement l'animer et l'alimenter pour qu'il soit suivi.
	
	\item[$\circ$] Un compte instagram, très simple à mettre en place. De la même façon que facebook, il faut régulièrement le mettre à jour avec des photographies stylisées dans l'idéal.

\end{itemize}

\chapter{Culture des céréales}
\label{chap:cereales}

\section{Contraintes du système paysan-boulanger}

\subsection{Place minimale des céréales dans la rotation}

Un système paysan-boulanger se distingue des grandes cultures par la transformation en pain d'une partie de ses cultures, permettant de capter une grande partie de la valeur ajoutée lors de la vente de ses produits : en bio, le pain est vendu aux alentours de 5\euro{}/kg, soit 5 000\euro{} la tonne, tandis que le grain de blé panifiable s'échangent à 400\euro{} la tonne. Par ailleurs, la législation limite chez les paysans boulanger à hauteur de 30\% l'achat de grain extérieur par rapport à la consommation totale nécessaire en boulangerie. 

En conséquence, les céréales vouées à la transformation (blé, seigle, ...) doivent représenter un minimum de 1/3 dans l'assolement pour assurer une bonne santé financière à la ferme. C'est un délicat équilibre entre la nécessaire diversité dans la rotation et les impératifs économiques.

\subsection{Surface minimale d'installation}

La valorisation en pain permet de décupler le chiffre d'affaire à l'hectare\footnote{En AB, on peut atteindre $\sim$12 000\euro{}/ha par le vente de pain contre $\sim$1 200\euro{} par la vente de grain, avec des rendement de 30qx/ha et un rendement de transformation en pain de 1 (1 kg de grain = 1 kg de pain).}. Si la surface d'un système paysan-boulanger est donc plus petite qu'une installation en grande culture, il faut cultiver un minimum de surface pour générer une activité économiquement viable. Etant donné des investissements lourds et indispensables, il faut viser au minimum un CA de l'ordre de 50 000\euro{} pour une installation individuelle. \textbf{La surface minimale est donc d'une quinzaine d'hectares, en prenant en compte l'assolement (seul 1/3 à la moitié de la surface est dévolu à la culture des céréales).} 

\subsection{Nécessité de la mécanisation}

Dans un système grande culture, il est impossible d’intervenir manuellement ou ponctuellement sur les culture du fait de l’importante taille des parcelles, contrairement au maraîchage, où les interventions manuelles sont possibles. Il faut donc pouvoir gérer l’enherbement, la fertilisation ou l’impact des ravageurs à grande échelle, avec des outils mécanisés. Même sur \textbf{une quinzaine d'hectares, ce quiune surface reste modeste, cultivables nécessitent impérativement d'être mécanisés}, bien que ce soit une limite à l'approche agroécologique décrite plus haut. Dans notre système socio-économique, il est tout à fait inenvisageable, pour le moment, de se baser sur la traction animale. 

Il y a deux possibilités : acheter le matériel adéquat ou externaliser ce travail en prestations. Dans le premier cas, \textbf{l'investissement est conséquent, de l'ordre de 40 à 50 000\euro{}} pour un tracteur de 90ch et les outils du sols d'occasion, mais permet de maitriser les cycles de production. C'est un choix intéressant économiquement par rapport à la prestation dans le cas d'une CUMA. \textbf{Le travail cultural réalisé en totalité en prestation revient peu ou prou à 400\euro{} par hectare}, soit 8000\euro{} de charges annuelles pour 20 hectares.

\section{Choix et démarche}

Agroécologie : durable, donc généralisable.
Agroécologie la démarche doit être généralisable : il fau tune autosuffisance ou ce qui est produit à un endroit soit consommé à un autre et inversement.
Ce sont des choix que je m'impose en plus des pratiques nécessaire à avoir en AB (cf la rotation plus bas) : rien ne 'moblige au couver tpermnent ou au minimum de labour.

\subsection{Autonomie agronomique}

Pour une ferme suivant une démarche agroécologique, il faut pouvoir \textbf{réduire au strict minimum les intrants, en plus particulièrement ceux issus de ressources fossiles}. En effet, dnas cette démarche durable, il faut donc s'affranchir au possible de ces intrants non renouvelables, tels les produits pétroliers (carburants, plastiques, engrais et chimie de synthèse) mais aussi des engrais minéraux, comme le Patentkali$^\mathrm{\textregistered}$\footnote{Cet engrais minéral "\textit{est issu des dépôts naturels de sels en Europe, formés par l’évaporation de l’eau de mer il y a plusieurs millions d’années}" d'après le producteur. C'est donc une ressource non renouvelable.}. \textbf{D'autre part, sur les ressources renouvelables (fumier, paille, grain), le bilan global de la ferme doit être neutre}, c'est-à-dire qu'il doit sortir autant de ressources qu'il n'en rentre, afin d'assurer l'équilibre écologique. Il convient aussi de produire le maximum de ces ressources sur place, pour limiter leur transport entre deux fermes.

\paragraph{Conséquences :} La chimie et les engrais de synthèse sont prohibés ; d'autre part l'usage du plastique est fortement restreint en grandes cultures. Les efforts doivent donc se concentrer sur la consommation de carburant, inévitable en grande culture, \textbf{en limitant au maximum les travaux du sol énergivores}, comme le labour. Et \textbf{d'un point de vue agronomique il faut aussi veiller à ce que les exportations (grain, foin, paille) compensent les apports de la vie du sol ou de fumier.}

\subsection{Vie du sol et biodiversité}

Les systèmes de grandes cultures conventionnels ont certes permis d'assurer une production croissante au cours de la seconde moitié du XX$^{\mathrm{i\grave{e}me}}$ siècle, mais cette croissance s'est basée sur une exploitation minière sur la qualité agronomique des sols\footnote{Les pratiques agricoles érodent les sols, jusqu'à des proportions inquiétantes dans le monde entier, y compris en France. Par exemple, le labour et des sols nus permettent aux intempéries de faiire ruisseler les particules de terre argileuse, qui ont mettent des dizaines de milliers d'années à se former, vers les fleuves, et \textit{in fine}, à l'océan.} et surtout grâce à l'utilisation massive d'engrais azotés, d'origine fossile\footnote{Les engrais azotés de synthèse, à l'orgine de l'explosion des rendements agricoles, sont produits grâce au procédé de Haber-Bosch, qui permet de fixer l'azote atmosphérique à partir du craquage du méthane, issus des gisements de pétrole ou de gaz naturel.}. Non durables, ces méthodes, associées à l'utilisation massive de pesticides, ne sont pas sans conséquences écologiques : contribution au réchauffmeent climatique, appauvrissement de la vie du sol, diminution du taux de matière organique dans les sols, effondrement de la biodiversité générale...

Pour être durable, \textbf{un système de grandes cultures doit entrer dans une démarche globale et nécessaire de préservation de l'environnement}, au prix, hélas, d'une diminution des rendements\footnote{Les rendements en AB de blé tendre dans la région Rhône-Alpes sont compris entre 25 et 35qx/ha. Pour comparaison, ils sont à 60 qx/ha en conventionnel dans la région, et à 40 qx/ha en moyenne en AB en Île-de-France. Source : \textit{Rotations pratiquées en grandes cultures biologiques en France : état des lieux par région, ITAB, 2011}}. Le nécessaire passage en agriculture biologique implique, du fait du cahier des charges, à favoriser une rotation diversifiée et à implanter des couverts d'interculture dès que le sol est à nu. Il est possible d'aller plus loin en limitant les exportations de pailles et de chaumes ou en rétablissant des haies naturelles entre les parcelles cultivées.

\textbf{Les intérêts agronomiques de ces pratiques sont multiples}. L'augmentation du taux de matière organique stimule la vie du sol en quantité et en qualité, améliore la qualité des sols par une meilleure structuration, quelque soit sa texture et \textit{in fine} améliore la santé globale des végétaux cultivés\footnote{En agriculture de conservation, le mot d'ordre est de maximiser au possible la vie du sol par la production de matière végétale : les sols sont toujours couverts, les agriculteurs suivant cette démarche constatent une diminution continue d'herbicides et de fongicides, preuve de l'amélioration de la santé végétale de leurs cultures, grâce à une vie du sol riche.}. Une biodiversité plus riche est aussi un sytème écologique plus équilibré, permettant de limiter l'impact des maladies et des ravageurs.

Au-delà de ces intérêts purement agronomiques, il est aussi essentiel de \textbf{rétablir une certaine beauté dans le paysage}, avec des champs aux cultures variées, des bandes fleuries, des haies structurant le paysage et une présence de diversité animale et végétale.

\paragraph{Conséquences :} En AB, comme l'utilisation des pesticides est interdite, préserver la vie du sol et la biodiversité implique principalement de \textbf{réduire le labour au strict minimum, c'est-à-dire ne le faire que lorsque l'enherbement devient incontrôlable}. Le labour, en retournant les horisons du sol, est particulièrement nocif pour tous les organismes du sol et favorise l'érosion. Il faut aussi pouvoir implanter un couvert végétal dès que le sol reste à nu plus de trois mois durant. Une autre possibilité est le rétablissement de haies entre au bord des parcelles. 

Il faut garder à l'esprit que ces différentes pratiques ont, à court terme, un coût direct (implantation ses couverts, diminution des rendements sans engrais azotés) et indirect (allongement de la rotation au profit de cultures moins rentables). Néanmoins, ces "coûts" ne le sont pas vraiment, car il s'agit plutôt de ne pas dilapider le capital naturel et de préserver une production à très long terme.


\section{Rotation}

Les choix agroécologiques et les contraintes propres aux grandes cultures, définis précédemment, peuvent être en grande partie respectés avec une rotation de culture bien adaptée. Définir une rotation avisée est primordial, car elle détermine totalement la production de grain et les charges culturales, et, détermine en partie la gamme de produits et le temps de travail. Dans cette partie, on présente les choix des différentes cultures et leur place dans l'assolement.

\subsection{Importance de la rotation en grande culture}

En grandes cultures biologiques, les deux contraintes forte en AB sont la fertilisation et la gestion de l'enherbement. La rotation des cultures permet de répondre à ces contraintes, déjà présente au Moyen-âge où l'on pratiquait déjà des rotations triennales, en alternant céréales, légumineuses et jachères. Conformes à cet esprit originel, celles pratiquées aujourd'hui sont plus élaborées, parfois jusqu'à 12 années successives. Les principaux critères d'une rotation sont :
\begin{itemize}

	\item[$\clubsuit$] Alterner les familles botaniques permet de casser le cycles de ravageurs et de varier les besoins en fertilisation. Les cultures appartiennent le plus souvent aux poacées (ou graminées, ou encore céréales) et les fabacées (ou les légumineuses). 

	\item[$\clubsuit$] Alterner les cultures exigeantes (blé, maïs, colza) et les légumineuses (luzerne, soja, pois), qui en fixant l'azote atmosphérique, fertilise le sol. 
	
	\item[$\clubsuit$] Alterner les cultures de printemps et d'hiver permet de casser le cycle des adventices annuelles.
	
	\item[$\clubsuit$] Respecter les temps de non-retour, de trois ans pour l es légumineuses jusqu'à 6 ans pour les crucifères. Dans le cas des céréales, on évite de cultiver la même céréales d'une année sur l'autre : par exemple, la succession blé $\rightarrow$ seigle présente beaucoup moins de risque que blé $\rightarrow$ seigle.
	
	\item[$\clubsuit$] Une période de jachère sur deux ou trois ans en tête de rotation permet de contrôler l'enherbement, et donc les interventions mécaniques. Elle peut être constituée d'une prairie semée avec une légumineuse, comme de la luzerne ou du trèfle.

\end{itemize}

Il existe encore de nombreux critères, plus spécifiques aux successions d'une espèce par rapport à une autre. Evidemment, une rotation doit aussi prendre en compte les contraintes spécifiques à la ferme : présence ou non d'irrigation, conditions pédo-climatiques, impératifs économiques, etc.

\subsection{Rotation choisie}

Au vu de mes besoins en céréales et des contraintes agronomiques, je choisis une rotation sur 8 ans, avec 1/4 de blé, 1/8 de seigle/engrain, soja, sarrasin et une prairie luzerne-graminées en tête de rotation :
\begin{align*}
\left\langle \right. \mathrm{Prairie\ sem\acute{e}e}  \left\rangle \right._{N+1\rightarrow N+3}  \longrightarrow
\left\langle \right. \mathrm{Bl\acute{e}}  \left\rangle \right._{N+4}  
\longrightarrow
\left\langle \right. \mathrm{Seigle,\ engrain}  \left\rangle \right._{N+5} \\
\longrightarrow 
\left\langle \right.\mathrm{Soja}\left\rangle \right._{N+6}
\longrightarrow
\left\langle \right. \mathrm{Bl\acute{e},\ seigle} \left\rangle \right._{N+7}
\longrightarrow
\left\langle \right. \mathrm{Sarrasin} \left\rangle \right._{N+8}
\end{align*}

Cette rotation est relativement "classique" en grandes cultures bio, avec un choix marqué pour des cultures panifiables : blé, seigle, engrain et sarrasin. Elle permet de respecter les principaux critères définis précédement. 

\begin{figure}[h!]
\begin{center}
	\includegraphics[scale=0.35]{rotation_projet_cercle.pdf}
	\caption{Parcelle de blé, fin novembre. Le blé a levé, mais on voit de nettes traces d'érosion et de tassement suite à un semis dans des conditions difficiles : terrain mal ressuyé et en pente.}
	\label{fig:rot_cercle}
\end{center}
\end{figure}

\subsection{Détail de la rotation}
\label{part:rot_detail}

\subsubsection{Années 1$\rightarrow$3 : Prairie semée (luzerne et graminées)}

\paragraph{$\circ$ Rôle} La tête de rotation est une prairie semée composée principalement de luzerne et de graminées (fétuque, ray-grass; dactyle). La luzerne enrichit en azote le sol, le structure et nettoie le champ des vivaces. Une durée trois ans assure un nettoyage efficace, mais ne peut être prolongé au-delà : les pieds de luzerne se fatiguent et prend une trop grande proportion dans la rotation. Les graminées, en plus petite proportion, permettent de diversifier l'exploration racinaire du sol et favorise une valorisation directe en fourrage pour de l'élevage. On prend des espèces qui ne sont pas utilisées dans le reste de rotation pour éviter les maladies.

\begin{figure}[h!]
\begin{center}
	\includegraphics[scale=0.3]{luzerne.jpg}
\end{center}
\end{figure}

\paragraph{$\circ$ Conduite} Après le déchaumage du sarrasin restant, on réalise au pirntemps un faux semis permettant de préparer le sol pour le semis, qui se fait à une dose de 20kg/ha pour la luzerne et 8kg/ha pour les graminées. Les deux premières années de prairie seront pourront être utilisées pour du fourrage, soit par coupe, soit en faisant paître des bêtes directement dessus. L'exportation de et d'éléments minéraux sera en partie composée par du fumier ajouté. La dernière année, la prairie sera simplement broyée, d'une part pour limiter les exportations de MO, d'autre part pour favoriser sa minéralisation avant la culture du blé suivante. La prairie est détruite par un labour à la mi-septembre de la troisième année.

\paragraph{$\circ$ Valorisation} Principalement agronomique, pour reposer le sol et le recharger en azote. La production de fourage est estimée à 10t de matière sèche par hectare pour 2 années sur 3, qui peut être vendue en fourrage à 50\euro{}/t. Les débouchés pour l'élevage en région grenobloise ne sont a priori pas un problème. 

\subsubsection{Années 4 : Blé}

\paragraph{$\circ$ Rôle} Pour valoriser au mieux l'azote restituée par la luzerne de la prairie, on place une culture 100\% blé à sa suite. C'est la culture clé de la rotation.

\begin{figure}[h!]
\begin{center}
	\includegraphics[scale=0.1]{triticum_aestivum.jpg}
\end{center}
\end{figure}

\paragraph{$\circ$ Variétés} On prendra pour moitié des semences commerciales (Energo) et paysannes (Rouge de bordeaux et Florence Aurore). Les semences commerciales assurent un certain rendement et permettent d'éviter les problème de verse par leur hauteur de taille limitée. Les semences paysannes, quant à elles, sont plus incertaines en rendement et on risque avéré de verse, mais leur diversité génétique assurent aussi une certaine résistance face aux ravageurs. Le mélange de ces deux types de semences permet de moyenner les avantages et les inconvénients. 

\paragraph{$\circ$ Conduite} On réalise un ou plusieurs faux semis de début octobre en passant la bineuse rapidement dans le champ, de préférence avant une pluie, pour favoriser la levée des adventices. Le faux-semis permet aussi de préparer le sol pour le semis, réalisé début novembre, à une dose de 200kg/ha. La herse étrille est passée une ou deux fois au printemps, pour réchauffer et réoxygéner le sol, et détruire les pousses d'adventices. Le blé est récolté courant juillet en prestation, estimée à 120\euro{}/ha. 

\paragraph{$\circ$ Valorisation} Meunerie pour produire la farine vendue en direct et transformation en pain.

\subsubsection{Années 5 : Seigle \& engrain}

\paragraph{$\circ$ Rôle} Pour valoriser au mieux l'azote restituée par la luzerne de la prairie, on place une culture 100\% blé à sa suite. C'est la culture clé de la rotation. Les pailles sont exportées pour être valorisées et limiter le risque de maladie pour les céréales suivantes.

\paragraph{$\circ$ Conduite} Identique au blé. Les doses de semis sont respectivement de 120 et 150kg/ha pour l'engrain et le seigle. L'engrain est semé à la mi-octobre ; son cycle est plus long que celui du blé ou du seigle.

\paragraph{$\circ$ Valorisation} Meunerie pour produire la farine vendue en direct et transformation en pain. Les pailles sont valorisées en litière pour l'élevage. 

\subsubsection{Année 6 : Soja}

\paragraph{$\circ$ Rôle} Le soja est cultivé après les deux pailles précédentes. Son rôle de légumineuse est donc de fournir de l'azote par fixation au sol et de couper le cycle des céréales. C'est une culture de printemps qui permet de casser le cycle des adventices d'hiver, favorisées par les cultures précédentes. On peut envisager, en fonction du climat, d'autres légumnieuses telle le pois, la lentille ou le pois chiche, dont les conduites sont proches. Dans l'idéal, on pourra privilégier une valorisation à destination humaine.

\begin{figure}[h!]
\begin{center}
	\includegraphics[scale=0.3]{glycine_max.jpg}
\end{center}
\end{figure}

\paragraph{$\circ$ Conduite} Le soja est précédée d'une interculture d'avoine de printemps et de phacélie, qui seront potentiellement faibles après l'hiver, qu'on déchaumera au mois de mars. Deux faux-semis sont réalisés entre mars et avril pour préparer le sol et détruire les adventices concurrentes. On sème durant à la fin du mois d'avril dans un sol réchauffé à une dose de 100kg/ha. Les résidus de cultures (des tourteaux principalement) sont laissés sur place après le déchaumage fin septembre, début octobre.

\paragraph{$\circ$ Valorisation} Vente directe en coopérative à 700\euro{}/t pour l'alimentation humaine.

\subsubsection{Années 7 : Blé \& seigle}

\paragraph{$\circ$ Rôle} Pour valoriser l'azote restituée par la légumineuse précédente, on place une culture blé et de seigle à sa suite. La proportion de seigle dépendra des besoins en boulangerie. On replace une culture d'hiver après la culture de printemps, pour casser de nouveau le cycle des adventices. Le blé et le seigle ont un potentiel rôle nettoyant au printemps.

\paragraph{$\circ$ Conduite} Conduite identique à celle du blé de l'année 4. Il est semé après le déchaumage du soja, début novembre.

\paragraph{$\circ$ Valorisation} Meunerie pour produire la farine vendue en direct et transformation en pain.

\subsubsection{Années 8 : Sarrasin}

\paragraph{$\circ$ Rôle} Culture de fin de rotation, qui n'est pas exigeante, nettoyante, rustique. Elle permet de valoriser les reliquats d'azote et de fourniture en matière organique en toute fin de rotation. De la famille des \textit{hydrophylacées}, le sarrasin permet d'ajouter une culture autre que les céréales et les légumineuses dans la rotation. C'est une culture très nettoyante, permettant de limiter l'enherbement en fin de rotation. Ne supportant pas la moindre gelée, il est donc semé tard, au plus tôt à la fin avril.

\paragraph{Variétés} Acheté la 1$^{\mathrm{\grave{e}re}}$ année, puis ré-ensemençage avec les semences récolotées. Au vu des contraintes de la ferme et des conditions climatiques, il serait intéressant d'essayer la variété \textit{Petit Gris}. C'est la variété la plus utilisée en France avec \textit{La Harpe}, en Bretagne principalement ; il est donc aisé de trouver un revendeur. D'autre part, c'est l'une de variété les plus productives et qui est totalement adaptée à la production de farine, qui est notre débouché principal (pour la vente en direct etla transformation en pain). Il est aussi reconnu pour ses qualités gustatives. Son cycle n'est ni tardif, ni précoce, de 100 à 120 jours. Il peut s'insérer dans la rotation s'il n'est pas semé trop tard (fin avril), pour pouvoir implanter la prairie semée mi-octobre.

\begin{figure}[h!]
\begin{center}
	\includegraphics[scale=0.4]{sarrasin.jpg}
\end{center}
\end{figure}

\paragraph{$\circ$ Conduite} Le sarrasin est implanté après une interculture de moutarde et de tournesol, qui seront affaiblis par l'hiver et déchaumés d'ébut mars, en fonction des conditions climatiques\footnote{Ou même en début d'hiver s'il y a un risque de montée en graine pour la moutarde.}. Le semis étant tardif, au début du mois de mai, le sarrasin peut être concurrencé par les adventices en début de cycle : deux faux-semis seront réalisés durant le printemps avant le semis. Rien n'est fait jusqu'à la récolte, réalisée lorsque 3/4 des grains sont mûrs\footnote{Contrairement aux autres cultures, la floraison du sarrasin est indéterminée et les grains ne sont pas mûrs au même moment.}, à partir de mi-septembre. Il est déchaumé 

\paragraph{$\circ$ Valorisation} Meunerie pour produire la farine vendue en direct et transformation en pain. Attention, les rendements sont très dépendants des conditions météorologiques au moment de la floraison, donc très variables. La floraison est par ailleurs très intense, il peut être possible de faire installer des ruches.

\subsection{Gestion de l'interculture}

Dans la rotation, il y a deux successions $\left\langle \right. $culture de d'hiver $\left\rangle \right. $ $\longrightarrow$ $\left\langle \right. $culture de printemps$\left\rangle \right. $. Comme la culture d'hiver est récoltée entre mi-juillet et début août et la culture de printemps n'est semée qu'en avril-mai, il y a une période de 9 mois avec un sol n'ayant que les chaumes de céréales, mais sans culture vivante. Pour éviter l'apparition d'adventices et pour enrichir le sol avec un couvert végétal vivant, il est agronomiquement très intéressant d'implanter une culture intermédiaire, sans récolte à la suite de la culture d'hiver. 

En effet, implanté en août, ces cultures ont un potentiel de production de biomasse important, permettent de contrôler l'enherbement et de retenir les éléments minéraux dans le sol dans leur organisme. Ces cultures n'ont pas de rentabilité directe, il faut que leur implémentation et leur destruction pour la culture suivante soit simple. Je choisis donc des cultures plutôt sensible au gel ou à un passage mécanique, de sorte à ce qu'elle soit morte en sortie d'hiver ou peu vigoureuse. 

Le premier couvert se situe à la cinquième année :
\begin{align*}
\left\langle \right. \mathrm{Seigle,\ engrain}  \left\rangle \right._{N+5}  \longrightarrow \left\langle \right.\mathrm{Avoine,\ phac\acute{e}lie}\left\rangle \right.\longrightarrow \left\langle \right. \mathrm{Soja} \left\rangle \right._{N+6}
\end{align*}
L'avoine de printemps est une graminée autre que celles présentes dans la rotation, produisant une biomasse intéressante, peu coûteux et étoufant. Il se porte bien en association avec la phacélie, qui a l'avantage d'être une \textit{hydrophyllacée}, une famille qui casse la succession des légumnieuses et des graminées. Les deux espèces ressortiront de l'hiver affaiblies et seront détruites par un roulage ou un déchaumage.

Le second couvert est à la septième année :
\begin{align*}
\left\langle \right. \mathrm{bl\acute{e}}  \left\rangle \right._{N+7}  \longrightarrow \left\langle \right.\mathrm{Moutarde,\ Tournesol}\left\rangle \right.\longrightarrow \left\langle \right. \mathrm{Sarrasin} \left\rangle \right. _{N+8}
\end{align*}
La moutarde (brassicacée) et le tournesol (astéracée) sont des familles absente de la rotation, elles cassent les cycles présents. La moutarde se dévellope très rapidement et est nettoyante. Le tournesol se déveloope rapidement, mais en hauteur. Sa racine pivotante est très intéressante pour la structuration du sol.

\textit{Remarque} : on aurait pu choisir d'implanter une légumineuse afin d'apporter de l'azote supplémentaire dans la rotation. Cependant, le premier couvert étant avant une légumineuse, on évite une succession trop impotante de cette famille. D'autre part, le second couvert est avant un sarrasin, auquel on souhaite éviter un apport trop riche en azote.

\section{Travaux du sol : coûts et temps de travail}
\label{part:W_sol}

Combien de temps, de carburant et d'argent faut-il pour travailler mécaniquement une parcelle ? Bien que la réponse dépende du matériel utilisé et des conditions pédo-climatiques, il existe des valeurs indicatives en grandes cultures, qui sont rapportées dans le tableau \ref{tab:W_sol1}. On considère que la moisson, qui nécessite un matériel particulier, est toalement sous traitée. De la même façon, les coupes de prairie semée sont sous-traitées.

\begin{figure}[h!]
\center
\footnotesize
\begin{tabular}{ | p{1,6cm} | p{1,6cm}| p{2cm}| p{1,6cm} | p{1,5cm} | p{2cm} | p{1,8cm}| }
\hline
	Tâche & Coût / ha (\euro{}/ha) & Coût presta. / ha (\euro{}/ha) & Temps / ha (h/ha) & Temps total (h) & Conso. GNR / ha (L/ha) & Conso GNR horaire (L/h) \\ \hline
	Herse & 15 & 30 & 0.3 & 4.5 & 3 & 10 \\ \hline
	Coupe & - & 100 & - & - & - & - \\ \hline
	Labour & 50 & 100 & 0.55 & 1.375 & 15 & 27.3 \\ \hline
	Récolte & - & 120 & - & - & - & - \\ \hline
	Semis & 40 & 80 & 1.5 & 30 & 15 & 10 \\ \hline
	Binage & 20 & 40 & 0.5 & 2.5 & 5 & 10 \\ \hline
	Faux-semis & 20 & 40 & 0.5 & 8.75 & 5 & 10 \\ \hline
	Déchaumage & 40 & 80 & 0.6 & 10.5 & 10 & 16.7 \\ \hline
\end{tabular}
\caption{Valeurs indicatives de coûts, temps et consommation des travaux du sol pour différents travaux en grandes cultures. L'absence d'informations, représentée par un "-", indique une réalisation par une prestation. Valeurs indicatives de coût, temps et consommation de GNR pour différents travaux du sol.}
\label{tab:W_sol1}
\end{figure}

La première colonne, "Coût / ha (\euro{}/ha)", comprend l'ensemble des charges de mécanisation, dont l'amortissement, pour une utilisation "normale" en grandes cultures, c'est-à-dire environ 600h/an. Les valeurs dans le cas de la rotation proposée à la figure \ref{fig:rot_cercle} sont présentées en \ref{tab:W_sol2}, avec les itinéraires techniques décrits dans la partie \ref{part:rot_detail}, en prenant en compte les couverts d'interculture.

\begin{figure}[h!]
\footnotesize
\center
\begin{tabular}{ | p{1,6cm} | p{1,6cm}| p{2cm}| p{2cm} | p{1,5cm} | p{1,8cm} | }
\hline
	Tâche & Opérations par an & Coût total (\euro{}) & Coût total presta. (\euro{}) & Temps total (h) & Conso. GNR totale (L) \\ \hline
	Herse & 6 & 225 & 450 & 4.5 & 45 \\ \hline
	Coupe & 6 & 1500 & 1500 & - & - \\ \hline
	Labour & 1 & 125 & 250 & 1.375 & 37.5 \\ \hline
	Récolte & 5 & 1500 & 1500 & - & - \\ \hline
	Semis & 8 & 800 & 1600 & 30 & 300 \\ \hline
	Binage & 2 & 100 & 200 & 2.5 & 25 \\ \hline
	Faux-semis & 7 & 350 & 700 & 8.75 & 87.5 \\ \hline
	Déchaumage & 7 & 700 & 1400 & 10.5 & 175 \\ \hline
	\textbf{Total} & \textbf{42} & \textbf{5300} & \textbf{7600} & \textbf{57.63} & \textbf{670} \\ \hline
\end{tabular}
\caption{Valeurs indicatives de coûts, temps et consommation des travaux du sol pour la rotation proposée à la figure \ref{fig:rot_cercle}. Chaque opération est réalisée sur un bloc de 2,5ha (rotation de 8 ans qur 20 ha).}
\label{tab:W_sol2}
\end{figure}

Avec à peine 60h de travaux de culture, le temps de travail reste très limité. Ce n'est pas surprenant pour 20ha avec la moisson et les coupes qui sont sous-traitées. Si le coût de la prestation semble réaliste, il faut néanmoins relativiser le coût total de 5300\euro{}, qui part de l'hyothèse d'une utilisation intensive du tracteur. Ces coûts seront rediscutés au chapitre \ref{part:chap_eco}.

\subsubsection{Mécanisation}

L'ensemble de l'outillage mécanique est estimé à 50 000\euro{}, avec un amortissement sur 12 ans, soit 4167\euro{} par an.

On considère que l'entretien et la reparation de ce matériel est de 5\euro{} de l'heure. Il comprend les pneus, l'huile et la réparation des outils. C'est une valeur haute : pour une utilisation de 600h/an, pour un tracteur de 90 ch, il s'établit à 3,3\euro{}/h\footnote{\textit{Coûts des Opérations Culturales 2018 des Matériels Agricoles}, Chambres d'agriculture de France.}. Comme le tracteur est beaucoup moins utilisé (moins de 100h/an), le coût horaire peut augmenter (si par exemple il faut faire la vidange toutes les 400h ou les 2 ans). On se calque donc sur une valeur surestimée.

Le carburant, le gazole non routier (GNR), est pris à 0,85\euro{}/l\footnote{Il s'établissait à 0,85\euro{}/l. Les prix des carburants pétroliers étant très variables, on prend la valeur haute.}. On arrive donc à un coût de 670\euro{}.

Les prestations incluent les coupes et les récoltes, qui représentent 3000\euro{}.

\chapter{Meunerie et boulangerie}

\section{Stockage du grain}

\subsection{Céréales}

\subsection{Cas du sarrasin}

\section{Meunerie}

Matériel nécessaire : un moulin astrié. Comme difificle à trouver d'occasion, achat en neuf. ça permet de s'assurer de la fiabilité, c'est un rouage essentiel dans le fonctionnement de la ferme.

\section{Boulangerie}

\subsection{Technique de fabrication}

Tradition levain, repos du pain, four à bois.

\subsection{Aménagement}

Travaux pour isoler et contrôler la température et l'humidité. 

\subsection{Matériel nécessaire}

Essentiel : 
Pétrin mécanique : diminue grandement le temps de travail et la fatigue physique
Four à bois grande capacité 200kg. Cher à l'investissmeent mais permet d'économiser sur l'énergie et de réaliser des fournées d'un bloc.

\subsection{Organisation}

2 fournées de 150kg pour 8h de travail chacune. Réalisées le soir tard ou le matin tôt

\chapter{Analyse économique}
\label{part:chap_eco}

La partie précédente a agronomique a permis de voir la conduite des culture pour assurer à la ferme une production viable agronomiquement parlant. L'analyse économique sert à savoir si le projet est viable financièrement.

\section{Processus de production global de la ferme}

Pour réaliser l'analyse économique de la ferme, \textbf{il faut avoir une vue globale sur l'ensemble de ses productions, qui feront son chiffre d'affaire, mais aussi des intrants nécessaires à cette production, pour estimer les charges}. Un système paysan-boulanger est plus complexe que les grandes cultures, car il y a trois processus de production : la culture des céréales, la meunerie et la boulangerie. Les schémas présentés dans la suite permettent de comprendre les différentes sources de revenus et les postes de charges spécifiques à chaque production.

\subsection{Culture des céréales}

La conduite des céréales a été présentée au chapitre \ref{chap:cereales}. Il est résumé, dans une vision comptable dans la figure \ref{fig:flux_champ}. 

Pour produire, le champ "reçoit" des semences commerciales (prairie semée, soja, blé, et couverts d'interculture) ainsi que du travail mécanique, consommant 670L de GNR (ou l'équivalent de 7600\euro{} de travail en prestation).


Chaque année, une partie de la production de grain est réutilisée en semence (paysanne). Finalement, une fois déduite les besoins en semences, la production de grain s'élève à :
\begin{itemize}

\item[$\diamondsuit$] 114 qx de blé tendre, sur 4,75ha ; 
\item[$\diamondsuit$] 17,63 qx de seigle, sur 0,75ha ; 
\item[$\diamondsuit$] 27,6 qx d'engrain, sur 2 ha ; 
\item[$\diamondsuit$] 24,1 qx de sarrasin sur 2,5ha ; 
\item[$\diamondsuit$] 50 qx de soja sur 2,5ha.

\end{itemize}
En grain, seul le soja est vendu directement en coopérative, au cours de 700\euro{} la tonne ; le reste est sotcké en silo avant de rejoindre la meunerie.

Le fourrage est exporte deux année sur trois, soit 500 quitaux de matière sèches. Le reste est laissé sur place pour apporter de la matière organique au sol.

\begin{figure}[h!]
\begin{center}
	\includegraphics[scale=0.6]{flux_physique_champ.pdf}
	\caption{Description des flux physiques nécessaire à la bonne production des cultures. Les intrants sont les semences et le carburant pour le tracteur (ou la prestation pour réaliser les différentes tâches). Les productions de grains sont en partie (blé, seigle, engrain et sarrasin) utilisées en semences paysannes, le reste est sotcké en silo. Une année sur 3, le fourrage est laissé sur place, soit l'équivalent de 250qx de matière sèche.}
	\label{fig:flux_champ}
\end{center}
\end{figure}

\subsection{Meunerie}

L'ensemble des grains, sauf le soja, sont moulus pour être valorisé en farine, puis, en partie, en pain. Le rendement en farine\footnote{C'est le rapport entre la masse de farine obtenue et la masse de graine moulue.} est variable selon le type de grain : le meilleur rendement est celui du blé, le plus faible est celui du petit épeautre, qu'il faut décortiqué avant de moudre. Dans l'estiamtion du rendement, on a pris en compte les 2\% de perte.

La production totale de farine s'élève à :
\begin{itemize}

\item[$\diamondsuit$] 8 940kg de farine de blé ; 
\item[$\diamondsuit$] 1 040kg de farine de seigle
\item[$\diamondsuit$] 1 325kg de farine de petit épeautre; 
\item[$\diamondsuit$] 1 530 kg de farine de sarrasin ; 

\end{itemize}

\begin{figure}[h!]
\begin{center}
	\includegraphics[scale=0.7]{flux_physique_moulin.pdf}
	\caption{Description des flux physiques nécessaire à la bonne production des cultures. Les intrants sont les semences et le carburant pour le tracteur (ou la prestation pour réaliser les différentes tâches).}
	\label{fig:flux_moulin}
\end{center}
\end{figure}

Au débit de 15kg de grain moulu par heure, il faut au total 1200h environ à l'année pour moudre l'ensemble du grain, soit en moyenne 3h20min par jour. Le moulin consomme alors 1 200kWh d'électricité. Le grain est moulu au fur et à mesure de la fabrication du pain. 

\subsection{Boulangerie}

Une fois le grain moulu, une partie seulement de la farine est utilisée pour la boulangerie. En effet, la quantité de pain maximale que je peux produire est de 300kg, afin de limiter le temps de travail. Pour comparer les besoins en boulangerie avec la production de farine, il faut regarder les besoins en farine pour chaque pain, estimés sur la figure \ref{fig:flux_boulange}. 

\begin{figure}[h!]
\begin{center}
	\includegraphics[scale=0.5]{flux_physique_boulangerie.pdf}
	\caption{Processus de fabrication des différents pains. Les valeurs indiquées représentent les proportions des ingrédients dans la fabrication du pain ou dans la répartion de la farine.}
	\label{fig:flux_boulange}
\end{center}
\end{figure}

%La farine de blé occupe une palce centrale, car elle rentre dans la fabrication des pains de camapgne, complets et au sarrasin. 

A partir de la figure \ref{fig:flux_boulange}, on peut calculer les besoins en farine pour fabriquer 300kg de pain par semaine, avec la gamme définie en partie \ref{part:pains} : 
\begin{itemize}

	\item[$\ast$] 60 kg de pain complet : 40,2kg de farine de blé ;
	\item[$\ast$] 180kg de pain de campagne : 107,3kg de farine de blé et 16,9kg de farine de seigle;
	\item[$\ast$] 30kg de pain compleau sarrasin : 7,59kg de farine de blé et 11,85kg de farine de sarrasin ;
	\item[$\ast$] 30kg de pain de petit épeautre : 19,7kg de farine de petit épeautre.

\end{itemize}

Au vu des quantités de farine nécessaire pour le pain, il existe un surplus hebdomadaire de 31kg de farine de blé, 5 kg de farine de seigle, 8kg de farine de petit épeautre et 20 kg de farine de sarrasin. On considère que le travail est effectué sur 48 semaines à l'année.

\section{Bilan de départ}
\label{bilan_ouverture}

Le bialn comptable de départ comptabilise l'ensemble des biens mobiliers, immobiliers et fonciers nécessaire au démarrage de l'activité et les moyens à travers lesquels ils ont été financés

\subsection{Actifs d'ouverture}

L'actif d'ouverture représente la valeur comptable de l'ensemble des biens de l'entreprise au démarrage de l'activité, classé par liquidé croissante. Dans le projet présenté ici, il s'élève à environ 215 000\euro{}, dont les détails sont présentés dans le tableau \ref{tab:actif}. Les prix sont estimés en se basant sur les prix du marché, et en tenant en compte de l'autoconstruction, pour l'aménagement de la boulnagerie et les silos par exemple.

\begin{figure}[h!]
\footnotesize
\center
\begin{tabular}{ | p{3cm} | p{2cm}| p{2cm}| }

\hline
	Foncier & Terrain & 60 000 \euro{} \\ 
	 & Bâtiment & 50 000 \euro{} \\ 
	 & Travaux & 10 000\euro{} \\ \hline
	Matériel agricole & Tracteur & 30 000 \euro{} \\ 
	 & Semoir & 1 000 \euro{} \\ 
	 & Herse & 2 000 \euro{} \\ 
	 & Charrue & 3 000 \euro{}\\ 
	 & Déchaumeur & 5 000 \euro{}\\ 
	 & Bineuse & 2 000 \euro{}\\ 
	 & Silo & 5 000 \euro{}\\ \hline
	Matériel boulangerie & Meule & 12 000 \euro{}\\ 
	 & Ensacheuse & 2 000 \euro{}\\ 
	 & Matériel divers & 1 000  \\ 
	 & Pétrin & 3 000 \euro{}\\ 
	 & Four & 10 000 \euro{}\\ \hline
	Matériel divers & Camionnette & 8 000  \\ 
	 & Outillage & 1 000 \euro{}\\ 
	 & Caisse & 500 \euro{}\\ \hline
	\textbf{Total actif immo.} & &\textbf{205 500 \euro{}}  \\ \hline
	\hline
	Stock & Grain & 3 757 \euro{}\\ 
	 & Semence & 3 565 \euro{}\\ \hline
	Liquidités & Banque & 2 000 \euro{}\\ \hline
	\textbf{Total actif circulant} & & \textbf{9 322 \euro{}}  \\ \hline
	\hline
	\textbf{Total actif} & & \textbf{214 822 \euro{}}  \\ \hline

\end{tabular}
\caption{Actif du bilan d'ouverture. Les prix sont des estimations réalisés sur le site des constructeur ou sur des sites de matériel d'occasion.}
\label{tab:actif}
\end{figure}

\begin{itemize}
	
	\item[$\bigstar$]	Le prix du foncier agricole, aux alentours de Mens, est estimé à environ 3 000\euro{} l'hectare. De même, un bâtiment agricole est estimé à 50 000\euro{}, dans lequel il serait possible de faire des travaux de l'ordre de 10 000\euro{} pour aménager la boulangerie.
	\item[$\bigstar$] La valeur matériel agricole, acheté d'occasion, s'élève à 48 000\euro{}. Le tracteur nécessaire pour le projet doit avoir 90ch avec 5000h au compteur.
	\item[$\bigstar$] Dans le matériel de boulangerie, la meule est achetée neuve, soit 12 000\euro{}, le reste peut être acheté d'occasion.
	\item[$\bigstar$] Dans l'actif circulant, 7 322\euro{} sont prévus en achat de grains, directement pour démarrer l'activité de boulangerie et en semences pour démmarer la partie agricole. Ces montants sont justifiés dans la partie \ref{part:previsionnel}.
\end{itemize}

\subsection{Passifs d'ouverture}
\label{part:passif}

Le passif d'ouverture de l'entreprise correspond au financement de l'actif d'ouverture et son montant est nécessairement égal à ce dernier. Pour ce projet, il présenté dans le tableau \ref{tab:passif}. Il se décompose principalement dans les capitaux propres, c'est-à-dire les fonds dont je dispose personnellement pour démarrer l'activité, et les emprunts.

\begin{figure}[h!]
\footnotesize
\center
\begin{tabular}{ | p{2,5cm} | p{3,5cm}| p{2cm}| }

\hline
	Capital individuel & Capitaux propres & 60 000 \euro{}\\ 
	& DJA & 45 000 \euro{}\\ 
	& Subventions & 0 \euro{}\\ \hline
	\textbf{Total capital} & & \textbf{105 000} \euro{}\\ \hline
	 \hline
	Emprunts & Bancaire LT & 50 500 \euro{} \\ 
	& Financement participatif & 50 000 \euro{}\\ 
	& Bancaire CT & 9 322 \euro{}\\ \hline
	\textbf{Total emprunts} & & \textbf{109 822 \euro{}}\\ \hline
	 \hline
	\textbf{Total passif} & &\textbf{214 822 \euro{}}\\ \hline

\end{tabular}
\caption{Passif du bilan d'ouverture.}
\label{tab:passif}
\end{figure}

\begin{itemize}

	\item[$\bigstar$] Dans le capitaux propres, je peux investir personnellement 60 000\euro{} dans la ferme. Je compte aussi sur une DJA d'un montant de 45 000\euro{}, qui correspond à l'installation d'un jeune agriculteur en zone de montagne (dont fait partie le Trièves), en AB, avec les investissements proposés sur le tableau \ref{tab:actif}.
	
	\item[$\bigstar$] Je dois compléter mes capitaux propres par des emprunts, pour atteindre la totalité de la valeur de l'actif de démarrage, soit une dette de 109 822\euro{}. Pour éviter des annuités trop importantes, les emprunts sont à long terme, pour financer principalement les actifs à l'amortissement long (cf le tableau \ref{tab:amortissement} ci-dessous), comme l'achat du terrain, des bâtiments ou de la meule. L'emprunt à court terme (CT) permet de financer les semences de la première année.
	
	\item[$\bigstar$] Je souhaiterais diversifier les sources d'emprunts. Plus particulièrement, je souhaiterais utiliser en partie un emprunt participatif à travers mon réseau, acommpagné par des structures adaptées, notamment celle proposée par Sébastien Kraft. Je compte lever 50 000\euro{} à un taux d'emprunt de l'ordre de 1\%.
	
	\item[$\bigstar$] Les financements long terme seront aussi complétés par des emprunts bancaire, aussi de l'ordre de 50 000\euro{}, à un taux de 1,5\%.

\end{itemize}

Le projet proposé nécessite donc des investissements élevés, propre au système paysan-boulanger, comme la valeur élevée du matériel nécessaire (matériel de boulangerie, de meunerie, etc.) mais aussi du fait de mes choix, principalement par l'achat du foncier et du matériel agricole : ces postes représentent 150 000\euro{} d'investissement.

\subsubsection{Taux d'endettement}

\textbf{Le taux d'endettement est de 51\%}, ce qui est élevé, mais acceptable pour une installation.

\subsection{Annuités d'emprunts}

Les emprunts envisagés dans le passif d'ouverture, détaillés en partie \ref{part:passif} engendrent des annuités à rembourser chaque année. Le tableau \ref{tab:annuites} présente les différents emprunts, avec le taux et leur durée.

\begin{figure}[h!]
\footnotesize
\center
\begin{tabular}{ | p{3,5cm} | p{3cm}| p{1cm}| p{1,5cm}|p{2cm}| }
\hline
	& Montant de l'emprunt & Taux & Durée (ans) & Annuités \  \\ \hline
	Bancaire LT & 56 257\euro{} & 1,5\% & 15 & 4 216\euro{} \\ \hline
	Financement participatif & 50000\euro{} & 1\% & 10 & 5 279\euro{} \\ \hline
	Bancaire CT & 3 565\euro{} & 4\% & 2 & 1 890\euro{} \\ \hline
	\textbf{Total} & \textbf{109 822\euro{}} &  & & \textbf{11 385\euro{}}  \\ \hline
\end{tabular}
\caption{Dtail des emprunts, de leur taux, de leur durée et des annuités correspondantes.}
\label{tab:annuites}
\end{figure}

\textbf{Les annuités s'élèvent donc à environ 10 000\euro{}/an, charges financières comprises.} Notons que l'emprunt courte durée, ne dure que 2 ans : les annuités sont donc 11 000\euro{} les deux premières années, puis à environ 9000\euro{} après. 

\section{Produits et charges à terme}

Pour évaluer la viabilité, il peut être intéressant de regarder les produits et les charges à terme, c'est-à-dire une fois que la production est stabilisée au objectifs visés. On ne prend donc pas ici la montée en puissance de la production les premières années : ce prévisionnel sur les 3 premières années est présenté en partie \ref{part:previsionnel}. Il est alors possible de calculer les charges associées, et d'en déduire ainsi la viabilité économique de la ferme.

\subsection{Chiffre d'affaire}

A partir des quantités de pains et de farines produites estimées dans la partie précédente, en considérant une production de 300kg de pain hebdomadaire, on peut calculer le chiffre d'affaire de la ferme. Avec les prix fixés à la partie \ref{part:prix}, il s'élève à 1 507\euro{} par semaine\footnote{On considère otujours 48 semaines de production à l'année.}, \textbf{soit 78 300\euro{} à l'année, en tenant compte aussi de la vente du soja et du fourrage.} On ne tient pas compte des aides de la PAC, qui pourraient être de l'ordre de 4000\euro{}.

Le tableau suivant décrit la répartition du chiffre d'affaire en fonction des produits : 
\begin{figure}[h!]
\footnotesize
\center
\begin{tabular}{ | p{3cm} | p{2cm}| p{2,5cm} |p{2cm}| }
\hline
	& Production annuelle (kg) & Prix de vente (\euro{}/kg) & CA annuel (\euro{})  \\ \hline
	Pain complet & 2880 & 4 & 11 520 \\ 
	Pain campagne & 8 640 & 4 & 34 560 \\ 
	Pain sarrasin & 1 440 & 6 & 8 640 \\
	Pain petit épeautre & 1440 & 7 & 10 080 \\ \hline
	Total pain & 14 400 & - & 64 800 \\ \hline
	\hline
	Farine blé & 1 483 & 1.5 & 2 225 \\
	Farine seigle & 224 & 2 & 448 \\ 
	Farine petit épeautre & 379 & 4 & 1 517 \\ 
	Farine sarrasin & 964 & 3.5 & 3 374 \\ \hline
	Total farine & 3 050 & - & 7 564 \\ \hline
	\hline
	Soja & 5 000 & 0.7 & 3 500 \\ 
	Prairie & 50 000 & 0.05 & 2 500 \\ \hline
	\hline
	\textbf{Total produits} & - & - & \textbf{78 364} \\ \hline
\end{tabular}
\caption{Chiffre d'affaire réalisés par les ventes du pain, de la farine et des produits végétaux.}
\label{tab:CA}
\end{figure}

Le pain représente la grande partie du chiffre d'affaire, avec plus de 80\%, si la production visée est de 300kg par semaine. Il est donc essentiel que les débouchés commerciaux soient assurés. 

Par ailleurs, on peut remarquer que le chiffre d'affaire est élevé pour une exploitation agricole avec un UTH d'une 20ha : c'est une des caractéristiques spécifiques au système paysan-boulanger. 

\subsection{Charges}

\subsubsection{Charges variables et de structure}

Les dépenses liés directement à l'activité de l'entreprise, les charges, peuvent être ventilées suivant les postes d'activités (agricole, meunerie, boulangerie) ou en charge variables et de structure, comme présenté sur le tableau \ref{tab:charges}. On a pris en compte dans les charges variables prennent en compte toutes les dépenses qui augmentent en même temps que l'activité, comme les sachets de pain ou de farine. Les charges de structures regroupes les dépenses qui restent fixes quelques soit le niveau d'acitvité de l'entreprise. 

\begin{figure}[h!]
\footnotesize
\center
\begin{tabular}{ | p{3cm} | p{4cm}| p{2cm}| }

\hline
	Charges variables & Gazole & 570 \\ 
	&Prestations & 3 000 \\ 
	&Entretien mat. agricole & 288   \\ 
	&Semences & 1 779 \\ 
	&Carburant livraison & 1 411\\ 
	&Sachet pain & 360   \\
	&Sachet farine & 113\\
	&Eau & 27 \\ 
	&Electricité moulin & 259 \\
	&Bois four à pain & 2 340 \\ 
	&Sel & 486 \\
	&Autres ingrédients & 0 \\ \hline
	Total &  & 10 632 \\ \hline
	\hline
	Charges de structure & Entretien mat. agricole & 300 \\
	&Certif AB & 400\\
	&Entretien véhicule & 500\\
	&Eau courante & 160\\
	&Elec. Courante & 850 \\ 
	&Téléphone, internet & 500\\
	&Consommables bureautique & 500 \\
	&Divers entretien ménager & 500\\
	&Assurance & 2 000\\ 
	&Comptabilité & 1 000 \\
	&Banque & 300\\
	&MSA & 3 000\\
	&Impôts & 0 \\
	&Salariat & 0 \\ \hline
	Total &  &10 010  \  \\ \hline
	\hline
	\textbf{Total charges} &  & \textbf{20 642} \  \\ \hline
\end{tabular}
\caption{Description des charges variables et de structures.Dans les charges variables, les montants sont calculés à partir des estimations réalisées sur les parties précédentes. Les charges de structures sont estimés à partir de valeurs "claissques".}
\label{tab:charges}
\end{figure}

Quelques précisions sur ces charges :
\begin{itemize}

	\item[$\bigstar$] Le coût du GNR est pris à 0,85\euro{}. La consommation, le coût des prestations et les besoins en semences ont été estimés à 670L à la partie \ref{tab:W_sol2}. Le coût d'entretien du matériel agricole est estimé à 5\euro{}/h d'utilisation et 300\euro{} de frais fixes.
	\item[$\bigstar$] La livraison est basée sur 2 aller-retours Mens-Grenoble par semaine, soit $2\times140$km avec une consommation de 7L/100km et un pris du carburant de 1,5\euro{}/L. L'entretien du véhicule est estimé à 500\euro{}/an.
	\item[$\bigstar$] Côté boulangerie, on a cosidéré qu'il fallait un schaet par kilo de pain ou de farine, avec des prix de respectivement 0,03 et 0,04\euro{} le sachet. Le four consomme environ 1 stère de bois pour 400kg de pain\footnote{C'est une valeur aproximative qui dépend de l'isolation du four et du nombre de fournées par semaine.}, à 65\euro{} la stère. D'un point de vue énergétique, le bois est bien moins cher que l'électricité : 0,04\euro{}/kWh contre 0,17\euro{}/kWh, il est donc raisonnable d'investir dans un four à bois, amorti en 3 ans par rapport à un four électrique.
	\item[$\bigstar$] Les consommations d'eau et d'électricité sont basées sur celles d'un ménage français moyen, ramené à une personne. 
\end{itemize}

\subsubsection{Charges d'amortissement}

Les charges d'amortissement sont des dépenses indirectes, liées à l'usure naturelle du matériel. Leur calcul est basée sur des règles comptables artificielles mais ne réprésentent nécessairement de dépenses réelles sur les compte de l'entreprise : par exemple, le tracteur perdra d'une année sur l'autre une valeur comptable 4 285\euro{}, mais ne perdra pas pour autant de valeur utilitaire, s'il est parfaitement en état de marche. Les charges d'amortissement, présentées dans le tableau \ref{tab:amortissement}\footnote{On utilise un amortissement linéaire.}, permettent de tenir compte du coût de remplacement à terme du matériel.  

\begin{figure}[h!]
\footnotesize
\center
\begin{tabular}{ | p{2,8cm} | p{2,8cm}| p{2cm}| p{2cm}| }
\hline
	& & Durée (années) & Montant\\\hline
	Bâtiment & Achat & 15 & 3 333\euro{} \\ 
	 & Travaux & 15 & 667\euro{} \\ \hline
	Matériel agricole & Tracteur & 7 & 4 286\euro{} \\
	 & Semoir & 7 & 143\euro{} \\
	 & Herse & 7 & 286\euro{} \\ 
	 & Charrue & 7 & 429\euro{} \\ 
	 & Déchaumeur & 7 & 714\euro{} \\
	 & Bineuse & 7 & 286\euro{} \\
	 & Silo & 15 & 333\euro{} \\ \hline
	Matériel boulangerie & Meule & 10 & 1 200\euro{} \\ 
	 & Ensacheuse & 5 & 400\euro{} \\ 
	 & Matériel boulangerie & 5 & 200\euro{} \\ 
	 & Pétrin & 5 & 600\euro{} \\ 
	 & Four & 10 & 1 000\euro{} \\ \hline
	Matériel divers & Camionnette & 5 & 1 600\euro{} \\ 
	 & Outillage & 5 & 200\euro{} \\ 
	 & Caisse & 5 & 100\euro{} \\ \hline
	 \hline
	\textbf{Total} &  & & \textbf{15 776\euro{}} \  \\ \hline
\end{tabular}
\caption{Description des charges d'amortissement. L'amortissement du matériel agricole s'élève à 6 476\euro{}.}
\label{tab:amortissement}
\end{figure}

L'amortissement s'élève au total à 15 776\euro{}, dont 6 476\euro{} de matériel agricole. Il est important de noter que le matériel agricole représente un coût élevé dans les charges d'amortissement, alors qu'il est utilisé 60h/an.

\subsection{Indicateurs comptables}

Une fois estimés les produits et les charges, il est possible d'en déduire la viabilité de l'entreprise, à travers des indicateurs comptables permettant d'estimer le revenu de l'agriculteur, sa capacité à réinvestir dan le matériel usé, ou la maîtrise de son endettement. 

\subsubsection{EBE (Excédent brut d'exploitation)}

L'excédent brut d'exploitation, ou EBE, permet de mesurer la valeur créée par la ferme, une fois déduites des produits les charges variables et de structure. Dans notre cas, l'EBE s'établit à :


\noindent\fbox{\parbox{\linewidth\fboxrule-2\fboxsep}{
\textbf{EBE = 78 364\euro{} - 20 642 = 57 722\euro{}}
}}

Cet EBE va être ensuite réparti entre 3 postes : les amortissements à travers la capacité d'autofinancement, les annuités d'emprunts et le revenu disponible.

\subsubsection{Bénéfice}

Le bénéfice ou résultat d'exercice correspond à la valeur dégagée par l'entreprise une fois toutes les charges et annuités déduites du chiffre d'affaire. Dans notre cas, il s'élève à :

\noindent\fbox{\parbox{\linewidth\fboxrule-2\fboxsep}{
\textbf{Bénéfice} = Chiffre d'affaire - charges variables - charges de structure - amortissement - annuités  = \textbf{30 561\euro{}}}}

C'est un bénéfice élevé au regard du chiffre d'affaire : la rentabilité est de 39\%.

\subsubsection{Revenu disponible}

Les annuités ont été estimées à environ 10 000\euro{}/an (cf tableau  \ref{tab:annuites}). Déduites à l'EBE, on trouve alors un revenu disponible de :

\noindent\fbox{\parbox{\linewidth\fboxrule-2\fboxsep}{
\textbf{Revenu disponible = 57 700\euro{} - 10 000 = 47 700\euro{}}
}}

\subsubsection{Capacité d'autofinancement et niveau des prélèvements privés}

Le revenu disponible permet de financer \textbf{les prélèvements privés, à hauteur de 20 000\euro{}/an}. Le reste, 27 700\euro{}, permet de financer des investissements au sein de l'entreprise. Cette somme permet largement de couvrir la dépréciations du capital mesurée par les charges d'amortissement, ce qui se traduit par une excellente capacité d'autofinancement :

\noindent\fbox{\parbox{\linewidth\fboxrule-2\fboxsep}{
\textbf{Capacité d'autofinancement = (Revenu disponible - prélèvements privés)/amortissements = 175\%}
}}

\subsection{Conclusion}

A terme, les indicateurs économiques de l'activité en régime de croisière sont plutôt bons : \textbf{la rentabilité élevée (39\%) permet d'extraire des prélèvements privés de 20 000\euro{} tout en ayant une très bonne capacité à investir}. Ce résultat est établi sur une hypothèse majeure : la production de pain est de 300kg par semaine par une personne, ce qui nécessite une organisation pour la fabrication du pain extrêmement minutieuse. 

Nous allons étudier à la partie \ref{part:tps_travail} la faisabilité, en temps de travail, d'une telle hypothèse. Nous étudierons par ailleurs l'impact sur les résultats économiques d'une production de pain plus faible (manque de débouché, réduction du temps de travail) dans la partie \ref{part:reduc_pain}.

\section{Prévisionnel des quatres premières années}
\label{part:previsionnel}

\subsubsection{Déroulement}

Les chiffres présentés présentés sont les chiffres "en régime de croisière", c'est-à-dire en considérant que la production bat déjà son plein. Cependant, pour être plus réaliste, il faut considérer le démarrage de l'activité, avec une production plus faible et le décalage entre la mise en place des cultures et leur récolte. Ce démarrage est résumé dans le tableau \ref{tab:demarrage_temp} ci-dessous.

\begin{figure}[h!]
\footnotesize
\center
\begin{tabular}{ | p{2,2cm} | p{0,5cm}| p{0,5cm}| p{0,5cm}| p{0,5cm}| p{0,5cm}| p{0,5cm}| p{0,5cm}| p{0,5cm}| p{1.2cm}| p{1.2cm}| }
\hline
	& T1 & T2 & T3 & T4 & T5 & T6 & T7 & T8 & Année 3 & Année 4 \  \\ \hline
	Production pain & 0 & 50\% & 75\% & 75\% & 80\% & 80\% & 100\% & 100\% & 100\% & 100\% \\ \hline
	Production farine & 0 & 50\% & 75\% & 75\% & 80\% & 80\% & 100\% & 100\% & 100\% & 100\% \\ \hline
	Production blé, seigle, engrain & 0 & 0 & 0 & 0 & 0 & 0 & 0 & 80\% & 80\% & 100\% \\ \hline
	Achat grain & Non & Oui & Oui & Oui & Oui & Oui & Oui & Oui & Oui & Non \\ \hline
\end{tabular}
\caption{Détail du démmarage de l'activité les premiers trimestres et les premières années. La production de pain et de farine démarrent progressivement. La première récolte de céréales arrivent en août de la deuxième année, donc il faut compenser par l'achat de grain pour démarrer l'activité de boulangerie.}
\label{tab:demarrage_temp}
\end{figure}

Pour évaluer les conséquences économiques du démarrage, le scénrio de démrrage pourrait être le suivant :
\begin{itemize}

\item[$\blacklozenge$] Démarrage de l'activité le 1er janvier. Le premier trimestre après le démarrage est consacré à la construction de l'aménagement de la boulangerie et des bâtiments, de la construction des silos, de la recherche de débouchés commerciaux.

\item[$\blacklozenge$] Avant les cultures d'hiver, semées en octobre, on plante des prairies temporaires pour amender le sol. Les cultures de printemps sont semées (prairie semée, soja, sarrasin) comme prévu, fin avril. On ajoute un sarrasin dans l'assolement cet années à la place de l'engrain précédent le soja. Le besoin en semence est estimé à 3 550\euro{}.

\item[$\blacklozenge$] La production de pain et de farine démarre au deuxième trimestre, à 50\% des capacités pour s'adapter aux contraintes de fabrication et de débouchés, soit 150kg/semaine. Pour cela, on achète le grain nécessaire pour un montant de 3 750\euro{}.

\item[$\blacklozenge$] On monte progressivement en puissance la porduction de pain pour atteindre la production maximale au troisième trimestre de la deuxième année. Il faut acheter du grain en attendant les récoltes qui n'arrivent qu'en août de la deuxième année.

\end{itemize}

\subsubsection{Impact économique}

D'un point de vue économique, l'impact du démarrage est décrit dans le tableau \ref{tab:demarrage_eco}. La faible production de la première année et le besoin important en grain et en semences pour le démarrage implique de faibles produits et des charges importantes, qui s'améliore dès la deuxième année.

\begin{figure}[h!]
\footnotesize
\center
\begin{tabular}{ | p{3cm} | p{1,2cm}| p{1,2cm}| p{1,2cm}| p{1,2cm}| }
\hline
	& Année 1 & Année 2 & Année 3 & Année 4 \  \\ \hline
	Produits & 34 895 & 67 506 & 78 196 & 78 364 \\ \hline
	Charges & 21 154 & 26 330 & 22 362 & 20 641 \\ \hline
	\textbf{EBE} & \textbf{13 741} & \textbf{41 176} & \textbf{55 834} & \textbf{57 723} \\ \hline
	\hline
	Annuités & 12 502 & 12 503 & 9 495 & 9 495 \\ \hline
	Revenu dispo. & 1 239 & 28 673 & 46 339 & 48 228 \\ \hline
	\textbf{Prélèvements privés} & \textbf{1 239} & \textbf{12 897} & \textbf{20 000} & \textbf{20 000} \\ \hline
	\hline
	\textbf{Résultat} & \textbf{\textcolor{red}{-14 537}} & \textbf{0} & \textbf{\textcolor{green}{10 563}} & \textbf{\textcolor{green}{12 452}} \\ \hline
	Actif & 199 046 & 199 046 & 209 609 & 222 061 \\ \hline
\end{tabular}
\caption{Prévisionnel des quatre premières années. L'exercice de l'année 1 est fortement déficitaire du fait de la faible production et des besoins élevés en grain et en semences. On se retrouve en "régime de croisière" dès l'année 4, avec des exercices positifs.}
\label{tab:demarrage_eco}
\end{figure}

La première année, l'EBE est faible (13 000\euro{}), à peine supérieur aux annuités (12 500\euro{}). La ferme dégage une trésorerie de 1 239\euro{}, permet de me payer (maigrement) mais ne permet pas de recouvrir l'amortissement du matériel : l'exercice est donc fortement déficitiaire, de plus de 14 000\euro{}. L'impact de cet exercice négatif est limité, car il se fait au détriment de la dotation aux amortissement, je n'ai donc pas besoin d'emprunter pour avoir de la trésorerie.

Dès l'année 2, l'EBE augmente fortement, permettant de recouvrir les annuités d'emprunts, la dotation aux amortissements et de dégager un revenu de 12 900\euro{}, encore inférieur au SMIC. L'entreprise ne perd cependant pas de valeur.

Les années 3 et 4 bénéficient d'une augmentation dela production, d'une diminution des charges (il n'y a plus besoin d'acheter du grain), les emprunts à court terme sont remboursés. Il est possible d'atteindre le revenu de 20 000\euro{} annuel tout en dégageant plus de 10 000\euro{} sur les exercices, compensant la perte engendrée la première année.

\section{Analyse des performances}

\subsection{Rentabilité des différents produits}

Il peut être intéressant de calculer le chiffre d'affaire généré par chaque culture, en regard de la surface utilisée. Pour attribuer un chiffre d'affaire à une culture, on retient le chiffre d'affaire d'une production de pain au regard de la proportion de farine de cette culture\footnote{Exemple : le pain de campagne contient 86\% de farine de blé et 14\% de seigle, 86\%  du CA généré par le pain est attribué au blé et 14\% au seigle.}. Les résultats sont présentés dans le tableau \ref{tab:renta_prod}.

\begin{figure}[h!]
\footnotesize
\center
\begin{tabular}{ | p{2cm} | p{1.5cm}| p{1.8cm}| p{1.8cm}| p{1.8cm}| }
\hline
	& Surface (ha) & CA global par hectare & CA pain par hectare & CA farine par hectare \  \\ \hline
	Blé & 4.75 & 9 439.12 & 11 317.31 & 2 822.4 \\ \hline
	Seigle & 0.75 & 6 864.72 & 7 996.8 & 2 763.6 \\ \hline
	Petit épeautre & 2 & 5 798.54 & 7 060.47 & 2 650.7 \\ \hline
	Sarrasin & 2.5 & 3 423.17 & 5 587.86 & 2 145.89 \\ \hline
	Soja & 2.5 & 1 400 & - & - \\ \hline
	Prairie & 7.5 & 333.33 & - & - \\ \hline
\end{tabular}
\caption{Description des charges d'amortissement. L'amortissement du matériel agricole s'élève à 6 476\euro{}.}
\label{tab:renta_prod}
\end{figure}

Bien évidemment, le pain permet de très bien valoriser les cultures : on multiplie par 3 le chiffre d'affaire des farines. Il est intéressant de noter que si \textbf{les farines et les pains à base de blé sont les moins chers, ce sont les plus rentables à l'hectare}. Cela s'explique en grande partie le rendement bien supérieur en farine du blé par rapport aux autres cultures. 

N'oublions pas que les cultures moins rentables, le sarrasin et le petit épeautre, restent indispensables pour proposer une gamme diversifiée aux clients et pour le bon équilibre agronomique de la rotation des cultures.

\subsection{Faut-il acheter son matériel agricole ?}

Dans la partie \ref{part:W_sol}, le temps d'utilisation du tracteur a été estimé à une soixantaine d'heures par an, ce qui est faible, mais relativement courant pour une vingtaine d'hectares. Sachant que ces machines sont plutôt dimensionnées pour fonctionner environ 600h/semaine, le tracteur acheté pour ce projet peut paraître "un luxe". Néanmois, c'est une préférence personnelle, pour que le projet ne se résume pas à uniquement à la meunerie et à la boulangerie.

On peut se poser la question de la pertinence économique de l'achat du matériel agricole, estimé à 48 000\euro{}. Cet achat implique le remboursement d'annuités et des provisions pour charges d'amortissements, respectivement à hauteur de 4 000\euro{} et 6 000\euro{}. L'utilisation de ce matériel revient à 1 100\euro{} et nécessite tout de même 3 000\euro{} de prestation (coupe et moisson). \textbf{La possession d'un tracteur revient alors à 13 000\euro{}/an, alors qu'en sous-traitant l'ensemble du travail agricole, le coût de la prestation a été estimé à environ 7600\euro{}/an.} 

Sous traiter l'ensemble des travaux agricoles permet \textbf{une économie de l'ordre de 5 500\euro{}/an : c'est une piste intéressante pour réduire les charges du projet}, si par exemple la surface du foncier à acheter est inférieure à celle envisagée, ou que les débouchés commerciaux sont incertains. 

Il peut être aussi intéressant de \textbf{différer l'achat de ce matériel}, en l'achetant après la troisième année d'installation, pour favoriser l'investissmeent dans du matériel de qualité en boulangerie. Une autre piste intéressante serait d'envisager à la \textbf{participation en CUMA} : le matériel en commun permet de l'utiliser à son dimensionnement prévu et de diminuer les coûts.



\subsection{Quelle est la quantité minimale de pain à produire pour atteindre un revenu de 20 000\euro{} annuel ?}
\label{part:reduc_pain}

Nous avons présenté le projet jusqu'ici en nous basant sur une hypothèse de production de pain de 300kg par semaine, ce qui est une quantité importante pour une personne. Cette quantité permet de dégager 20 000\euro{} de revenus et 10 000\euro{} de trésorerie pour la ferme chaque année. Cet objectif de 300kg est atteignable sous réserve d'une bonne organisation, mais quel est l'impact sur la ferme si celui-ci n'est pas atteint ? 

Pour cela, nous avons regardé en figure \ref{fig:renta} l'évolution du bénéfice, ou résultat, en fonction de la quantité de pain produite chaque semaine.

\begin{figure}[h!]
\begin{center}
	\includegraphics[scale=0.5]{revenu_prod_pain.png}
	\caption{En bleu : résultat (ou bénéfice) en fonction de la quantité de pain produite. La barre rouge (respectivement jaune) représente le seuil d'un revenu de 20 000\euro{}/an (respectivement du SMIC), il est atteint pour une production de 212kg (respectivement 176kg) de pain par semaine. Le seul de rnetabilité se situe à 82kg de pain par semaine.}
	\label{fig:renta}
\end{center}
\end{figure}

On remarque que \textbf{le seul de rentabilité}, c'est-à-dire lorsque les produits couvrent parfaitement les charges, \textbf{est à 82kg de pain}. Si toute la production est transformée est farine (c'est-à-dire qu'il n'ya pas de production de pain), l'exercice est négatif de 12 000\euro{}. La production est donc indissociable du projet.

\textbf{Le SMIC, ou 14 400\euro{} annuels, est atteint pour une production de 176kg hebdomadaire. Le seuil des 20 000\euro{} nécessite lui 212kg.} Ces quantités sont tout à fait atteingables pour une personne, en faisant deux fournées par semaine. La capacité de l'entreprise à investir au-delà des charges d'amortissement est limitée mais reste envisageable.

Ce graphe met en lumière quelques spécificités du système paysan-boulanger :
\begin{itemize}

	\item[-] L'activité "pain" est très rentable, puisqu'elle rapporte 154\euro{}/an supplémentaire pour chaque kilo de pain fabriqué chaque semaine.
	\item[-] Les investissements sont élevés (achat foncier, matériel agricole et de boulangerie) justifiant une quantité minimale assez importante pour atteindre le seuil de rentabilité et le SMIC.
	\item[-] Une fois ces seuils atteints, le résultat augmente très vite avec la production.

\end{itemize}

Pour conclure, le système laisse une relative marge de maneuvre pour en vivre, malgré les investissements élevés. Sous résevre d'assurer les débouchés commerciaux, il peut être intéressant d'investir plus dans l'outil de production (notamment le four) pour assurer une grosse production par une personne.  

\section{Temps de travail}
\label{part:tps_travail}

48 smeaines par an
45h par semaine
2 fournée de 8h, soi t150 kg.

\section{Choix des statuts juridique, social et fiscal}

\end{document}
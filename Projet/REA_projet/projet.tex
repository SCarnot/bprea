\documentclass{book}
 
\usepackage[utf8]{inputenc}
\usepackage[francais]{babel}
\usepackage[T1]{fontenc}      
\usepackage{amsmath}
\usepackage[top=3.5cm, bottom=3cm, left=3.0cm, right=3.0cm]{geometry}
\usepackage{graphicx}
\usepackage{eurosym}
\graphicspath{{figures/}{../figures}}
\usepackage[activate={true,nocompatibility},final,tracking=true,kerning=true,spacing=true,factor=1100,stretch=10,shrink=10]{microtype}
\title{Projet d'installation dans un système paysan-boulanger}
\date{Mai 2020}

\makeindex

\begin{document}

\maketitle

\chapter{Démarche, finalités et objectifs du projet}

\section{Démarche personnelle : des technologies de pointes à la terre}

Le projet de reconversion en paysan-boulanger présenté ici a été le fruit de longues réflexions, qui l'ont construit laborieusement mais consciensieusement. Depuis mon choix de me reconvertir dans l'agriculture, j'ai souvent changé de cap pour en arriver ici aujourd'hui ; et ce ne seront certainement pas les derniers amendements à ce projet. Néanmois, la direction à suivre semble être désormais la bonne : celle-ci s'est fixée autant à partir de mes réflexions que de mon vécu. Je vais ainsi commencer par présenter quelques éléments personnels, qui permettent de comprendre la plupart de mes choix et des orientations de mon projet.\textbf{ Je m'excuse auprès du lecteur pour ces digressions personnelles ; si celui-ci souhaite directement connaître le projet, il peut débuter sa lecture à la partie XX.}

\subsection{Pourquoi cette reconversion agricole ?}

D'où je viens : normale sup, thèse physqiue, IA. Ce que ça m'a apporté 
Mode de vie : ce que j'en ai retenu
Ce que j'ai appris : catastrophe climatique et pénurie des ressources. Comme bcp, j'ai lu les limties de la croissances.
Disonnance cognotive

Comment vivre avec cela avec mon mode de vie parisien ? Des changements certes, mais cosmétiques. Le militantisme ne suffit pas 

Conséquence : je suis intimement convaincu de la nécessité d'un retour massif à la vie de camapgne et nécessairement à l'agriculture. 

\begin{figure}
\begin{center}
	\includegraphics[scale=0.4]{3ydis3.jpg}
	\caption{Parcelle de blé, fin novembre. Le blé a levé, mais on voit de nettes traces d'érosion et de tassement suite à un semis dans des conditions difficiles : terrain mal ressuyé et en pente.}
\end{center}
\end{figure}

\subsection{Pourquoi être paysan-boulanger ?}

\subsubsection{Le maraîchage et ses limites}

De l'impératif à entamer une nécessaire et massive reconversion dans le monde agricole, il nous vient souvent à l'esprit l'image caricaturale du jeune maraîcher subversif au chapeau de paille, cultivant sur une petite surface sans mécanisation, en permaculture, vendant en circuit-court. A la question "que vous inspire l'agroécologie ?", on pense souvent à cette image-là, popularisée et portée par des personnes comme Jean-Martin Fortier, Pascal Poot ou des projets la ferme du Bec Hellouin. 

\begin{figure}[h!]
\centering
\begin{minipage}{.5\textwidth}
  \centering
  \includegraphics[width=0.9\textwidth]{pascal_poot.jpg}
  \label{fig:test1}
\end{minipage}%
\begin{minipage}{.5\textwidth}
  \centering
  \includegraphics[width=0.98\textwidth]{ferme_bec_hellouin.jpg}
  \label{fig:test2}
\end{minipage}
\caption{\textit{(Gauche)} Pascal Poot, maraîcher et semencier, réputé pour les qualités d'adaptation de ses semences et de ses plants. \textit{(Droite)} Charles et Perrine Hervé-Gruyer, de la ferme du Bec Hellouin, popularisée grâce au film \textit{Demain}. La ferme est réputée pour son approche permacultrice du maraîchage.}
\label{fig:test}
\end{figure}

Que cette représentation soit réaliste ou non, j'ai remarqué que c'est celle là qui imprègne l'imaginaire des gens. Modèle pour certains, repoussoir pour d'autre, c'est  cette image en tête avec laquelle j'ai démarré ma reconversion. Je me suis dirigé vers la voie maraîchère, que j'ai pu rejoindre grâce à un heureux concours de circonstances rejoindre la formation TPH en janvier 2019 au CFPPA de Saint-Ismier. 

Si 

Constat : métier passionnant difficile, c'est ce que je veux faire. Mais plusiseurs choses m'interpelle : énormément d'intrants, énromément d'eau, dépendance avec l'exérieur, inadaption au climat. Humainement : beaucoup de milantistisme et de considération avec lesquelles je me suis senti en décalage (subversion,  gourou approche biodynamique). D'autre part est-ce généralisable à grande échelle ? 

Est-ce que les fondements du maraichages seraient la civilisation thermo-industrielle ?

Si l'on prend l'histoire, le maraichage s'est développé au début du 19 siècle lors de l'industrialisation du bassin parisien, possibilité de mécanisation, d'obtenir du verre en grande quantité, du guano. Avant : grand potager ou jardin.

\subsubsection{Des céréales...}

A l'inverse, comment cultivait nos ancètres ? Principakement des céréales et de l'élevag. Plus préciésement rotation triennale : céréale, légumineuse, jachère. 

Pourquoi ? PArce que c'est le plus adapté à notre climat !

A cette même période, j'ai pu décourvir les cérélaes métier de paysan boulanger. Pourquoi les céréales ?
Le blé est un symbole de l'agriculture : c'est la céréale reine pour la panification, aliment pilier de la culture Française. Le blé est transformé en farine, avec laquelle on fabrique les denrées alimentaires de base. Cette céréales est historiquement une des sources caloriques élémentaires dans la société française. Ainsi, si dans l'imaginaire collectif, la transition agroécologique est représentée surtout sur la production maraichère, elle doit, ou devra, impérativement se focaliser sur la culture des céréales qui sont la base de notre alimentation à grande échelle. Enfin, si la culture du blé et céréalière en général n'est pas très rentable (entre 100 et 1000\euro{} par hectare en AB, à comparer aux rendements de 30000\euro{}/ha en maraichage), le blé peut être très bien valorisé à travers la panification, qui décuple la valeur ajoutée. 

En terme agronomique, l'intérêt du blé (et les grandes cultures en général), en comparaison au maraîchage, est que le temps dédié aux cultures est très faible (de l'ordre d'une dizaine d'heures par an et par hectare), que les variétés sont très adaptées aux climats locaux et ne nécessitent pas systématiquement d'irrigation. En revanche, la mécanisation est indispensable au regard des surfaces travaillées. 

\subsubsection{A la boulangerie}

\subsection{Démarche}

Pas peur de lâcher de la thune à l'investissement pour être le plus efficacee et produire le plus par UTH

\section{Présentation globale du projet}

\subsection{Conditions pédo-climatiques}

\chapter{Culture des céréales}

\section{Contraintes du système paysan-boulanger}

\subsection{Place minimale des céréales dans la rotation}

Un système paysan-boulanger se distingue des grandes cultures par la transformation en pain d'une partie de ses cultures, permettant de capter une grande partie de la valeur ajoutée lors de la vente de ses produits : en bio, le pain est vendu aux alentours de 5\euro{}/kg, soit 5 000\euro{} la tonne, tandis que le grain de blé panifiable s'échangent à 400\euro{} la tonne. Par ailleurs, la législation limite chez les paysans boulanger à hauteur de 30\% l'achat de grain extérieur par rapport à la consommation totale nécessaire en boulangerie. 

En conséquence, les céréales vouées à la transformation (blé, seigle, ...) doivent représenter un minimum de 1/3 dans l'assolement pour assurer une bonne santé financière à la ferme. C'est un délicat équilibre entre la nécessaire diversité dans la rotation et les impératifs économiques.

\subsection{Surface minimale d'installation}

La valorisation en pain permet de décupler le chiffre d'affaire à l'hectare\footnote{En AB, on peut atteindre $\sim$12 000\euro{}/ha par le vente de pain contre $\sim$1 200\euro{} par la vente de grain, avec des rendement de 30qx/ha et un rendement de transformation en pain de 1 (1 kg de grain = 1 kg de pain).}. Si la surface d'un système paysan-boulanger est donc plus petite qu'une installation en grande culture, il faut cultiver un minimum de surface pour générer une activité économiquement viable. Etant donné des investissements lourds et indispensables, il faut viser au minimum un CA de l'ordre de 50 000\euro{} pour une installation individuelle. \textbf{La surface minimale est donc d'une quinzaine d'hectares, en prenant en compte l'assolement (seul 1/3 à la moitié de la surface est dévolu à la culture des céréales).} 

\subsection{Nécessité de la mécanisation}

Dans un système grande culture, il est impossible d’intervenir manuellement ou ponctuellement sur les culture du fait de l’importante taille des parcelles, contrairement au maraîchage, où les interventions manuelles sont possibles. Il faut donc pouvoir gérer l’enherbement, la fertilisation ou l’impact des ravageurs à grande échelle, avec des outils mécanisés. Même sur \textbf{une quinzaine d'hectares, ce quiune surface reste modeste, cultivables nécessitent impérativement d'être mécanisés}, bien que ce soit une limite à l'approche agroécologique décrite plus haut. Dans notre système socio-économique, il est tout à fait inenvisageable, pour le moment, de se baser sur la traction animale. 

Il y a deux possibilités : acheter le matériel adéquat ou externaliser ce travail en prestations. Dans le premier cas, \textbf{l'investissement est conséquent, de l'ordre de 40 à 50 000\euro{}} pour un tracteur de 90ch et les outils du sols d'occasion, mais permet de maitriser les cycles de production. C'est un choix intéressant économiquement par rapport à la prestation dans le cas d'une CUMA. \textbf{Le travail cultural réalisé en totalité en prestation revient peu ou prou à 400\euro{} par hectare}, soit 8000\euro{} de charges annuelles pour 20 hectares.

\section{Choix et démarche}

Agroécologie : durable, donc généralisable.
Agroécologie la démarche doit être généralisable : il fau tune autosuffisance ou ce qui est produit à un endroit soit consommé à un autre et inversement.
Ce sont des choix que je m'impose en plus des pratiques nécessaire à avoir en AB (cf la rotation plus bas) : rien ne 'moblige au couver tpermnent ou au minimum de labour.

\subsection{Autonomie agronomique}

Pour une ferme suivant une démarche agroécologique, il faut pouvoir \textbf{réduire au strict minimum les intrants, en plus particulièrement ceux issus de ressources fossiles}. En effet, dnas cette démarche durable, il faut donc s'affranchir au possible de ces intrants non renouvelables, tels les produits pétroliers (carburants, plastiques, engrais et chimie de synthèse) mais aussi des engrais minéraux, comme le Patentkali$^\mathrm{\textregistered}$\footnote{Cet engrais minéral "\textit{est issu des dépôts naturels de sels en Europe, formés par l’évaporation de l’eau de mer il y a plusieurs millions d’années}" d'après le producteur. C'est donc une ressource non renouvelable.}. \textbf{D'autre part, sur les ressources renouvelables (fumier, paille, grain), le bilan global de la ferme doit être neutre}, c'est-à-dire qu'il doit sortir autant de ressources qu'il n'en rentre, afin d'assurer l'équilibre écologique. Il convient aussi de produire le maximum de ces ressources sur place, pour limiter leur transport entre deux fermes.

\paragraph{Conséquences :} La chimie et les engrais de synthèse sont prohibés ; d'autre part l'usage du plastique est fortement restreint en grandes cultures. Les efforts doivent donc se concentrer sur la consommation de carburant, inévitable en grande culture, \textbf{en limitant au maximum les travaux du sol énergivores}, comme le labour. Et \textbf{d'un point de vue agronomique il faut aussi veiller à ce que les exportations (grain, foin, paille) compensent les apports de la vie du sol ou de fumier.}

\subsection{Vie du sol et biodiversité}

Les systèmes de grandes cultures conventionnels ont certes permis d'assurer une production croissante au cours de la seconde moitié du XX$^{\mathrm{i\grave{e}me}}$ siècle, mais cette croissance s'est basée sur une exploitation minière sur la qualité agronomique des sols\footnote{Les pratiques agricoles érodent les sols, jusqu'à des proportions inquiétantes dans le monde entier, y compris en France. Par exemple, le labour et des sols nus permettent aux intempéries de faiire ruisseler les particules de terre argileuse, qui ont mettent des dizaines de milliers d'années à se former, vers les fleuves, et \textit{in fine}, à l'océan.} et surtout grâce à l'utilisation massive d'engrais azotés, d'origine fossile\footnote{Les engrais azotés de synthèse, à l'orgine de l'explosion des rendements agricoles, sont produits grâce au procédé de Haber-Bosch, qui permet de fixer l'azote atmosphérique à partir du craquage du méthane, issus des gisements de pétrole ou de gaz naturel.}. Non durables, ces méthodes, associées à l'utilisation massive de pesticides, ne sont pas sans conséquences écologiques : contribution au réchauffmeent climatique, appauvrissement de la vie du sol, diminution du taux de matière organique dans les sols, effondrement de la biodiversité générale...

Pour être durable, \textbf{un système de grandes cultures doit entrer dans une démarche globale et nécessaire de préservation de l'environnement}, au prix, hélas, d'une diminution des rendements\footnote{Les rendements en AB de blé tendre dans la région Rhône-Alpes sont compris entre 25 et 35qx/ha. Pour comparaison, ils sont à 60 qx/ha en conventionnel dans la région, et à 40 qx/ha en moyenne en AB en Île-de-France. Source : \textit{Rotations pratiquées en grandes cultures biologiques en France : état des lieux par région, ITAB, 2011}}. Le nécessaire passage en agriculture biologique implique, du fait du cahier des charges, à favoriser une rotation diversifiée et à implanter des couverts d'interculture dès que le sol est à nu. Il est possible d'aller plus loin en limitant les exportations de pailles et de chaumes ou en rétablissant des haies naturelles entre les parcelles cultivées.

\textbf{Les intérêts agronomiques de ces pratiques sont multiples}. L'augmentation du taux de matière organique stimule la vie du sol en quantité et en qualité, améliore la qualité des sols par une meilleure structuration, quelque soit sa texture et \textit{in fine} améliore la santé globale des végétaux cultivés\footnote{En agriculture de conservation, le mot d'ordre est de maximiser au possible la vie du sol par la production de matière végétale : les sols sont toujours couverts, les agriculteurs suivant cette démarche constatent une diminution continue d'herbicides et de fongicides, preuve de l'amélioration de la santé végétale de leurs cultures, grâce à une vie du sol riche.}. Une biodiversité plus riche est aussi un sytème écologique plus équilibré, permettant de limiter l'impact des maladies et des ravageurs.

Au-delà de ces intérêts purement agronomiques, il est aussi essentiel de \textbf{rétablir une certaine beauté dans le paysage}, avec des champs aux cultures variées, des bandes fleuries, des haies structurant le paysage et une présence de diversité animale et végétale.

\paragraph{Conséquences :} En AB, comme l'utilisation des pesticides est interdite, préserver la vie du sol et la biodiversité implique principalement de \textbf{réduire le labour au strict minimum, c'est-à-dire ne le faire que lorsque l'enherbement devient incontrôlable}. Le labour, en retournant les horisons du sol, est particulièrement nocif pour tous les organismes du sol et favorise l'érosion. Il faut aussi pouvoir implanter un couvert végétal dès que le sol reste à nu plus de trois mois durant. Une autre possibilité est le rétablissement de haies entre au bord des parcelles. 

Il faut garder à l'esprit que ces différentes pratiques ont, à court terme, un coût direct (implantation ses couverts, diminution des rendements sans engrais azotés) et indirect (allongement de la rotation au profit de cultures moins rentables). Néanmoins, ces "coûts" ne le sont pas vraiment, car il s'agit plutôt de ne pas dilapider le capital naturel et de préserver une production à très long terme.


\section{Rotation}

Les choix agroécologiques et les contraintes propres aux grandes cultures, définis précédemment, peuvent être en grande partie respectés avec une rotation de culture bien adaptée. Définir une rotation avisée est primordial, car elle détermine totalement la production de grain et les charges culturales, et, détermine en partie la gamme de produits et le temps de travail. Dans cette partie, on présente les choix des différentes cultures et leur place dans l'assolement.

\subsection{Importance de la rotation en grande culture}

En grandes cultures biologiques, les deux contraintes forte en AB sont la fertilisation et la gestion de l'enherbement. La rotation des cultures permet de répondre à ces contraintes, déjà présente au Moyen-âge où l'on pratiquait déjà des rotations triennales, en alternant céréales, légumineuses et jachères. Conformes à cet esprit originel, celles pratiquées aujourd'hui sont plus élaborées, parfois jusqu'à 12 années successives. Les principaux critères d'une rotation sont :
\begin{itemize}

	\item[$\clubsuit$] Alterner les familles botaniques permet de casser le cycles de ravageurs et de varier les besoins en fertilisation. Les cultures appartiennent le plus souvent aux poacées (ou graminées, ou encore céréales) et les fabacées (ou les légumineuses). 

	\item[$\clubsuit$] Alterner les cultures exigeantes (blé, maïs, colza) et les légumineuses (luzerne, soja, pois), qui en fixant l'azote atmosphérique, fertilise le sol. 
	
	\item[$\clubsuit$] Alterner les cultures de printemps et d'hiver permet de casser le cycle des adventices annuelles.
	
	\item[$\clubsuit$] Respecter les temps de non-retour, de trois ans pour l es légumineuses jusqu'à 6 ans pour les crucifères. Dans le cas des céréales, on évite de cultiver la même céréales d'une année sur l'autre : par exemple, la succession blé $\rightarrow$ seigle présente beaucoup moins de risque que blé $\rightarrow$ seigle.
	
	\item[$\clubsuit$] Une période de jachère sur deux ou trois ans en tête de rotation permet de contrôler l'enherbement, et donc les interventions mécaniques. Elle peut être constituée d'une prairie semée avec une légumineuse, comme de la luzerne ou du trèfle.

\end{itemize}

Il existe encore de nombreux critères, plus spécifiques aux successions d'une espèce par rapport à une autre. Evidemment, une rotation doit aussi prendre en compte les contraintes spécifiques à la ferme : présence ou non d'irrigation, conditions pédo-climatiques, impératifs économiques, etc.

\subsection{Rotation choisie}

Au vu de mes besoins en céréales et des contraintes agronomiques, je choisis une rotation sur 8 ans, avec 1/4 de blé, 1/8 de seigle/engrain, soja, sarrasin et une prairie luzerne-graminées en tête de rotation :
\begin{align*}
\left\langle \right. \mathrm{Prairie\ sem\acute{e}e}  \left\rangle \right._{N+1\rightarrow N+3}  \longrightarrow
\left\langle \right. \mathrm{Bl\acute{e}}  \left\rangle \right._{N+4}  
\longrightarrow
\left\langle \right. \mathrm{Seigle,\ engrain}  \left\rangle \right._{N+5} \\
\longrightarrow 
\left\langle \right.\mathrm{Soja}\left\rangle \right._{N+6}
\longrightarrow
\left\langle \right. \mathrm{Bl\acute{e},\ seigle} \left\rangle \right._{N+7}
\longrightarrow
\left\langle \right. \mathrm{Sarrasin} \left\rangle \right._{N+8}
\end{align*}

Cette rotation est relativement "classique" en grandes cultures bio, avec un choix marqué pour des cultures panifiables : blé, seigle, engrain et sarrasin. Elle permet de respecter les principaux critères définis précédement. 

\begin{figure}[h!]
\begin{center}
	\includegraphics[scale=0.35]{rotation_projet_cercle.pdf}
	\caption{Parcelle de blé, fin novembre. Le blé a levé, mais on voit de nettes traces d'érosion et de tassement suite à un semis dans des conditions difficiles : terrain mal ressuyé et en pente.}
\end{center}
\end{figure}

\subsection{Détail de la rotation}

\subsubsection{Années 1$\rightarrow$3 : Prairie semée (luzerne et graminées)}

\paragraph{$\circ$ Rôle} La tête de rotation est une prairie semée composée principalement de luzerne et de graminées (fétuque, ray-grass; dactyle). La luzerne enrichit en azote le sol, le structure et nettoie le champ des vivaces. Une durée trois ans assure un nettoyage efficace, mais ne peut être prolongé au-delà : les pieds de luzerne se fatiguent et prend une trop grande proportion dans la rotation. Les graminées, en plus petite proportion, permettent de diversifier l'exploration racinaire du sol et favorise une valorisation directe en fourrage pour de l'élevage. On prend des espèces qui ne sont pas utilisées dans le reste de rotation pour éviter les maladies.

\begin{figure}[h!]
\begin{center}
	\includegraphics[scale=0.3]{luzerne.jpg}
\end{center}
\end{figure}

\paragraph{$\circ$ Conduite} Après le déchaumage du sarrasin restant, on réalise au pirntemps un faux semis permettant de préparer le sol pour le semis, qui se fait à une dose de 20kg/ha pour la luzerne et 8kg/ha pour les graminées. Les deux premières années de prairie seront pourront être utilisées pour du fourrage, soit par coupe, soit en faisant paître des bêtes directement dessus. L'exportation de et d'éléments minéraux sera en partie composée par du fumier ajouté. La dernière année, la prairie sera simplement broyée, d'une part pour limiter les exportations de MO, d'autre part pour favoriser sa minéralisation avant la culture du blé suivante. La prairie est détruite par un labour à la mi-septembre de la troisième année.

\paragraph{$\circ$ Valorisation} Principalement agronomique, pour reposer le sol et le recharger en azote. La production de fourage est estimée à 10t de matière sèche par hectare pour 2 années sur 3, qui peut être vendue en fourrage à 50\euro{}/t. Les débouchés pour l'élevage en région grenobloise ne sont a priori pas un problème. 

\subsubsection{Années 4 : Blé}

\paragraph{$\circ$ Rôle} Pour valoriser au mieux l'azote restituée par la luzerne de la prairie, on place une culture 100\% blé à sa suite. C'est la culture clé de la rotation.

\begin{figure}[h!]
\begin{center}
	\includegraphics[scale=0.1]{triticum_aestivum.jpg}
\end{center}
\end{figure}

\paragraph{$\circ$ Variétés} On prendra pour moitié des semences commerciales (Energo) et paysannes (Rouge de bordeaux et Florence Aurore). Les semences commerciales assurent un certain rendement et permettent d'éviter les problème de verse par leur hauteur de taille limitée. Les semences paysannes, quant à elles, sont plus incertaines en rendement et on risque avéré de verse, mais leur diversité génétique assurent aussi une certaine résistance face aux ravageurs. Le mélange de ces deux types de semences permet de moyenner les avantages et les inconvénients. 

\paragraph{$\circ$ Conduite} On réalise un ou plusieurs faux semis de début octobre en passant la bineuse rapidement dans le champ, de préférence avant une pluie, pour favoriser la levée des adventices. Le faux-semis permet aussi de préparer le sol pour le semis, réalisé début novembre, à une dose de 200kg/ha. La herse étrille est passée une ou deux fois au printemps, pour réchauffer et réoxygéner le sol, et détruire les pousses d'adventices. Le blé est récolté courant juillet en prestation, estimée à 120\euro{}/ha. 

\paragraph{$\circ$ Valorisation} Meunerie pour produire la farine vendue en direct et transformation en pain.

\subsubsection{Années 5 : Seigle \& engrain}

\paragraph{$\circ$ Rôle} Pour valoriser au mieux l'azote restituée par la luzerne de la prairie, on place une culture 100\% blé à sa suite. C'est la culture clé de la rotation. Les pailles sont exportées pour être valorisées et limiter le risque de maladie pour les céréales suivantes.

\paragraph{$\circ$ Conduite} Identique au blé. Les doses de semis sont respectivement de 120 et 150kg/ha pour l'engrain et le seigle. L'engrain est semé à la mi-octobre ; son cycle est plus long que celui du blé ou du seigle.

\paragraph{$\circ$ Valorisation} Meunerie pour produire la farine vendue en direct et transformation en pain. Les pailles sont valorisées en litière pour l'élevage. 

\subsubsection{Année 6 : Soja}

\paragraph{$\circ$ Rôle} Le soja est cultivé après les deux pailles précédentes. Son rôle de légumineuse est donc de fournir de l'azote par fixation au sol et de couper le cycle des céréales. C'est une culture de printemps qui permet de casser le cycle des adventices d'hiver, favorisées par les cultures précédentes. On peut envisager, en fonction du climat, d'autres légumnieuses telle le pois, la lentille ou le pois chiche, dont les conduites sont proches. Dans l'idéal, on pourra privilégier une valorisation à destination humaine.

\begin{figure}[h!]
\begin{center}
	\includegraphics[scale=0.3]{glycine_max.jpg}
\end{center}
\end{figure}

\paragraph{$\circ$ Conduite} Le soja est précédée d'une interculture d'avoine de printemps et de phacélie, qui seront potentiellement faibles après l'hiver, qu'on déchaumera au mois de mars. Deux faux-semis sont réalisés entre mars et avril pour préparer le sol et détruire les adventices concurrentes. On sème durant à la fin du mois d'avril dans un sol réchauffé à une dose de 100kg/ha. Les résidus de cultures (des tourteaux principalement) sont laissés sur place après le déchaumage fin septembre, début octobre.

\paragraph{$\circ$ Valorisation} Vente directe en coopérative à 700\euro{}/t pour l'alimentation humaine.

\subsubsection{Années 7 : Blé \& seigle}

\paragraph{$\circ$ Rôle} Pour valoriser l'azote restituée par la légumineuse précédente, on place une culture blé et de seigle à sa suite. La proportion de seigle dépendra des besoins en boulangerie. On replace une culture d'hiver après la culture de printemps, pour casser de nouveau le cycle des adventices. Le blé et le seigle ont un potentiel rôle nettoyant au printemps.

\paragraph{$\circ$ Conduite} Conduite identique à celle du blé de l'année 4. Il est semé après le déchaumage du soja, début novembre.

\paragraph{$\circ$ Valorisation} Meunerie pour produire la farine vendue en direct et transformation en pain.

\subsubsection{Années 8 : Sarrasin}

\paragraph{$\circ$ Rôle} Culture de fin de rotation, qui n'est pas exigeante, nettoyante, rustique. Elle permet de valoriser les reliquats d'azote et de fourniture en matière organique en toute fin de rotation. De la famille des \textit{hydrophylacées}, le sarrasin permet d'ajouter une culture autre que les céréales et les légumineuses dans la rotation. C'est une culture très nettoyante, permettant de limiter l'enherbement en fin de rotation. Ne supportant pas la moindre gelée, il est donc semé tard, au plus tôt à la fin avril.

\paragraph{Variétés} Acheté la 1$^{\mathrm{\grave{e}re}}$ année, puis ré-ensemençage avec les semences récolotées. Au vu des contraintes de la ferme et des conditions climatiques, il serait intéressant d'essayer la variété \textit{Petit Gris}. C'est la variété la plus utilisée en France avec \textit{La Harpe}, en Bretagne principalement ; il est donc aisé de trouver un revendeur. D'autre part, c'est l'une de variété les plus productives et qui est totalement adaptée à la production de farine, qui est notre débouché principal (pour la vente en direct etla transformation en pain). Il est aussi reconnu pour ses qualités gustatives. Son cycle n'est ni tardif, ni précoce, de 100 à 120 jours. Il peut s'insérer dans la rotation s'il n'est pas semé trop tard (fin avril), pour pouvoir implanter la prairie semée mi-octobre.

\begin{figure}[h!]
\begin{center}
	\includegraphics[scale=0.4]{sarrasin.jpg}
\end{center}
\end{figure}

\paragraph{$\circ$ Conduite} Le sarrasin est implanté après une interculture de moutarde et de tournesol, qui seront affaiblis par l'hiver et déchaumés d'ébut mars, en fonction des conditions climatiques\footnote{Ou même en début d'hiver s'il y a un risque de montée en graine pour la moutarde.}. Le semis étant tardif, au début du mois de mai, le sarrasin peut être concurrencé par les adventices en début de cycle : deux faux-semis seront réalisés durant le printemps avant le semis. Rien n'est fait jusqu'à la récolte, réalisée lorsque 3/4 des grains sont mûrs\footnote{Contrairement aux autres cultures, la floraison du sarrasin est indéterminée et les grains ne sont pas mûrs au même moment.}, à partir de mi-septembre. Il est déchaumé 

\paragraph{$\circ$ Valorisation} Meunerie pour produire la farine vendue en direct et transformation en pain. Attention, les rendements sont très dépendants des conditions météorologiques au moment de la floraison, donc très variables. La floraison est par ailleurs très intense, il peut être possible de faire installer des ruches.

\subsection{Gestion de l'interculture}

Dans la rotation, il y a deux successions $\left\langle \right. $culture de d'hiver $\left\rangle \right. $ $\longrightarrow$ $\left\langle \right. $culture de printemps$\left\rangle \right. $. Comme la culture d'hiver est récoltée entre mi-juillet et début août et la culture de printemps n'est semée qu'en avril-mai, il y a une période de 9 mois avec un sol qu'avec les chaumes de céréales, mais sans culture vivante. Pour éviter l'apparition d'adventices et d'enrichir le sol avec un couvert végétal vivant, il est agronomiquement très intéressant d'implanter une culture intermédiaire, sans récolte à la suite de la culture d'hiver. 

En effet, implanté en août, ces cultures ont un potentiel de production de biomasse important, permettent de contrôler l'enherbement et de retenir les éléments minéraux dans le sol dans leur organisme. Ces cultures n'ont pas de rentabilité directe, il faut que leur implémentation et leur destruction pour la culture suivante soit simple. Je choisis donc des cultures plutôt sensible au gel ou à un passage mécanique, de sorte à ce qu'elle soit morte en sortie d'hiver ou peu vigoureuse. 

Le premier couvert se situe à la cinquième année :
\begin{align*}
\left\langle \right. \mathrm{Seigle,\ engrain}  \left\rangle \right._{N+5}  \longrightarrow \left\langle \right.\mathrm{Avoine,\ phac\acute{e}lie}\left\rangle \right.\longrightarrow \left\langle \right. \mathrm{Soja} \left\rangle \right._{N+6}
\end{align*}
L'avoine de printemps est une graminée autre que celles présentes dans la rotation, produisant une biomasse intéressante, peu coûteux et étoufant. Il se porte bien en association avec la phacélie, qui a l'avantage d'être une \textit{hydrophyllacée}, une famille qui casse la succession des légumnieuses et des graminées. Les deux espèces ressortiront de l'hiver affaiblies et seront détruites par un roulage ou un déchaumage.

Le second couvert est à la septième année :
\begin{align*}
\left\langle \right. \mathrm{bl\acute{e}}  \left\rangle \right._{N+7}  \longrightarrow \left\langle \right.\mathrm{Moutarde,\ Tournesol}\left\rangle \right.\longrightarrow \left\langle \right. \mathrm{Sarrasin} \left\rangle \right. _{N+8}
\end{align*}
La moutarde (brassicacée) et le tournesol (astéracée) sont des familles absente de la rotation, elles cassent les cycles présents. La moutarde se dévellope très rapidement et est nettoyante. Le tournesol se déveloope rapidement, mais en hauteur. Sa racine pivotante est très intéressante pour la structuration du sol.

\textit{Remarque} : on aurait pu choisir d'implanter une légumineuse afin d'apporter de l'azote supplémentaire dans la rotation. Cependant, le premier couvert étant avant une légumineuse, on évite une succession trop impotante de cette famille. D'autre part, le second couvert est avant un sarrasin, auquel on souhaite éviter un apport trop riche en azote.

\section{Travaux du sol}

La conduite de ces céréales est nécessairement mécanisée

\begin{figure}[h!]
\center
\begin{tabular}{ | p{2cm} | p{2cm}| p{2cm}| p{2cm} | p{1,5cm} | p{2cm} | p{2cm}| }
\hline
	Tâche & Coût par hectare (\euro{}/ha) & Coût prestation par hectare (\euro{}/ha) & Temps par hectare (h/ha) & Temps total (h) & Conso. GNR par hectare (L/ha) & Conso GNR horaire (L/h) \\ \hline
	Herse & 15 & 30 & 0.3 & 4.5 & 3 & 10 \\ \hline
	Coupe & - & 100 & - & - & - & - \\ \hline
	Labour & 50 & 100 & 0.55 & 1.375 & 15 & 27.3 \\ \hline
	Récolte & - & 120 & - & - & - & - \\ \hline
	Semis & 40 & 80 & 1.5 & 30 & 15 & 10 \\ \hline
	Binage & 20 & 40 & 0.5 & 2.5 & 5 & 10 \\ \hline
	Faux-semis & 20 & 40 & 0.5 & 8.75 & 5 & 10 \\ \hline
	Déchaumage & 40 & 80 & 0.6 & 10.5 & 10 & 16.7 \\ \hline
	Total & 57.625 & \  & \  & \  & \  & \  \\ \hline
\end{tabular}
\caption{Interventions nécessaires pour chaque rotation}
\label{tab:compa_rot}
\end{figure}

\begin{figure}[h!]
\center
\begin{tabular}{|p{3cm}|p{2cm}|p{2cm}|p{2cm}|}
\hline
Intervention & $2\times\mathrm{rot._1}$ & $\mathrm{rot._1}$ & $\Delta$ \\
\hline
Moisson &  4 & 4 & 0  \\
\hline
Déchaumage & 4 & 5 & +1 \\
\hline
Semis & 6 & 7 & +1\\
\hline
Broyage & 4 & 4 & 0\\
\hline
Labour & 2 & 2 & 0 \\
\hline
Total & 20 & 22 & +2\\
\hline
\end{tabular}
\caption{Interventions nécessaires pour chaque rotation}
\label{tab:compa_rot}
\end{figure}

\section{Stockage du grain}

\subsection{Céréales}

\subsection{Cas du sarrasin}

\subsubsection{Mécanisation}

L'ensemble de l'outillage mécanique est estimé à 50 000\euro{}, avec un amortissement sur 12 ans, soit 4167\euro{} par an.

On considère que l'entretien et la reparation de ce matériel est de 5\euro{} de l'heure. Il comprend les pneus, l'huile et la réparation des outils. C'est une valeur haute : pour une utilisation de 600h/an, pour un tracteur de 90 ch, il s'établit à 3,3\euro{}/h\footnote{\textit{Coûts des Opérations Culturales 2018 des Matériels Agricoles}, Chambres d'agriculture de France.}. Comme le tracteur est beaucoup moins utilisé (moins de 100h/an), le coût horaire peut augmenter (si par exemple il faut faire la vidange toutes les 400h ou les 2 ans). On se calque donc sur une valeur surestimée.

Le carburant, le gazole non routier (GNR), est pris à 0,85\euro{}/l\footnote{Il s'établissait à 0,85\euro{}/l. Les prix des carburants pétroliers étant très variables, on prend la valeur haute.}. On arrive donc à un coût de 670\euro{}.

Les prestations incluent les coupes et les récoltes, qui représentent 3000\euro{}.

\chapter{Analyse économique}

\section{Processus de fonctionnement global de la ferme}

\section{Bilan de départ}

\section{Produits et charges à terme}

\section{Prévisionnel des trois premières années}

\section{Analyse des performances}

\subsection{Rentabilité des différents produits}

\subsection{Faut-il acheter son matériel agricole ?}

\end{document}
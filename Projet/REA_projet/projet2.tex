\documentclass{article}
 
\usepackage[utf8]{inputenc}
\usepackage[francais]{babel}
\usepackage[T1]{fontenc}      
\usepackage{amsmath}
\usepackage[top=3.5cm, bottom=3cm, left=3.0cm, right=3.0cm]{geometry}
\usepackage{graphicx}
\usepackage{eurosym}
\graphicspath{{figures/}{../figures}}
\usepackage[activate={true,nocompatibility},final,tracking=true,kerning=true,spacing=true,factor=1100,stretch=10,shrink=10]{microtype}
\title{Culture céréalière pour la transformation boulangère : exemple d'un paysan-boulanger dans le pays Voironnais}
\date{Octobre 2019}

\begin{document}

\section{Rotation}

\subsection{Prairie semée : luzerne et graminées}

La luzerne semble la plus adaptée dans une rotation céréalière : elle enrichit en azote le sol, le structure et nettoie le champ des vivaces. Une durée trois ans assure un nettoyage efficace, mais ne peut être prolongé au-delà : les pieds de luzerne se fatiguent et prend une trop grande proportion dans la roation. Elle sera associée à une ou plusieurs graminées (fétuque, dactyle, ray-grass), pour favoriser une utilisation directe en fourrage pour de l'élevage.

Les deux premières années de prairie seront pourront être utilisée pour du fourrage, soit par coupe, soit en faisant paître des bêtes directement dessus. L'exportation de et d'éléments minéraux sera en partie composée par du fumier ajouté. La dernière année, la prairie sera simplement broyée, d'une part pour limiter les exportations de MO, d'autre part pour favoriser sa minéralisation avant la culture du blé suivante.

\subsection{Gestion de l'interculture}

Dans la rotation, il y a deux successions $\left\langle \right. $culture de d'hiver $\left\rangle \right. $ $\longrightarrow$ $\left\langle \right. $culture de printemps$\left\rangle \right. $. Comme la culture d'hiver est récoltée entre mi-juillet et début août et la culture de printemps n'est semée qu'en avril-mai, il y a une période de 9 mois avec un sol qu'avec les chaumes de céréales, mais sans culture vivante. Pour éviter l'apparition d'adventices et d'enrichir le sol avec un couvert végétal vivant, il est agronomiquement très intéressant d'implanter une culture intermédiaire, sans récolte à la suite de la culture d'hiver. 

En effet, implanté en août, ces cultures ont un potentiel de production de biomasse important, permettent de contrôler l'enherbement et de retenir les éléments minéraux dans le sol dans leur organisme. Ces cultures n'ont pas de rentabilité directe, il faut que leur implémentation et leur destruction pour la culture suivante soit simple. Je choisis donc des cultures plutôt sensible au gel ou à un passage mécanique, de sorte à ce qu'elle soit morte en sortie d'hiver ou peu vigoureuse. 

Le premier couvert se situe à la cinquième année :
\begin{align*}
\left\langle \right. \mathrm{Seigle,\ engrain}  \left\rangle \right._{N+5}  \longrightarrow \left\langle \right.\mathrm{Avoine,\ phac\acute{e}lie}\left\rangle \right.\longrightarrow \left\langle \right. \mathrm{Soja} \left\rangle \right._{N+6}
\end{align*}
L'avoine de printemps est une graminée autre que celles présentes dans la rotation, produisant une biomasse intéressante, peu coûteux et étoufant. Il se porte bien en association avec la phacélie, qui a l'avantage d'être une \textit{hydrophyllacée}, une famille qui casse la succession des légumnieuses et des graminées. Les deux espèces ressortiront de l'hiver affaiblies et seront détruites par un roulage ou un déchaumage.

Le second couvert est à la septième année :
\begin{align*}
\left\langle \right. \mathrm{bl\acute{e}}  \left\rangle \right._{N+7}  \longrightarrow \left\langle \right.\mathrm{Moutarde,\ Tournesol}\left\rangle \right.\longrightarrow \left\langle \right. \mathrm{Sarrasin} \left\rangle \right. _{N+8}
\end{align*}
La moutarde (brassicacée) et le tournesol (astéracée) sont des familles absente de la rotation, elles cassent les cycles présents. La moutarde se dévellope très rapidement et est nettoyante. Le tournesol se déveloope rapidement, mais en hauteur. Sa racine pivotante est très intéressante pour la structuration du sol.

\textit{Remarque} : on aurait pu choisir d'implanter une légumineuse afin d'apporter de l'azote supplémentaire dans la rotation. Cependant, le premier couvert étant avant une légumineuse, on évite une succession trop impotante de cette famille. D'autre part, le second couvert est avant un sarrasin, auquel on souhaite éviter un apport trop riche en azote.

\section{Etude économique}

\subsection{Charges}

\subsubsection{Mécanisation}

L'ensemble de l'outillage mécanique est estimé à 50 000\euro{}, avec un amortissement sur 12 ans, soit 4167\euro{} par an.

On considère que l'entretien et la reparation de ce matériel est de 5\euro{} de l'heure. Il comprend les pneus, l'huile et la réparation des outils. C'est une valeur haute : pour une utilisation de 600h/an, pour un tracteur de 90 ch, il s'établit à 3,3\euro{}/h\footnote{\textit{Coûts des Opérations Culturales 2018 des Matériels Agricoles}, Chambres d'agriculture de France.}. Comme le tracteur est beaucoup moins utilisé (moins de 100h/an), le coût horaire peut augmenter (si par exemple il faut faire la vidange toutes les 400h ou les 2 ans). On se calque donc sur une valeur surestimée.

Le carburant, le gazole non routier (GNR), est pris à 0,85\euro{}/l\footnote{Il s'établissait à 0,85\euro{}/l. Les prix des carburants pétroliers étant très variables, on prend la valeur haute.}. On arrive donc à un coût de 670\euro{}.

Les prestations incluent les coupes et les récoltes, qui représentent 3000\euro{}.

\end{document}
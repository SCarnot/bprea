\documentclass{article}
 
\usepackage[utf8]{inputenc}
\usepackage[francais]{babel}
\usepackage[T1]{fontenc}      
\usepackage[top=3.5cm, bottom=3cm, left=3.0cm, right=4.0cm]{geometry}
\usepackage{graphicx}
\usepackage{eurosym}
\graphicspath{{figures/}{../figures}}
\usepackage[activate={true,nocompatibility},final,tracking=true,kerning=true,spacing=true,factor=1100,stretch=10,shrink=10]{microtype}
\title{Culture céréalière pour la transformation boulangère : exemple d'un paysan-boulanger dans le pays Voironnais}
\date{Octobre 2019}

\begin{document}

\maketitle

\section*{Présentation}

Nous nous intéressons ici à la culture céréalière réalisée dans la ferme "La poule aux fruits d'or", un GAEC de trois associés installé à Saint-Étienne de Crossey. Les céréales cultivées sont principalement le blé, le seigle et le petit épeautre, et dans une moindre mesure, le soja et du maïs. Toute la production est en agriculture biologique, et avec pour ligne directrice de réduire au minimum les intrants extérieurs. Les céréales sont directement transformées dans leur activité boulangère ou utilisées en aliment pour leurs poules pondeuses.

\section{Conditions pédoclimatiques}

Il y a deux lots de parcelles, de 24ha au total : environ 22ha dans la plaine de Moirans et 2ha à Saint-Étienne de Crossey. 

\subsubsection*{Climat}

Le climat du pays Voironnais est semi-continental, similaire à celui que l'on trouve dans les "Terres froides" entre Grenoble et Vienne, caractérisé par un été chaud et un hiver rigoureux. La pluviométrie est assez importante, à cause de l'effet Foehn\footnote{L'effet Foehn est un phénomène météorologique créé par la circulation de masse d'air sur un relief. La principale conséquence est l'apparition d'une "zone d'ombre météorologique" : dans l'hémisphère nord, les zones à l'ouest des massifs sont généralement plus humides que les zones à l'est. Par exemple, le pays Voironnais est plus arrosé que le Grésivaudan.} engendré par le massif de la Chartreuse. En conséquent, les parcelles sont particulièrement humides et n'ont pas besoin d'irrigation. Par ailleurs, le climat est suffisamment chaud l'été pour la culture du maïs et du soja.

\subsubsection*{Plaine de Moirans}

La plaine de Moirans est réputée pour avoir des terres plates et fertiles, permettant à l'agriculture conventionnelle d'obtenir des rendements exceptionnels en maïs. Le sol correspond à l'ancien lit alluvionnaire de l'Isère, qui est principalement limoneux. Il présente de nombreux avantages : 
\begin{itemize}
	\item[-] le sol se travaille facilement et est riche ;
	\item[-] plus léger qu'un sol argileux, il se réchauffe vite au printemps ;
	\item[-] la nappe phréatique se trouvant à une faible profondeur, le sol est tout le temps humide, même en sécheresse, ce qui est particulièrement apprécié pour les céréales. Ainsi, même la culture maïs, qui nécessite beaucoup d'eau, n'a pas besoin d'être irriguée ;
	\item[-] les terrains sont plats, facilitant grandement le travail du sol au tracteur. 
\end{itemize}
Cependant, le sol étant limoneux, une croute de battance se forme facilement après une pluie intense, en particulier lorsque les sols sont mis à nus après un labour. Il est nécessaire de passer une griffe, qui permet à la fois de détruire la croute de battance et les adventices concurrentes lors des premiers stades de développement de la céréale. 

%La plupart des surfaces exploitées dans l'environnement proche sont en agriculture conventionnelle : après un labour au printemps, un passage d'herbicide et un apport de fertilisant azoté, du maïs est semé fin avril. Il est récolté en octobre et le sol est la plupart du temps laissé à nu durant l'hiver, avant de repartir sur le même cycle. Cette conduite conventionnelle offre un environnement défavorable pour l'agriculture biologique et l'état d'esprit associé, surtout en terme d'érosion du sol ou de diffusion de phytosanitaire. Cependant, la ferme tire profit de la mise en commun du matériel lourd, qui permet de sous-traiter certaines tâches, comme le labour ou la moisson. 

\subsubsection*{Saint Étienne de Crossey}

Située à 10 km au nord des plaines de Moirans, les terrains à Saint Étienne de Crossey sont quant à eux principalement argileux et en pente douce, au pied d'une colline. Le sol est en permanence humide, en cas de pluie, le sol est rapidement saturé en eau, au point de faire de légères résurgences. 

Ainsi, à l'inverse des plaines de Moirans, ces parcelles sont plus compliquées à travailler : 
\begin{itemize}
	\item[- ] la terre étant plus lourde, elle est plus difficile à travailler et s'accumule rapidement dans une herse rotative ;
	\item[- ] les créneaux pour travailler le sol sont étroits. En effet, le temps de réssuyage est long et les jours de pluie sont nombreux, laissant peut de marge de manœuvre sur les période de travail du sol. 
	\item[- ] l'humidité est parfois telle que le risque d'embourber le matériel est réel. Aussi, de grandes flaques peuvent apparaître est empêcher tout semis, ou travail du sol.
\end{itemize}

\subsubsection*{Comparaison entre les terrains}

Si les terres de Saint-Étienne de Crossey sont plus techniques à travailler, elles restent néanmoins assez productives, avec des rendements similaires à ceux des terrains dans la plaine de Moirans\footnote{Ces terrains ayant été acquis récemment, il n'y a pas assez de recul pour réaliser une véritable comparaison de rendements. Néanmoins, il n'y a pas de différence \textit{a priori} notable.}. Il faut aussi mettre en regard la différence de prix à l'achat : environ 5000\euro{} par hectare à Crossey, plus de 20000 à Moirans. 

\section{Moyens techniques en œuvre}

\subsection{Type de culture}

Trois céréales sont cultivées pour la fabrication du pain :

\begin{itemize}
\item[$\clubsuit$] le blé tendre d'hiver (\textit{Triticum aestivum}) : blé semé en novembre et récolté à l'été suivant. Des variétés modernes et anciennes, avec récupération des semences sont utilisées ;
\item[$\clubsuit$] le seigle (\textit{Secale cereale}), céréale très rustique, mélangée avec la farine de blé pour faire le pain de campagne ;
\item[$\clubsuit$] le petit épeautre (\textit{Triticum monococcum}) : céréale à faible teneur en gluten, utilisée pour les pains uniquement au petit épeautre, apprécié par les personnes sensibles à l'excès de gluten dans leur alimentation.
\end{itemize}

\subsection{Conduite des céréales}

La conduite des céréales se distingue dans de nombreux points de celle du maraichage. En particulier, les interventions sont beaucoup moins nombreuses et plus simples que dans le cas du maraîchage, mais la quantité de travail s'étend sur des surfaces bien plus grandes. La mécanisation est alors indispensable.

\subsubsection*{Mécanisation}

Comme la culture céréalière s'étend sur de grandes surfaces (8 ha cultivés sur les 24), un tracteur puissant est indispensable pour réaliser des différents travaux : celui utilisé fait 90ch. Il est équipé de :
\begin{itemize}
	\item[$\spadesuit$] un semoir de 3m de large avec 17 verseurs ;
	\item[$\spadesuit$] une herse étrille, destinée à désherber au printemps ;
	\item[$\spadesuit$] une herse rotative (sans prise de force) composée d'un gros rouleau à temps, de griffes à ressort et enfin d'un rouleau en "cage" ;
	\item[$\spadesuit$] une charrue à 6 socs, destinée au labour.
\end{itemize}
Chaque outil a une utilité spécifique dans la conduite des céréales. 

\subsubsection*{Préparation du sol}

Dans la rotation (qui sera décrite ultérieurement) choisie par les exploitants, les céréales suivent une culture de soja, qui aura été préalablement récolté, puis labouré à une profondeur moyenne (une vingtaine de cm), au mois de septembre ou octobre. Avant le semis, le terrain est ensuite passé à la herse rotative (une ou deux fois, en fonction des conditions météorologiques) afin de préparer un lit de semence homogène et lisse.

\subsubsection*{Semis et préparation des semences}

En agriculture biologique, la grande difficulté avec les céréales est la gestion de l'enherbement : il est donc impératif d'avoir une semence pure, sans graines d'adventice. Toutes les céréales utilisées ici sont des semences paysannes, où une partie des grains d'une année est semée pour l'année suivante\footnote{Dans le cas du blé, 1/6 de la quantité semée provient d'une variété moderne, achetée à un semencier.}. Il faut donc les trier soigneusement, à l'aide d'un trieur alvéolaire, qui trie les grains par leur taille et leur forme. Les graines d'adventices typiquement problématiques sont le fol avoine, la veisse ou le pois. 

Le semis est réalisé idéalement avant une averse de pluie, pour éviter de travailler le sol sur un sol humide, et optimiser au mieux la levée des graines. La période idéale est octobre pour le petit épeautre, fin octobre mi-novembre pour le blé et le seigle. Ces céréales s'implantent au début de l'hiver dans le sol et sont capables de pousser dès que la température est supérieure à 5$^\circ$C. Elles ont alors un avantage concurrentiel par rapport aux à la plupart des adventices qui germent au printemps, en étant déjà installées avec un réseau racinaire et du feuillage.

\textit{NB} : Le petit épeautre est semé plus tôt, car son cycle de développement est plus long d'un mois que celui du blé ou du seigle.

\subsubsection*{Gestion de l'enherbement et entretien des cultures}

Ces céréales poussent vite et haut au printemps, ce qui leur donne l'avantage compétitif de pouvoir étouffer la végétation concurrente. Néanmoins, ce "nettoyage" naturel nécessite quelques précautions. En premier lieu, il faut passer la herse étrille (dite "la griffe") dans le champs de blé au printemps, lors des premières chaleurs, lorsque les conditions sont réunies pour la germination des adventices. Le passage va détruire les jeunes pousses d'adventices et déranger les racines du blé. Cette opération, appelée le tallage, est paradoxalement bénéfique pour le blé, qui va alors renforcer son réseau racinaire et développer de nouvelles tiges. 

Rond de chardon

Rumex, à la main

\subsubsection*{Récolte}


\subsection{Rotation}

\begin{center}
	\includegraphics[scale=0.5]{rotation.pdf}
\end{center}


\end{document}
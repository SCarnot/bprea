\documentclass{article}
 
\usepackage[utf8]{inputenc}
\usepackage[francais]{babel}
\usepackage[T1]{fontenc}      
\usepackage[top=3.5cm, bottom=3cm, left=3.0cm, right=4.0cm]{geometry}
\usepackage{graphicx}
\usepackage{eurosym}
\graphicspath{{figures/}{../figures}}
\usepackage[activate={true,nocompatibility},final,tracking=true,kerning=true,spacing=true,factor=1100,stretch=10,shrink=10]{microtype}
\title{Culture céréalière pour la transformation boulangère : exemple d'un paysan-boulanger dans le pays Voironnais}
\date{Octobre 2019}

\begin{document}

\maketitle

\section{Présentation de la culture}

\subsection{Point botanique}

On s'intéresse à la culture du blé tendre, ou \textit{Triticum aestivum}. C'est une graminée, cultivée depuis 6000 ans, à la base dans le croissant fertile (aujourd'hui de la Turquie jusqu'à l'Iraq). C'est une des plantes à la base de la civilisation indo-européenne. Son cycle est annuel : adaptée au climat chaud et sec en été et rigoureux en hiver, elle s'implante en début d'hiver, durant lequel elle pousse lentement mais prend l'avantage sur ces concurentes en étant déjà présente au printemps, puis termine son cycle (croissance végétative et épiaison) au milieu de l'été. Les pieds sèchent alors rapidement, les épis se recroqueville, pour déposer les grains naturellement au sol afin de regermer l'année suivante. 

Parler aussi : le cycle complet, les variétés d'hiver et de printemps, le tallage. 

\subsection{Pourquoi choisir cette culture}

Le blé est un symbole de l'agriculture : c'est la céréale reine pour la panification, aliment pilier de la culture Française. Avec le seigle, le blé est transformé en farine, avec laquelle on fabrique les denrées alimentaires de base. Ces céréales sont historiquement la source calorique élémentaire dans la société française. Ainsi, si dans l'imaginaire collectif, la transition agroécologique est représentée surtout sur la production maraichère, elle doit, ou devra, impérativement se focaliser sur la culture des céréales qui sont la base de notre alimentation à grande échelle. Enfin, si la culture du blé et céréalière en général n'est pas très rentable (entre 100 et 1000\euro{} par hectare en AB, à comparer aux rendements de 30000\euro{}/ha en maraichage), le blé peut être très bien valorisé à travers la panification, qui décuple la valeur ajoutée. 

En terme agronomique, l'intérêt du blé (et les grandes cultures en général), en comparaison au maraîchage, est que le temps dédié aux cultures est très faible, que les variétés sont très adaptées aux climats locaux et ne nécessitent pas systématiquement d'irrigation. En revanche, la mécanisation est indispensable sur les surfaces travaillées. 

\section{Présentation de la structure et place de la culture}

\subsection{Présentation de la ferme}

Nous nous intéressons ici à la culture céréalière réalisée dans la ferme "La poule aux fruits d'or", un GAEC de trois associés installé à Saint-Étienne de Crossey. Les céréales cultivées sont principalement le blé, le seigle et le petit épeautre, et dans une moindre mesure, le soja et du maïs. Toute la production est en agriculture biologique, et avec pour ligne directrice de réduire au minimum les intrants extérieurs. Les céréales sont directement transformées dans leur activité boulangère ou utilisées en aliment pour leurs poules pondeuses.

\subsection{Gamme}

La ferme propose à la vente au marché, en fonction de la saison :
\begin{itemize}
	\item[-] du pain de campagne, pain complet, pain aux graines, pain de petit épeautre, pain de mie et brioche ;
	\item[-] des fruits frais : pomme, poire, kiwi, fraise, framboise, noix ;
	\item[-] de la confiture, des sorbets, des jus, des coulis et des sirops ;
	\item[-] des oeufs de ponte.
\end{itemize}

La ferme produit par ailleurs du soja pour l'alimentation humaine, mais aussi du triticale, du pois, de la féverolle et du maïs pour l'alimentation des poules pondeuses.

Les produits de boulangerie sont fabriqués exclusivement à partir des céréales cultivées à la ferme\footnote{Sauf lors d'années avec une catastrophe climatique, ce qui est exceptionnel)} : blé tendre, seigle et petit épeautre. Toutes sont dévolues à la boulangerie à travers la gamme de produits suivants :
\begin{itemize}
	\item[-] pain de campagne : 78\% de farine de blé T80, 12\% de farine de seigle ;
	\item[-] pain complet : 97\% de farine de blé T130, 3\% de son ;
	\item[-] pain petit épeautre : 100\% petit épeautre. Cette céréale ne contient que très peu de gluten, elle est difficile à panifier, mais permet d'attirer une clientèle sensible à l'excès de gluten dans son alimentation.
\end{itemize}
Les pains sont préparés deux fois par semaine (le jeudi et le samedi), soit "moulés" soit en baguette, soit en "olive". 

\subsection{Caractéristiques des cultures}

Les cultures de blé se répartissent à Saint-Etienne de Crossey, à Saint-Jean de Moirans et à Moirans, en rotation avec le seigle, la prairie semée et les légumineuses. La rotation sera présentée dans la partie \ref{part:rotation}. Les céréales sont cultivées les surfaces suivantes :
\begin{itemize}
	\item[-] 5ha de blé
	\item[-] 0,5ha de seigle
	\item[-] 2,5ha de petit épeautre	
\end{itemize}
Le blé et le seigle ont des rendements similaires (aux alentours de 30 quintaux à l'hectare), les surfaces sont cultivées en proportion de l'utilisation en boulangerie : 1/10 de seigle par rapport au blé. 

\subsection{Caractéristiques pédologiques}

Le parcellaire est géographiquement assez éclaté, et les terrains sont assez différents. Tous les terrains agricoles dédiés aux céréales sont entourés de cultures en pratique conventionnelle. Olivier laisse une bande tampon permettant aussi le passage d'engins agricoles. 

\subsubsection{Plaine de Moirans} 

C'est une terre principalement limoneuse dans le lit alluvionnaire de l'Isère, avec des terrains plats. 

\textbf{Avantages :}
\begin{itemize}
	\item[-] terre facile à travailler sur terrain plat
	\item[-] terre très profonde, avec une grande réserve en nutriment et en eau
	\item[-] nappe phréatique située à quelque mètre sous le sol assurant une humidité permanente quelque soit la météo. Evite toute catastrophe liée à la sécheresse.
\end{itemize}

\textbf{Inconvénients : }
\begin{itemize}
	\item[-] une croûte de battance peut se former après une pluie et peut empêcher une bonne levée du semis. Il faut passer la herse étrille pour aérer le sol.
	\item[-] ces terrains ne souffrent jamais d'un excès de sécheresse mais l'humidité, idéale pour le maïs ou le soja, peut limiter les rendements du blé, d'après les retours d'un agriculteur en conventionnel. 
\end{itemize}

La plaine de Moirans est réputée pour avoir des terres plates et fertiles, permettant à l'agriculture conventionnelle d'obtenir des rendements exceptionnels en maïs. La culture du blé en AB est donc relativement aisée à conduire. 

\subsubsection{Saint Étienne de Crossey} 

Situés à 10 km au nord des plaines de Moirans, les terrains à Saint Étienne de Crossey sont quant à eux principalement argileux et en pente douce, au pied d'une colline. Le sol est en permanence humide, en cas de pluie, le sol est rapidement saturé en eau, au point de faire de légères résurgences.

Ainsi, à l'inverse des plaines de Moirans, ces parcelles sont plus compliquées à travailler : 
\begin{itemize}
	\item[- ] la terre étant plus lourde, elle est plus difficile à travailler et s'accumule rapidement dans une herse rotative ;
	\item[- ] les créneaux pour travailler le sol sont étroits. En effet, le temps de réssuyage est long et les jours de pluie sont nombreux, laissant peut de marge de manœuvre sur les période de travail du sol. 
	\item[- ] l'humidité est parfois telle que le risque d'embourber le matériel est réel. Aussi, de grandes flaques peuvent apparaître est empêcher tout semis, ou travail du sol.
\end{itemize}

Néanmoins, les rendements semblent similaires\footnote{Ces terrains ont été acquis récemment, il n'y a pas assez de recul avec les rotations pour avoir une idée précise du rendement.} à ceux de la plaine de Moirans. Olivier souligne qu'ils sont plus difficiles à travailler, avec des créneaux de passage au champ plus restreints et un risque plsu élevé en cas de sécheresse de printemps.

\subsection{Rendement attendu}

Les rendements pour le blé attendus (et obtenus en cas d'année normale, c'est-à-dire sans accident) sont de 30 à 35 quintaux par hectare. Ils se situent dans la fourchette haute de la culture du blé en agriculture biologique, qui sont plutôt autour de 25-30 qx/ha. En comparaison, les rendements atteints en "terres à céréales" du bassin Parisien atteignent 40qx/ha en AB. En agriculture conventionnelle, les terres environnantes à un rendement de 60-70 qx dans les terres environnantes, et jusqu'à 100qx dans le bassin parisien et le nord de la France\footnote{Source : \textit{Rotations pratiquées en grandes cultures biologiques en France : état des lieux par région, ITAB, 2011}, et témoignage de différents agriculteurs de la plaine du Grésivaudan}. 

A noter que les rendements obtenus par le GAEC sont plutôt bons, malgré l'utilisation de variétés anciennes, ce qui semble être principalement du à la rotation, qui permet un apport suffisant de matière azotée et à la qualité des terres de Moirans.

Avec ces rendements, la production totale de blé est de 5$\times$30qx, soit 15 tonnes de grain. La production suffit donc à la production de pain et de réensemencement. Les 15 tonnes de grains ont le potentiel de donner 10,5 tonnes de farine (en enlevant le son), et suffisent à la boulangerie qui consomment environ 10 tonnes de farine à l'année.

\section{Calendrier de culture et rotation}
\label{part:rotation}

\subsection*{Semis}

\textbf{Le blé tendre est semé à la fin de l'automne, à la mi-novembre.} Le blé est une graminée capable de pousser dès qu'il fait 5$^\circ$C : il s'implante donc doucement durant l'hiver, où il prend l'avantage sur ces concurrentes. Au printemps, déjà implanté, il pousse fort et étouffe les concurrentes potentielles. La principale difficulté est de trouver le créneau météorologique pour semer à l'automne, où les terrains sont souvent gorgés d'eau. 

Hiver mais possible semis d eprintemps. Risque du printemps : sécheresse des plantules, terres qui se sèche ou qui gèle et qui casse les racines en se contractant. Moins de concurrence face aux adventices.

\subsection{Rotation}

\subsubsection{Importance de la rotation dans un système paysan-boulanger AB}

Dans un système paysan-boulanger, une importance cruciale est donnée à la place du blé dans la rotation, car c'est la culture qui est de loin la plus utilisée \textit{in fine} en boulangerie (environ 200kg de farine de blé par semaine) et qui a une importance économique cruciale : la boulangerie représente un chiffre d'affaire d'environ 100 000\euro{}, soit presque la moitié du GAEC. La rotation est donc pensée de sorte à donner une place centrale au blé tout en variant les cultures sur une parcelle donnée pour avoir les effets positifs d'une rotation : apport azoté, structure du sol, maîtrise des adventices, résistances aux maladies. Il y a un juste compromis à trouver entre la qualité agronomique d'une rotation longue et très diversifiée et l'impératif de valorisation économique en laissant une place prépondérante au blé meunier.

 Les différents systèmes paysan-boulanger peuvent avoir des surfaces consacrées au blé variant de 33\% à 50\% (soit d'un tiers à la moitié des blocs)\footnote{Ces valeurs sont des témoignages de paysans-boulangers du Dauphiné.}. Il est à noter que le rendement du blé semble décroître lorsque sa présence augmente dans l'assolement. Dans la région dauphinoise, une rotation "classique" de paysan-boulanger est constituée de 2 ou 3 ans de luzerne ou de trèfle, puis 2 ou 3 ans de céréales (blé et seigle). 

Dans des systèmes de grandes cultures locales, c'est-à-dire dans la plaine alluvionnaire de l'Isère (dont nous reparlerons des caractéristiques pédoclimatique plus loin) où se situent les terrains du GAEC, les agriculteurs sont en très grande majorité en agriculture conventionnelle, avec une rotation maïs-soja, qui sont les cultures les plus productives localement, avec parfois du blé. 

\subsubsection{Rotation choisie}

La rotation des céréales à la ferme "La poule aux fruits d'or" se fait donc en 3 temps, sur 3 blocs de 8ha. Le premier bloc est dédié aux céréales (blé, seigle, petit épeautre), le deuxième est constitué d'une prairie enherbée jouant le rôle d’engrais vert (trèfle, ray-grass) et le dernier est dédié aux légumineuses (soja, féverole, triticale-pois)\footnote{On inclut dans la partie légumineuse l'association "triticale-pois". Le triticale est une céréale de la même famille du blé (graminée) permettant de tutorer le pois, qui lui est une légumineuse. La fixation en azote du pois permet le bon développement du triticale et un reliquat pour les céréales qui suivent. Le mélange est dédié à l'alimentation des poules de la ferme.}. La proportion de blé est donc ici ans plutôt basse (33\%), mais les rendements sont dans la fourchette haute en AB, à 35 quintaux l'hectare, et ce, malgré l'utilisation de variétés anciennes\footnote{Les rendements en AB de blé tendre dans la région Rhône-Alpes sont compris entre 25 et 35qx/ha. Pour comparaison, ils sont à 60 qx/ha en conventionnel dans la région, et à 40 qx/ha en moyenne en AB en Île-de-France. Source : \textit{Rotations pratiquées en grandes cultures biologiques en France : état des lieux par région, ITAB, 2011}}. 

\begin{center}
	\includegraphics[scale=0.9]{rotation0.pdf}
\end{center}

Le raisonnement suivi pour cette rotation est le suivant : 
\begin{itemize}

	\item[1 - ] \textbf{Céréales} : l'intérêt est tout d'abord économique, à travers la transformation en pain. Les céréales sont implantées en octobre (petit épeautre) et en novembre (blé et seigle). Une caractéristique des céréales est de pouvoir se développer durant l'hiver (principalement le réseau racinaire), de sorte à pouvoir prendre l'avantage sur leurs concurrentes au printemps en poussant rapidement et en étouffant les adventices concurrentielles. Elles sont récoltées au mois de juillet suivant. Les pailles sont laissées sur place pour apporter de la matière organique.
	
	\item[2 - ] \textbf{Prairie semée} : cette culture de ray-grass et trèfle (avec éventuellement de la phacélie) est dévolue à "reposer" le sol et le "régénérer" avec un apport de matière organique, azotée et de structuration du sol. Elle est implantée en septembre et laissée 18 mois pour profiter de la fixation d'azote apportée par le trèfle. Le ray-grass produit énormément de matière organique et surpasse le trèfle dans un premier temps. Il est broyé le premier avril de son implémentation, laissant la lumière au trèfle qui peut s'exprimer, donnant une prairie assez équilibrée durant l'été. Un deuxième broyage est effectué à la fin de l'été, pour éviter la montée en graine du ray-grass. Les systèmes racinaires du ray-grass et du trèfle sont complémentaires (fasciculé pour le premier, pivot pour le second), explorent le sol sur différentes strates, permettant d'obtenir une bonne structure. Le ray-grass est extrêmement concurrentiel et a un rôle "nettoyant". La prairie est labourée en février, 18 mois après son implémentation.
	
	\item[3 - ] \textbf{Légumineuses} : la fin de rotation est faite avec du soja, de la féverole ou un mélange triticale-pois. Ces légumineuses permettent aussi de fixer l'azote de l'air et permettent d'exporter les graines pour l'alimentation des poules ou revendues directement pour l'alimentation humaine pour le soja. Le mélange triticale-pois est intéressant car il permet de faire pousser une céréale (triticale, variété de blé, nettoyant) en symbiose avec le pois qui va grimper le long de sa tige.

\end{itemize}

\textbf{Bilan} : les associés de la ferme sont globalement satisfaits de la rotation ; elle permet d'obtenir des rendements en blé dans la fourchette haute de l'AB, de n'apporter aucun intrant extérieur (il n'y a pas d'amendement) et l'enherbement est maitrisé. Si la surface consacrée au blé est inférieure à d'autres rotations de paysan-boulanger, les autres cultures sont aussi valorisées, comme le soja pour l'alimentation humaine ou le triticale-pois pour l'alimentation des poules de la ferme. Une limite sur cette rotation serait sa simplicité et son faible nombre de cultures (3 blocs seulement). Une justification est que la faible surface cultivée (24ha au total) permet difficile de diviser en un plus grand nombre de cultures.

\section{Choix variétal}

\subsection{Variétés}

Trois variétés sont utilisées : une moderne, l'Energo, et deux anciennes, la Florence aurore et le Rouge de bordeaux. Le choix entre variétés modernes et anciennes se fait selon les critères de rendement, de résistance à la verse et aux maladies, à la teneur en protéine, à la qualité boulangère mais aussi éthiques. 

Olivier a choisi d'utiliser une variété moderne avec deux anciennes pour moyenner les qualités de l'une et de l'autre. Utiliser uniquement des variétés anciennes peut augmenter le risque d'un gros accident de verse. Si l'Ernergo est semé dans uen parcelle qui lui est propre, les deux variétés anciennes sont semées dans le même champ et sont resemées ensemble. 

\subsubsection{Energo}

C'est une variété dite "moderne" ou commerciale, utilisée à hauteur de 2ha sur les 5ha. C'est un blé à paille longue (moyenne de 116 cm), qui présente un compromis intéressant entre rendement et teneur en protéines. Il présente aussi de bonnes qualités boulangères.

\subsubsection{Florence Aurore et Rouge de Bordeaux}

Ce sont des variétés dites "ancienne", cultivées sur 3ha. Elles font une paille très haute et leur diversité génétique en fait des variétés rustiques et résistantes aux maladies. Ce sont d'excellente variétés pour la panification.

\subsection{Tri des semences}

Réutilisation d'une partie des semences. Tri des semences avec un trieur à plaque (triage grossier) puis trieur alvéolaire (triage fin) et triage manuel (triage très fin !). 

\section{Travail du sol et préparation}

On s'intéresse au travail du sol pour la culture du blé dans la rotation. Il faut garder à l'esprit que le travail du sol s'organise à l'échelle de la rotation ; ainsi les étapes présentées sont pertinentes dans le cadre de cette rotation. Par exemple, la maîtrise des adventices se fait grâce à la qualité nettoyante de la prairie semée et du labour qui s'en suit, mais qui n'est pas directement inclus dans le travail du sol d'une parcelle à blé.

Les étapes sont les suivantes :
\begin{itemize}
	\item[-] \textit{Sept. année N} : déchaumage du soja
	\item[-] \textit{J-15 avant semis} : faux semis
	\item[-] \textit{Nov. année N} : semis
	\item[-] \textit{mars-avril année N+1} : herse étrille
	\item[-] \textit{juil. année N+1} : moisson
\end{itemize}

\subsection{Déchaumage}

Le déchaumage est effectué avec un déchaumeur à disque et à dents (avec pattes d'oie). Il permet d'extraire touts les chaumes du soja précédent pour les extraire du sol et faciliter leur composition. Les dents permettent aussi de décompacter le sol sans le déstructurer complètement ; contrairement au labour, il n'y a pas de retournement des horizons du sol. Le déchaumage est aussi moins gourmand est gazole (10-15L/ha) que le labour (20L/ha) et est plus respectueux de la vie du sol, car moins travaillé. Le déchaumage est fait en septembre, lorsque les températures sont encore chaudes, ce qui permet une décomposition rapide des restes de soja.

\subsection{Faux-semis}

Le faux semis n'est pas systématiquement réalisé, car il faut une "fenêtre de tir" météo précise. Une quinzaine de jour avant la date prévisionnelle du semis (fait en novembre), Olivier passe la bineuse rapidement sur le champ précédemment déchaumé, la veille ou l'avant veille d'une pluie. Ce passage permet d'obtenir un lit de semence assez grossier, mais qui va permettre la levée des graines d'adventices présentes. Les germes d'adventices seront alors détruits lors du semis, 15 jours plus tard.

\subsection{Préparation du lit de semence et semis}

L'objectif est de préparer un lit de semence assez fin pour le blé et de le semer dans la foulée. On utilise un combiné de semis, sur lequel Olivier fixe directement le semoir (figure \ref{fig:semis}). 

\begin{figure}[h!]
\centering
\begin{minipage}{.5\textwidth}
  \centering
  \includegraphics[width=0.9\textwidth]{combine_semis.jpeg}
  \label{fig:test1}
\end{minipage}%
\begin{minipage}{.5\textwidth}
  \centering
  \includegraphics[width=0.9\textwidth]{semoir.jpeg}
  \label{fig:test2}
\end{minipage}
\caption{\textit{(Gauche)} Combiné de semis Pichon. \textit{(Droite)} Semoir à descente Nodet}
\label{fig:semis}
\end{figure}

Le combiné de semis est composé d'un rouleau cage, de dents de vibroculteur et d'un rouleau Packer. Son passage permet d'obtenir un lit de semence assez fin et homogène.

Le semoir Nodet est fixé directement sur le combiné de semis et permet de semer directement durant le passage du combiné de semis, permettant de gagner du temps par rapport à la situation où il faudrait passer le combiné puis le semoir. 

Cependant, les deux outils attachés ensembles sont lourds, surtout si le semoir est rempli de semences (jusqu'à 200kg). Si le terrain est parfaitement ressuyé et plats, avec les chaumes précédentes bien décomposée, c'est en effet un gain de temps avec une efficacité certaine. Mais si le sol est mal ressuyé avec de la pente, le passage des deux outils est compliqué car le tracteur s'enfonce profondément dans la terre, peut maniable, et le lit de semence n'est pas parfaitement réalisé. 

C'est ce qu'il c'est passé lors du semis de ce mois de novembre, particulièrement humide, où la fenêtre de tir pour semer était trop étroite vis-à-vis de la météo. Pour éviter de trop charger le tracteur, la semence n'a pas été mise dans le semoir et il a été décidé de passer d'abord le combiné, puis ensuite semer. Malgré cette précaution, le tracteur a difficilement travaillé le sol gorgé d'eau, en s'enfonçant sur un terrain où apparaissait même des résurgences par endroit. En conséquence, le semis dans ces zones, sur lesquelles on voit des traces d'érosion, a été un échec. 

\begin{figure}
\begin{center}
	\includegraphics[scale=0.2]{IMG_3352.jpeg}
	\caption{Parcelle de blé, fin novembre. Le blé a levé, mais on voit de nettes traces d'érosion et de tassement suite à un semis dans des conditions difficiles : terrain mal ressuyé et en pente.}
\end{center}
\end{figure}

Il aurait été peut-être plus prudent, quitte à perdre un peu de temps, à monter uniquement le combiné, faire un aller-retour à la ferme, installer le semoir sans le combiné, afin d'alléger le tracteur, et d'éviter les zones trop humides. C'est un compromis délicat entre le temps passé à semer et l'efficacité du semis.

\subsection{Entretien de la culture}

L'entretien de la culture est relativement simple : il y a 2 ou 3 passages de herse étrille dans le champs de blé, à grande vitesse, au printemps. Ce passage à 2 objectifs : détruire les potentielles adventices de printemps qui sont à un stade bien plus précoce que celui du blé qui a été implanté depuis 5 mois ; et aérer le sol pour activer la vie micro-biologique du sol, qui permet de minéraliser l'azote. 

Le blé a une caractéristique intéressante : loin de souffrir du passage de la herse étrille, le fait d'être "griffé" par les dents arrache quelques feuilles, quelques racines qui ne compromet la survie du plant : au contraire, cela stimule le tallage, c'est-à-dire sa capacité à former de nouveaux épis.

Une fois passé le passage 3-4 feuilles, le blé pousse alors vite et devient assez étouffant : il ne laisse plus de lumière aux adventices concurrentes.

Il est à noter que si l'entretien de la culture du blé parait relativement simple par rapport à l'entretien de la plupart des cultures maraîchères, c'est parce qu'en amont, la rotation élaborée permet de prévenir l'apparition d'un enherbement ingérable. 

\subsection{Récolte}

La récolte est effectuée début juillet, par un agriculteur voisin qui possède une moissonneuse batteuse. 

\subsection{Cas du labour}

Dans la conduite du blé présentée, il n'est pas mentionnée la pratique d'un labour. Olivier précise que cette conduite est réalisée "quand tout va bien", c'est-à-dire sans un enherbement qui s'est implanté sur la culture de soja précédente et qui risque d'être problématique pour le semis du blé. En fonction du risque, Olivier peut réaliser un labour à la place du déchaumage pour éviter de se faire déborder. S'il cherche à éviter un labour, il ne se prive pas de cette option lorsque cela devient risqué, malgré les impacts sur l'érosion\footnote{Le risque d'érosion est uniquement présent sur les parcelles vallonnées de Saint-Etienne de Crossey, il est quasi-inexistant sur les terres plates de Moirans).}.

Le labour reste néanmoins présent dans la rotation, pour détruire la prairie semée, dont le ray-grass peut être particulièrement virulent. Ce labour permet de détruire convenablement la prairie et de prévenir le retour d'adventices. Néanmoins, on peut se poser la question de savoir si un passage de broyeur et un déchaumage suffiraient.

\section{Amendement, fertilisation et bilan humique}

Discussion potentiel agronomique Engrais vert : prairie semée. Estimation de l'azote apporté ? Satisfaction de la rotation ? Apport légumineuse ?

Pas de fertilisation, ni apport de fumier.

\subsection{Analyse de la fertilisation}

A demander, voir avec les rendements. 

\section{Gestion des adventices et des maladies}

\subsection{Principales adventices}

\textbf{A compléter}

Rumex et chardon très problématiques. Gestion : herse étrille passée au printemps, désherbage manuel ponctuel (exceptionnel, compliqué à cause de la taille des parcelles).

\subsection{Méthodes de lutte}

\textbf{A compléter}

Labour et herse étrille

Parler de l'intérêt des rotation en grandes culture. Nettoyage de la prairie semée, l'évaluer. Notamment du seigle et son effet allopathique.

\section{Protection phytosanitaire}

Les maladies et les ravageurs sont moins "prégnants" en grande culture qu'en maraichage. Le blé est globalement une plante rustique qui, si les rotations sont bien conduites, résiste assez bien aux maladies et aux ravageurs.

\subsection{Ravageurs et maladie}

\textbf{A compléter}

Principale maladies ou ravageurs : carie du blé, fusariose, ergot, verse (pas une maladie mais peut être déclenchée par un champignon), mycotoxine.

\subsection{Traitement}

\textbf{A compléter}

Rotation, choix de la densité. 

\section{Incidence du climat}

\subsection{Climat sur les différentes parcelles}

Détailler à Crossey et à Moirans. 

\subsection{Conditions pédo-climatique}

Détailler Crossey et Moirans. Incidence des terres sur la rétention en eau, la possibilité de faire du soja à Crossey.

\subsection{Besoin en eau}

Les besoins en eau du blé : fort à la croissance, à l'épiaison puis besoin de sec pour la maturation

\subsection{Climat et ravageurs}

Lien entre le climat et l'apparition de maladies

\subsection{Risques climatique}

Sécheresse au printemps (rendement), humidité en été (germination sur tige), orage, grêle

\section{Récolte, tri et stockage}

\subsection{Récolte}

Batteuse, période de séchage ? Combien de temps en silo avant transformation en farine ? Combien de temps peut se stocker le grain ?

\subsection{Tri}

Tri grossier pour le grain à moudre, fin pour les semences. Cf partie tri.

\section{Stockage}

En silo. Précautions, ventilation ? Forme des silo ? Quand aère t-on, comment ? 

\subsection{Transformation et vente}

\end{document}
\documentclass{article}
 
\usepackage[utf8]{inputenc}
\usepackage[francais]{babel}
\usepackage[T1]{fontenc}      
\usepackage[top=3.5cm, bottom=3cm, left=3.0cm, right=4.0cm]{geometry}
\usepackage{graphicx}
\usepackage{eurosym}
\graphicspath{{figures/}{../figures}}
\usepackage[activate={true,nocompatibility},final,tracking=true,kerning=true,spacing=true,factor=1100,stretch=10,shrink=10]{microtype}
\title{Culture céréalière pour la transformation boulangère : exemple d'un paysan-boulanger dans le pays Voironnais}
\date{Octobre 2019}

\begin{document}

\maketitle

\section{Présentation de la culture}

\subsection{Point botanique}

On s'intéresse à la culture du blé tendre, ou \textit{Triticum aestivum}. C'est une graminée, cultivée depuis 6000 ans, à la base dans le croissant fertile (aujourd'hui de la Turquie jusqu'à l'Iraq). C'est une des plantes à la base de la civilisation indo-européenne. Son cycle est annuel : adaptée au climat chaud et sec en été et rigoureux en hiver, elle s'implante en début d'hiver, durant lequel elle pousse lentement mais prend l'avantage sur ces concurentes en étant déjà présente au printemps, puis termine son cycle (croissance végétative et épiaison) au milieu de l'été. Les pieds sèchent alors rapidement, les épis se recroqueville, pour déposer les grains naturellement au sol afin de regermer l'année suivante. 

Parler aussi : le cycle complet, les variétés d'hiver et de printemps, le tallage. 

\subsection{Pourquoi choisir cette culture}

Le blé est un symbole de l'agriculture : c'est la céréale, avec le seigle, qui transformée en farine, permet d'obtenir les denrées alimentaires de base. C'est historiquement la source calorique élémentaire dans la société française. Ainsi, si dans l'imaginaire collectif, la transition agroécologique est représentée surtout sur la production maraichère, elle doit, ou devra, impérativement se focaliser sur la culture des céréales qui sont la base de notre alimentation à grande échelle. Enfin, si la culture du blé et céréalière en général n'est pas très rentable (entre 100 et 1000\euro{} par hectare en AB, à comparer aux rendements de 30000\euro{}/ha en maraichage), le blé peut être très bien valorisé à travers la panification, qui décuple la valeur ajoutée. 

En terme agronomique, l'intérêt du blé (et les grandes cultures en général), en comparaison au maraîchage, est que le temps dédié aux cultures est très faible, que les variété sont très adapté au climats locaux et ne nécessitent pas systématiquement d'irrigation. En revanche, la mécanisation est indispensable sur les surfaces travaillées. 

\section{Présentation de la structure et place de la culture}

\subsection{Présentation de la ferme}

Nous nous intéressons ici à la culture céréalière réalisée dans la ferme "La poule aux fruits d'or", un GAEC de trois associés installé à Saint-Étienne de Crossey. Les céréales cultivées sont principalement le blé, le seigle et le petit épeautre, et dans une moindre mesure, le soja et du maïs. Toute la production est en agriculture biologique, et avec pour ligne directrice de réduire au minimum les intrants extérieurs. Les céréales sont directement transformées dans leur activité boulangère ou utilisées en aliment pour leurs poules pondeuses.

\subsection{Gamme}

Céréales cultivées : blé tendre, seigle et petit épeautre. Toutes sont dévolues à la boulangerie à travers la gamme de produtis suivants :
\begin{itemize}
	\item[-] pain de campagne : x\% de farine de blé, 100-x\% de seigle
	\item[-] pain complet : x\% de farine de blé, 100-x\% de seigle + son
	\item[-] pain petit épeautre : pas de gluten, difficile à panifier
\end{itemize}

\subsection{Dimensions des cultures}

5ha de blé, 1 ha de seigle, 2,5 petit épeautre.

Présenter les conditions pédoclimatiques des parcelles

\textbf{Questions : disproportion du seigle par rapport à son utilisation en boulangerie ??}

\section{Rendement attendu}

\textbf{A demander, mais déjà posé, réponse floue...}, demander la production totale. Variable selon les terrains (plaine de Moirans ou Saint Étienne de Crossey)

\section{Calendrier de culture et rotation}

En trois temps : céréales (blé d'hiver, seigle, petit épeautre), prairie semée (ray-grass et trèfle), puis légumineuse (soja, triticale-pois, ).

\begin{center}
	\includegraphics[scale=0.5]{rotation.pdf}
\end{center}

\section{Choix variétal}

\subsection{Variétés}

Variété anciennes\textbf{ (lesquelles ??)} et modernes \textbf{(lesquelles ??)}

\subsection{Tri des semences}

Réutilisation d'une partie des semences. Tri des semences avec un trieur à plaque (triage grossier) puis trieur alvéolaire (triage fin) et triage manuel (triage très fin !). 

\section{Travail du sol et préparation}

\subsection{Labour}

Labour (quelle profondeur ?) quelques jours ou semaines avant semis, cela permet la décomposition de la culture ou de l'EV détruit. 
Le labour détruit totalement les adventices.

\subsection{Préparation du lit de semence}

On utilise un combiné de semis. Affine grandement la terre labourée avant semis.

\begin{figure}[h!]
\centering
\begin{minipage}{.5\textwidth}
  \centering
  \includegraphics[width=0.9\textwidth]{combine_semis.jpeg}
  \label{fig:test1}
\end{minipage}%
\begin{minipage}{.5\textwidth}
  \centering
  \includegraphics[width=0.9\textwidth]{semoir.jpeg}
  \label{fig:test2}
\end{minipage}
\caption{\textit{(Gauche)} Combiné de semis Pichon. \textit{(Droite)} Semoir à descente Nodet}
\label{fig:test}
\end{figure}

\subsection{Semis}

\subsubsection{Calibration du semoir}

Le blé, le seigle et le petit épeautre nécessitent des densités à l'hectare différente. Il faut calibrer le semoir : décrire protocole.

\subsubsection{Semis}

Fenêtre météo. Difficultés rencontrées (bourrage, casse, sol pas ressuyé)

\subsection*{Entretien de la culture}

Passage de la herse étrille. plus de détail dans la dernière partie

\subsection{Récolte}

Période ? Sous traitée à un agriculteur du coin. 
Risques ? Trop tôt : grain trop humide, trop tard : verse, grain qui tombe. 

\section{Amendement, fertilisation et bilan humique}

Discussion potentiel agronomique Engrais vert : prairie semée. Estimation de l'azote apporté ? Satisfaction de la rotation ? Apport légumineuse ?

Pas de fertilisation, ni apport de fumier.

\subsection{Analyse de la fertilisation}

A demander, voir avec les rendements. 

\section{Gestion des adventices et entretien de la culture}

\subsection{Principale advendices}

\textbf{A compléter}

Rumex et chardon très problématiques. Gestion : herse étrille passée au printemps, désherbage manuel ponctuel (exceptionnel, compliqué à cause de la taille des parcelles).

\subsection{Méthodes de lutte}

\textbf{A compléter}

Labour et herse étrille

Parler de l'intérêt des rotation en grandes culture. Nettoyage de la prairie semée, l'évaluer. Notamment du seigle et son effet allopathique.

\section{Protection phytosanitaire}

Les maladies et les ravageurs sont moins "prégnants" en grande culture qu'en maraichage. Le blé est globalement une plante rustique qui, si les rotations sont bien conduites, résiste assez bien aux maladies et aux ravageurs.

\subsection{Ravageurs et maladie}

\textbf{A compléter}

Principale maladies ou ravageurs : carie du blé, fusariose, ergot, verse (pas une maladie mais peut être déclenchée par un champignon), mycotoxine.

\subsection{Traitement}

\textbf{A compléter}

Rotation, choix de la densité. 

\section{Incidence du climat}

\subsection{Climat sur les différentes parcelles}

Détailler à Crossey et à Moirans. 

\subsection{Conditions pédo-climatique}

Détailler Crossey et Moirans. Incidence des terres sur la rétention en eau, la possibilité de faire du soja à Crossey.

\subsection{Besoin en eau}

Les besoins en eau du blé : fort à la croissance, à l'épiaison puis besoin de sec pour la maturation

\subsection{Climat et ravageurs}

Lien entre le climat et l'apparition de maladies

\subsection{Risques climatique}

Sécheresse au printemps (rendement), humidité en été (germination sur tige), orage, grêle

\section{Récolte, tri et stockage}

\subsection{Récolte}

Batteuse, période de séchage ? Combien de temps en silo avant transformation en farine ? Combien de temps peut se stocker le grain ?

\subsection{Tri}

Tri grossier pour le grain à moudre, fin pour les semences. Cf partie tri.

\section{Stockage}

En silo. Précautions, ventilation ? Forme des silo ? Quand aère t-on, comment ? 

\subsection{Transformation et vente}

\end{document}
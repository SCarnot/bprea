\documentclass{article}
 
\usepackage[utf8]{inputenc}
\usepackage[francais]{babel}
\usepackage[T1]{fontenc}      
\usepackage[top=3.5cm, bottom=3cm, left=3.0cm, right=4.0cm]{geometry}
\usepackage{graphicx}
\usepackage{eurosym}
\usepackage{amsmath}
\graphicspath{{figures/}{../figures}}
\usepackage[activate={true,nocompatibility},final,tracking=true,kerning=true,spacing=true,factor=1100,stretch=10,shrink=10]{microtype}
\title{Culture céréalière pour la transformation boulangère : exemple d'un paysan-boulanger dans le pays Voironnais}
\date{Octobre 2019}

\begin{document}

\maketitle

\section{Présentation de la culture}

\subsection{Point botanique}

On s'intéresse à la culture du blé tendre, ou \textit{Triticum aestivum}. Cette graminée est cultivée depuis 8000 ans dans le croissant fertile (correspondant à une zone couvrant de la Turquie jusqu'à l'Irak actuels). C'est une des plantes à la base de la civilisation indo-européenne. Son cycle est annuel : adaptée au climat chaud et sec en été et rigoureux en hiver, elle s'implante en début d'hiver, durant lequel elle pousse lentement mais prend l'avantage sur ces concurrentes en étant déjà présente au printemps, puis termine son cycle (croissance végétative et épiaison) au milieu de l'été. Les pieds sèchent alors rapidement, les épis se recroquevillent mais ne tombent au sol, contrairement à \textit{Triticum dicoccoides}, l'ancêtre sauvage du blé. 

\subsection{Pourquoi choisir cette culture}

Le blé est un symbole de l'agriculture : c'est la céréale reine pour la panification, aliment pilier de la culture Française. Le blé est transformé en farine, avec laquelle on fabrique les denrées alimentaires de base. Cette céréales est historiquement une des sources caloriques élémentaires dans la société française. Ainsi, si dans l'imaginaire collectif, la transition agroécologique est représentée surtout sur la production maraichère, elle doit, ou devra, impérativement se focaliser sur la culture des céréales qui sont la base de notre alimentation à grande échelle. Enfin, si la culture du blé et céréalière en général n'est pas très rentable (entre 100 et 1000\euro{} par hectare en AB, à comparer aux rendements de 30000\euro{}/ha en maraichage), le blé peut être très bien valorisé à travers la panification, qui décuple la valeur ajoutée. 

En terme agronomique, l'intérêt du blé (et les grandes cultures en général), en comparaison au maraîchage, est que le temps dédié aux cultures est très faible (de l'ordre d'une dizaine d'heures par an et par hectare), que les variétés sont très adaptées aux climats locaux et ne nécessitent pas systématiquement d'irrigation. En revanche, la mécanisation est indispensable au regard des surfaces travaillées. 

\section{Présentation de la structure et place de la culture}

\subsection{Présentation de la ferme}

Nous nous intéressons ici à la culture céréalière réalisée dans la ferme "La poule aux fruits d'or", un GAEC de trois associés installé à Saint-Étienne de Crossey. Les céréales cultivées sont principalement le blé, le seigle et le petit épeautre, et dans une moindre mesure, le soja et du maïs. Toute la production est en agriculture biologique, et avec pour ligne directrice de réduire au minimum les intrants extérieurs. Les céréales sont directement transformées dans leur activité boulangère ou utilisées en aliment pour leurs poules pondeuses.

\subsection{Gamme}

La ferme propose à la vente au marché, en fonction de la saison :
\begin{itemize}
	\item[-] du pain de campagne, pain complet, pain aux graines, pain de petit épeautre, pain de mie et brioche ;
	\item[-] des fruits frais : pomme, poire, kiwi, fraise, framboise, noix ;
	\item[-] de la confiture, des sorbets, des jus, des coulis et des sirops ;
	\item[-] des oeufs de ponte.
\end{itemize}

La ferme produit par ailleurs du soja pour l'alimentation humaine, mais aussi du triticale, du pois, de la féverolle et du maïs pour l'alimentation des poules pondeuses.

Les produits de boulangerie sont fabriqués exclusivement à partir des céréales cultivées à la ferme\footnote{Sauf lors d'années avec une catastrophe climatique, ce qui est exceptionnel)} : blé tendre, seigle et petit épeautre. Toutes sont dévolues à la boulangerie à travers la gamme de produits suivants :
\begin{itemize}
	\item[-] pain de campagne : 78\% de farine de blé T80, 12\% de farine de seigle ;
	\item[-] pain complet : 97\% de farine de blé T130, 3\% de son ;
	\item[-] pain petit épeautre : 100\% petit épeautre. Cette céréale ne contient que très peu de gluten, elle est difficile à panifier, mais permet d'attirer une clientèle sensible à l'excès de gluten dans son alimentation.
\end{itemize}
Les pains sont préparés deux fois par semaine (le jeudi et le samedi), soit "moulés" soit en baguette, soit en "olive". 

\subsection{Caractéristiques des cultures}

Les cultures de blé se répartissent à Saint-Etienne de Crossey, à Saint-Jean de Moirans et à Moirans, en rotation avec le seigle, la prairie semée et les légumineuses. La rotation sera présentée dans la partie \ref{part:rotation}. Les céréales sont cultivées les surfaces suivantes :
\begin{itemize}
	\item[-] 5ha de blé
	\item[-] 0,5ha de seigle
	\item[-] 2,5ha de petit épeautre	
\end{itemize}
Le blé et le seigle ont des rendements similaires (aux alentours de 30 quintaux à l'hectare), les surfaces sont cultivées en proportion de l'utilisation en boulangerie : 1/10 de seigle par rapport au blé. 

\subsection{Caractéristiques pédologiques}

Le parcellaire est géographiquement assez éclaté, et les terrains sont assez différents. Tous les terrains agricoles dédiés aux céréales sont entourés de cultures en pratique conventionnelle. Olivier laisse une bande tampon permettant aussi le passage d'engins agricoles. 

\subsubsection{Plaine de Moirans} 

C'est une terre principalement limoneuse dans le lit alluvionnaire de l'Isère, avec des terrains plats. 

\textbf{Avantages :}
\begin{itemize}
	\item[-] terre facile à travailler sur terrain plat
	\item[-] terre très profonde, avec une grande réserve en nutriment et en eau
	\item[-] nappe phréatique située à quelque mètre sous le sol assurant une humidité permanente quelque soit la météo. Evite toute catastrophe liée à la sécheresse.
\end{itemize}

\textbf{Inconvénients : }
\begin{itemize}
	\item[-] une croûte de battance peut se former après une pluie et peut empêcher une bonne levée du semis. Il faut passer la herse étrille pour aérer le sol.
	\item[-] ces terrains ne souffrent jamais d'un excès de sécheresse mais l'humidité, idéale pour le maïs ou le soja, peut limiter les rendements du blé, d'après les retours d'un agriculteur en conventionnel. 
\end{itemize}

La plaine de Moirans est réputée pour avoir des terres plates et fertiles, permettant à l'agriculture conventionnelle d'obtenir des rendements exceptionnels en maïs. La culture du blé en AB est donc relativement aisée à conduire. 

\subsubsection{Saint Étienne de Crossey} 

Situés à 10 km au nord des plaines de Moirans, les terrains à Saint Étienne de Crossey sont quant à eux principalement argileux et en pente douce, au pied d'une colline. Le sol est en permanence humide, en cas de pluie, le sol est rapidement saturé en eau, au point de faire de légères résurgences.

Ainsi, à l'inverse des plaines de Moirans, ces parcelles sont plus compliquées à travailler : 
\begin{itemize}
	\item[- ] la terre étant plus lourde, elle est plus difficile à travailler et s'accumule rapidement dans une herse rotative ;
	\item[- ] les créneaux pour travailler le sol sont étroits. En effet, le temps de réssuyage est long et les jours de pluie sont nombreux, laissant peut de marge de manœuvre sur les période de travail du sol. 
	\item[- ] l'humidité est parfois telle que le risque d'embourber le matériel est réel. Aussi, de grandes flaques peuvent apparaître est empêcher tout semis, ou travail du sol.
\end{itemize}

Néanmoins, les rendements semblent similaires\footnote{Ces terrains ont été acquis récemment, il n'y a pas assez de recul avec les rotations pour avoir une idée précise du rendement.} à ceux de la plaine de Moirans. Olivier souligne qu'ils sont plus difficiles à travailler, avec des créneaux de passage au champ plus restreints et un risque plsu élevé en cas de sécheresse de printemps.

\subsection{Rendement attendu}

Les rendements pour le blé attendus (et obtenus en cas d'année normale, c'est-à-dire sans accident) sont de 30 à 35 quintaux par hectare. Ils se situent dans la fourchette haute de la culture du blé en agriculture biologique, qui sont plutôt autour de 25-30 qx/ha. En comparaison, les rendements atteints en "terres à céréales" du bassin Parisien atteignent 40qx/ha en AB. En agriculture conventionnelle, les terres environnantes à un rendement de 60-70 qx dans les terres environnantes, et jusqu'à 100qx dans le bassin parisien et le nord de la France\footnote{Source : \textit{Rotations pratiquées en grandes cultures biologiques en France : état des lieux par région, ITAB, 2011}, et témoignage de différents agriculteurs de la plaine du Grésivaudan}. 

A noter que les rendements obtenus par le GAEC sont plutôt bons, malgré l'utilisation de variétés anciennes, ce qui semble être principalement du à la rotation, qui permet un apport suffisant de matière azotée et à la qualité des terres de Moirans.

Avec ces rendements, la production totale de blé est de 5$\times$30qx, soit 15 tonnes de grain. La production suffit donc à la production de pain et de réensemencement. Les 15 tonnes de grains ont le potentiel de donner 10,5 tonnes de farine (en enlevant le son), et suffisent à la boulangerie qui consomment environ 10 tonnes de farine à l'année.

\section{Rotation}
\label{part:rotation}

Dans un système grande culture, il est impossible d'intervenir manuellement ou ponctuellement sur les culture du fait de l'importante taille des parcelles, contrairement au maraîchage, où les interventions manuelles sont possibles. Il faut donc pouvoir gérer l'enherbement, la fertilisation ou l'impact des ravageurs à grande échelle, avec des outils mécanisés. En agriculture biologique, il est alors indispensable de s'appuyer sur des rotations longues et diversifiées, avec des critères d'alternances similaires au maraîchage.

\subsection{Calendrier de culture}

La culture du blé est caractérisée par seulement trois étapes importantes : le semis, le passage de la herse étrille (qui n'est pas systématique) et la moisson.

\subsubsection{Semis}

\textbf{Le blé tendre est semé à la fin de l'automne, à la mi-novembre.} Le blé est une graminée capable de pousser dès qu'il fait 5$^\circ$C : il s'implante donc doucement durant l'hiver, où il prend l'avantage sur ces concurrentes. Au printemps, déjà implanté, il pousse fort et étouffe les concurrentes potentielles. La principale difficulté est de trouver le créneau météorologique pour semer à l'automne, où les terrains sont souvent gorgés d'eau. 

\subsubsection{Passage de la herse étrille}

Réalisé au printemps, il permet de détruire les adventices qui se développent à la fin de l'hiver. Le blé supporte très bien son passage. Il permet aussi d'aérer le sol et de casser la croute de battance.  

\subsubsection{Moisson}

Réalisée début juillet, lorsque la récolte est mure : les grains sont totalement remplis, la tige se désèche, l'épi se recroqueville et le taux d'humidité descend à 15\%. La récolte est effectuée par la moissonneuse batteuse d'un agriculteur de la plaine de Moirans. 


\subsection{Importance de la rotation dans un système paysan-boulanger AB}

Dans un système paysan-boulanger, une importance cruciale est donnée à la place du blé dans la rotation, car c'est la culture qui est de loin la plus utilisée \textit{in fine} en boulangerie (environ 200kg de farine de blé par semaine) et qui a une importance économique cruciale : la boulangerie représente un chiffre d'affaire d'environ 100 000\euro{}, soit presque la moitié du GAEC. La rotation est donc pensée de sorte à donner une place centrale au blé tout en variant les cultures sur une parcelle donnée pour avoir les effets positifs d'une rotation : apport azoté, structure du sol, maîtrise des adventices, résistances aux maladies. Il y a un juste compromis à trouver entre la qualité agronomique d'une rotation longue et très diversifiée et l'impératif de valorisation économique en laissant une place prépondérante au blé meunier.

 Les différents systèmes paysan-boulanger peuvent avoir des surfaces consacrées au blé variant de 33\% à 50\% (soit d'un tiers à la moitié des blocs)\footnote{Ces valeurs sont des témoignages de paysans-boulangers du Dauphiné.}. Il est à noter que le rendement du blé semble décroître lorsque sa présence augmente dans l'assolement. Dans la région dauphinoise, une rotation "classique" de paysan-boulanger est constituée de 2 ou 3 ans de luzerne ou de trèfle, puis 2 ou 3 ans de céréales (blé et seigle). 

Dans des systèmes de grandes cultures locales, c'est-à-dire dans la plaine alluvionnaire de l'Isère (dont nous reparlerons des caractéristiques pédoclimatique plus loin) où se situent les terrains du GAEC, les agriculteurs sont en très grande majorité en agriculture conventionnelle, avec une rotation maïs-soja, qui sont les cultures les plus productives localement, avec parfois du blé. 

\subsection{Rotation choisie}

La rotation des céréales à la ferme "La poule aux fruits d'or" se fait donc en 3 temps, sur 3 blocs de 8ha. Le premier bloc est dédié aux céréales (blé, seigle, petit épeautre), le deuxième est constitué d'une prairie enherbée jouant le rôle d’engrais vert (trèfle, ray-grass) et le dernier est dédié aux légumineuses (soja, féverole, triticale-pois)\footnote{On inclut dans la partie légumineuse l'association "triticale-pois". Le triticale est une céréale de la même famille du blé (graminée) permettant de tutorer le pois, qui lui est une légumineuse. La fixation en azote du pois permet le bon développement du triticale et un reliquat pour les céréales qui suivent. Le mélange est dédié à l'alimentation des poules de la ferme.}. La proportion de blé est donc ici ans plutôt basse (33\%), mais les rendements sont dans la fourchette haute en AB, à 35 quintaux l'hectare, et ce, malgré l'utilisation de variétés anciennes\footnote{Les rendements en AB de blé tendre dans la région Rhône-Alpes sont compris entre 25 et 35qx/ha. Pour comparaison, ils sont à 60 qx/ha en conventionnel dans la région, et à 40 qx/ha en moyenne en AB en Île-de-France. Source : \textit{Rotations pratiquées en grandes cultures biologiques en France : état des lieux par région, ITAB, 2011}}. 

\begin{center}
	\includegraphics[scale=0.9]{rotation0.pdf}
\end{center}

Le raisonnement suivi pour cette rotation est le suivant : 
\begin{itemize}

	\item[1 - ] \textbf{Céréales} : l'intérêt est tout d'abord économique, à travers la transformation en pain. Les céréales sont implantées en octobre (petit épeautre) et en novembre (blé et seigle). Une caractéristique des céréales est de pouvoir se développer durant l'hiver (principalement le réseau racinaire), de sorte à pouvoir prendre l'avantage sur leurs concurrentes au printemps en poussant rapidement et en étouffant les adventices concurrentielles. Elles sont récoltées au mois de juillet suivant. Les pailles sont laissées sur place pour apporter de la matière organique.
	
	\item[2 - ] \textbf{Prairie semée} : cette culture de ray-grass et trèfle (avec éventuellement de la phacélie) est dévolue à "reposer" le sol et le "régénérer" avec un apport de matière organique, azotée et de structuration du sol. Elle est implantée en septembre et laissée 18 mois pour profiter de la fixation d'azote apportée par le trèfle. Le ray-grass produit énormément de matière organique et surpasse le trèfle dans un premier temps. Il est broyé le premier avril de son implémentation, laissant la lumière au trèfle qui peut s'exprimer, donnant une prairie assez équilibrée durant l'été. Un deuxième broyage est effectué à la fin de l'été, pour éviter la montée en graine du ray-grass. Les systèmes racinaires du ray-grass et du trèfle sont complémentaires (fasciculé pour le premier, pivot pour le second), explorent le sol sur différentes strates, permettant d'obtenir une bonne structure. Le ray-grass est extrêmement concurrentiel et a un rôle "nettoyant". La prairie est labourée en février, 18 mois après son implémentation.
	
	\item[3 - ] \textbf{Légumineuses} : la fin de rotation est faite avec du soja, de la féverole ou un mélange triticale-pois. Ces légumineuses permettent aussi de fixer l'azote de l'air et permettent d'exporter les graines pour l'alimentation des poules ou revendues directement pour l'alimentation humaine pour le soja. Le mélange triticale-pois est intéressant car il permet de faire pousser une céréale (triticale, variété de blé, nettoyant) en symbiose avec le pois qui va grimper le long de sa tige.

\end{itemize}

\textbf{Bilan} : les associés de la ferme sont globalement satisfaits de la rotation ; elle permet d'obtenir des rendements en blé dans la fourchette haute de l'AB, de n'apporter aucun intrant extérieur (il n'y a pas d'amendement) et l'enherbement est maitrisé. Si la surface consacrée au blé est inférieure à d'autres rotations de paysan-boulanger, les autres cultures sont aussi valorisées, comme le soja pour l'alimentation humaine ou le triticale-pois pour l'alimentation des poules de la ferme. Une limite sur cette rotation serait sa simplicité et son faible nombre de cultures (3 blocs seulement). Une justification est que la faible surface cultivée (24ha au total) permet difficile de diviser en un plus grand nombre de cultures.

\section{Choix variétal}

\subsection{Variétés}

Trois variétés sont utilisées : une moderne, l'Energo, et deux anciennes, la Florence aurore et le Rouge de bordeaux. Le choix entre variétés modernes et anciennes se fait selon les critères de rendement, de résistance à la verse et aux maladies, à la teneur en protéine, à la qualité boulangère mais aussi éthiques. 

Olivier a choisi d'utiliser une variété moderne avec deux anciennes pour moyenner les qualités de l'une et de l'autre. Utiliser uniquement des variétés anciennes peut augmenter le risque d'un gros accident de verse. Si l'Ernergo est semé dans uen parcelle qui lui est propre, les deux variétés anciennes sont semées dans le même champ et sont resemées ensemble. 

\subsubsection{Energo}

C'est une variété dite "moderne" ou commerciale, utilisée à hauteur de 2ha sur les 5ha. C'est un blé à paille longue (moyenne de 116 cm), qui présente un compromis intéressant entre rendement et teneur en protéines. Il présente aussi de bonnes qualités boulangères.

\subsubsection{Florence Aurore et Rouge de Bordeaux}

Ce sont des variétés dites "ancienne", cultivées sur 3ha. Elles font une paille très haute et leur diversité génétique en fait des variétés rustiques et résistantes aux maladies. Ce sont d'excellente variétés pour la panification.

\subsection{Tri des semences}

Réutilisation d'une partie des semences. Tri des semences avec un trieur à plaque (triage grossier) puis trieur alvéolaire (triage fin) et triage manuel (triage très fin !). 

\section{Travail du sol et préparation}

On s'intéresse au travail du sol pour la culture du blé dans la rotation. Il faut garder à l'esprit que le travail du sol s'organise à l'échelle de la rotation ; ainsi les étapes présentées sont pertinentes dans le cadre de cette rotation. Par exemple, la maîtrise des adventices se fait grâce à la qualité nettoyante de la prairie semée et du labour qui s'en suit, mais qui n'est pas directement inclus dans le travail du sol d'une parcelle à blé.

Les étapes sont les suivantes :
\begin{itemize}
	\item[-] \textit{Sept. année N} : déchaumage du soja
	\item[-] \textit{J-15 avant semis} : faux semis
	\item[-] \textit{Nov. année N} : semis
	\item[-] \textit{mars-avril année N+1} : herse étrille
	\item[-] \textit{juil. année N+1} : moisson
\end{itemize}

\subsection{Déchaumage}

Le déchaumage est effectué avec un déchaumeur à disque et à dents (avec pattes d'oie). Il permet d'extraire touts les chaumes du soja précédent pour les extraire du sol et faciliter leur composition. Les dents permettent aussi de décompacter le sol sans le déstructurer complètement ; contrairement au labour, il n'y a pas de retournement des horizons du sol. Le déchaumage est aussi moins gourmand est gazole (10-15L/ha) que le labour (20L/ha) et est plus respectueux de la vie du sol, car moins travaillé. Le déchaumage est fait en septembre, lorsque les températures sont encore chaudes, ce qui permet une décomposition rapide des restes de soja.

\subsection{Faux-semis}

Le faux semis n'est pas systématiquement réalisé, car il faut une "fenêtre de tir" météo précise. Une quinzaine de jour avant la date prévisionnelle du semis (fait en novembre), Olivier passe la bineuse rapidement sur le champ précédemment déchaumé, la veille ou l'avant veille d'une pluie. Ce passage permet d'obtenir un lit de semence assez grossier, mais qui va permettre la levée des graines d'adventices présentes. Les germes d'adventices seront alors détruits lors du semis, 15 jours plus tard.

\subsection{Préparation du lit de semence et semis}

L'objectif est de préparer un lit de semence assez fin pour le blé et de le semer dans la foulée. On utilise un combiné de semis, sur lequel Olivier fixe directement le semoir (figure \ref{fig:semis}). 

\begin{figure}[h!]
\centering
\begin{minipage}{.5\textwidth}
  \centering
  \includegraphics[width=0.9\textwidth]{combine_semis.jpeg}
  \label{fig:test1}
\end{minipage}%
\begin{minipage}{.5\textwidth}
  \centering
  \includegraphics[width=0.9\textwidth]{semoir.jpeg}
  \label{fig:test2}
\end{minipage}
\caption{\textit{(Gauche)} Combiné de semis Pichon. \textit{(Droite)} Semoir à descente Nodet}
\label{fig:semis}
\end{figure}

Le combiné de semis est composé d'un rouleau cage, de dents de vibroculteur et d'un rouleau Packer. Son passage permet d'obtenir un lit de semence assez fin et homogène.

Le semoir Nodet est fixé directement sur le combiné de semis et permet de semer directement durant le passage du combiné de semis, permettant de gagner du temps par rapport à la situation où il faudrait passer le combiné puis le semoir. 

Cependant, les deux outils attachés ensembles sont lourds, surtout si le semoir est rempli de semences (jusqu'à 200kg). Si le terrain est parfaitement ressuyé et plats, avec les chaumes précédentes bien décomposée, c'est en effet un gain de temps avec une efficacité certaine. Mais si le sol est mal ressuyé avec de la pente, le passage des deux outils est compliqué car le tracteur s'enfonce profondément dans la terre, peut maniable, et le lit de semence n'est pas parfaitement réalisé. 

C'est ce qu'il c'est passé lors du semis de ce mois de novembre, particulièrement humide, où la fenêtre de tir pour semer était trop étroite vis-à-vis de la météo. Pour éviter de trop charger le tracteur, la semence n'a pas été mise dans le semoir et il a été décidé de passer d'abord le combiné, puis ensuite semer. Malgré cette précaution, le tracteur a difficilement travaillé le sol gorgé d'eau, en s'enfonçant sur un terrain où apparaissait même des résurgences par endroit. En conséquence, le semis dans ces zones, sur lesquelles on voit des traces d'érosion, a été un échec. 

\begin{figure}
\begin{center}
	\includegraphics[scale=0.2]{IMG_3352.jpeg}
	\caption{Parcelle de blé, fin novembre. Le blé a levé, mais on voit de nettes traces d'érosion et de tassement suite à un semis dans des conditions difficiles : terrain mal ressuyé et en pente.}
\end{center}
\end{figure}

Il aurait été peut-être plus prudent, quitte à perdre un peu de temps, à monter uniquement le combiné, faire un aller-retour à la ferme, installer le semoir sans le combiné, afin d'alléger le tracteur, et d'éviter les zones trop humides. C'est un compromis délicat entre le temps passé à semer et l'efficacité du semis.

\subsection{Entretien de la culture}

L'entretien de la culture est relativement simple : il y a 2 ou 3 passages de herse étrille dans le champs de blé, à grande vitesse, au printemps. Ce passage à 2 objectifs : détruire les potentielles adventices de printemps qui sont à un stade bien plus précoce que celui du blé qui a été implanté depuis 5 mois ; et aérer le sol pour activer la vie micro-biologique du sol, qui permet de minéraliser l'azote. 

Le blé a une caractéristique intéressante : loin de souffrir du passage de la herse étrille, le fait d'être "griffé" par les dents arrache quelques feuilles, quelques racines qui ne compromet la survie du plant : au contraire, cela stimule le tallage, c'est-à-dire sa capacité à former de nouveaux épis.

Une fois passé le passage 3-4 feuilles, le blé pousse alors vite et devient assez étouffant : il ne laisse plus de lumière aux adventices concurrentes.

Il est à noter que si l'entretien de la culture du blé parait relativement simple par rapport à l'entretien de la plupart des cultures maraîchères, c'est parce qu'en amont, la rotation élaborée permet de prévenir l'apparition d'un enherbement ingérable. 

\subsection{Récolte}

La récolte est effectuée début juillet, par un agriculteur voisin qui possède une moissonneuse batteuse. 

\subsection{Cas du labour}

Dans la conduite du blé présentée, il n'est pas mentionnée la pratique d'un labour. Olivier précise que cette conduite est réalisée "quand tout va bien", c'est-à-dire sans un enherbement qui s'est implanté sur la culture de soja précédente et qui risque d'être problématique pour le semis du blé. En fonction du risque, Olivier peut réaliser un labour à la place du déchaumage pour éviter de se faire déborder. S'il cherche à éviter un labour, il ne se prive pas de cette option lorsque cela devient risqué, malgré les impacts sur l'érosion\footnote{Le risque d'érosion est uniquement présent sur les parcelles vallonnées de Saint-Etienne de Crossey, il est quasi-inexistant sur les terres plates de Moirans).}.

Le labour reste néanmoins présent dans la rotation, pour détruire la prairie semée, dont le ray-grass peut être particulièrement virulent. Ce labour permet de détruire convenablement la prairie et de prévenir le retour d'adventices. Néanmoins, on peut se poser la question de savoir si un passage de broyeur et un déchaumage suffiraient.

\section{Amendement, fertilisation et bilan humique}

A l'échelle de la rotation, il n'y a aucun intrant extérieur : pas de fertilisation, ni de fumier\footnote{Il y a une exception : il y a 0,5ha de maïs dans la rotation, à la place des légumineuses pour l'alimentation des poules. Ce maïs est fertilisé avec du Kerzaote. Comme c'est une surface négligeable à l'échelle de la rotation, nous en tiendrons pas compte.}. L'amendement provient uniquement de la prairie semée, de la paille du blé restituée. Nous allons voir que cette conduite semble suffisante pour à la fois augmenter le taux de matière organique stable dans le sol chaque année, mais aussi d'assurer les bons apports en NPK pour la culture du blé. 

\subsection{Bilan humique}

Si les terres cultivées à Moirans sont réputées pour certaines de leurs qualités, elles sont travaillées de manière intensives depuis plusieurs décennies si bien que les taux de matière organique (MO) sont devenus assez faibles (autour de 1,5\%). 

La rotation proposée pour la conduite du blé a, en partie, un objectif d'augmenter le taux de MO à travers une longue prairie semée sur place d'une part ; et en n'exportant pas les pailles après le blé. Pour vérifier que cette rotation conduit à une augmentation progressive de la MO stable dans le sol, nous allons effectuer un calcul d'ordre de grandeur.

Tout d'abord, Olivier estime le taux de MO à 2\% aujourd'hui\footnote{Par rapport à ses voisins, le taux de MO est logiquement supérieur aujourd'hui car la rotation laisse plus de déchets organiques sur place que la conduite "maïs-soja" pratiquée dans la région.}. Le sol très calcaire est à dominante limoneuse (il s'agit du même type de sol que celui de la Bâtie) : le coefficient de minéralisation\textbf{ $K_2$ est estimé à 0,8\%, c'est-à-dire que 0,8\%} de la matière organique stable est minéralisée chaque année. 

Avec les outils de déchaumage et de labour, on peut estimer que le sol est travaillé sur une profondeur de 40 cm\footnote{Le labour est réalisé à 20-25 cm de profondeur, mais les outils du déchaumeur peuvent rentrée plus profondément. Ainsi, la profondeur du sol réoxygénée régulièrement peut être estimée à 40 cm environ.}, représentant une masse de 4940t de terre fine par hectare, en prenant en compte la présence de 5\% de cailloux. \textbf{Il y a donc 98,8t de MO stable par hectare, et donc 98,8t$\times K_2$=790kg/ha/an de MO stable qui sont minéralisée chaque année.} Pour maintenir \textit{a minima} le taux de MO, il faut donc ajouter cette quantité de MO chaque année sur un hectare. 

Calculons désormais les apports de MO réalisées tout au long du cycle de la culture\footnote{Ces valeurs sont des masse de MO sèches estimées à partir d'un document ressource mis en ligne sur le Moodle.}. On utlise pour cela le coefficient $K_1$, qui représente le \% de la masse de MO sèche qui se transformera à terme en humus. 
\begin{itemize}

	\item[-] \textbf{Culture de blé} : Les racines et les pailles produisent une quantité de humus respectivement de 300 et 600kg/ha/an, soit 900kg/ha/an. 
	\item[-] \textbf{Prairie semée} : la totalité de la prairie est broyée régulièrement et laissée sur place. Une prairie de Ray-grass/trèfle incarnat produit annuellement une quantité \textit{a minima} de 10t de MO sèche, soit 5t de ray-grass et 5t de trèfle\footnote{L'estimation de 10t/ha/an est estimée à partir des différentes ressources sur l'élevage, qui cultivent ce genre de prairie pour produire du foin. On monte facilement à 16t en conventionnel.}, totalement restituée à la terre. Le coefficient $K_1$ de 0,15 permet d'obtenir une quantité d'humus de 1500kg/ha/an. 
	\item[-] \textbf{Culture de soja} : Les graines et les tourteaux de soja sont exportés, laissant uniquement les racines en MO. Pour obtenir des chiffres, on se base sur ceux de la luzerne, qui est aussi une légumineuse, en prenant les rendements les plus faibles (le soja ne semble pas créer pas autant de MO que la luzerne). On estime à 400kg/ha/an la quantité de humus restituée. 

\end{itemize}

Sur le cycle de la rotation, on obtient donc une quantité de :
\begin{align*}
	\frac{900+1500+400}{3}\simeq930\mathrm{kg/ha/an}
\end{align*}

La quantité de humus crée à terme chaque année est supérieure à celle qui est minéralisée. Par cette rotation, Olivier s'assure que le taux de MO augmente régulièrement chaque année. \textbf{Plus précisément, la quantité de humus devrait être égale à la quantité d'humus apportée chaque année $\times K_2$, cad $\simeq116$kg/ha, soit 2,4\% de MO.} Ces chiffres sont à prendre avec précautions et ne sont que des estimations ; plus particulièrement, le quantité de MO délivrée par le blé ne semble pas prendre en compte le grain qui exportée, mais c'est à vérifiée ; la prairie peut potentiellement produire plus car elle est broyée régulièrement et laissée plus d'un an en place ; enfin la quantité de MO produite par le soja est peut être inférieur à celle déduite des chiffres de la luzerne.

Néanmoins, la stratégie d'Olivier pour augmenter le taux de MO stable dans le sol semble tout à fait pertinente. On rappelle que c'est un objectif pour lui, afin d'avoir un sol vivant et des cultures plus résistantes aux conditions climatiques et face aux ravageurs. 

\subsection{Analyse de la fertilisation}

En sus du bilan humique, l'analyse précédente permet de connaitre les quantité en NPK restituées par les cultures à l'échelle de la rotation : c'est l'analyse de la fertilisation. On considère que seule la prairie est capable de mobilisée l'azote de l'air, comme rien n'est exportée, au bénéfice de la culture du blé. Les besoins en azote du blé sont de estimés à environ 3kg/t de grain produite, soit environ 110kg/ha d'azote. Les besoins en potassium et en phosphore sont difficile à trouver pour le blé ; on sait que c'est une culture peu exigeante en ces deux éléments.

\begin{figure}[h!]
\begin{tabular}{|p{5cm}|p{2cm}|p{2cm}|p{2cm}|}
\hline
 & N (kg/ha) & P (kg/ha) & K (kg/ha)\\
\hline
Trèfle incarnat (restitution) &  143 & 26 & 215  \\
\hline
Ray-grass italien (restitution) & 102 & 30 & 18 \\
\hline
Blé (besoin) & 110 & - & -\\
\hline
\end{tabular}
\end{figure}

Les apports azotés restitués par la prairie semée semblent largement suffisants pour les besoins du blé.

\section{Gestion des adventices et des maladies}

\subsection{Principales adventices}

Les principales adventices pouvant concurrencer le blé (et donc baisser les rendements) sont le rumex (\textit{Rumex Crispus}), le chardon des champs (\textit{Cirsium arvense}) et le liseron (\textit{Convolvulus arvensis}).

\subsubsection{Méthodes de lutte indirectes}

La principale méthode de lutte est la rotation proposée par Olivier, avec notamment le rôle de la prairie semée. Celle-ci est très nettoyante, avec le trèfle et le ray-grass qui sont très étouffants et privent de lumière toute les adventices. De plus, le labour qui s'en suit permet de s'assurer de leur totale destruction. 

Une seconde méthode indirecte s'effectue lors du tri des semences. Une partie du blé, les variétés anciennes, sont resemées. Elles sont triées par un trieur à alvéolaire qui sélectionne les grain par leur taille et leur forme. Ce tri permet d'éviter de resemer des graines d'adventices qui aurait été récoltées durant la moisson.

\subsubsection{Méthodes de lutte directes}

Le liseron recommence néanmoins à se développer lors de la culture du soja, salissante, et peut poser problème lors de la culture du blé suivante. Le passage de la herse étrille peut endommager ses tiges rampantes et limiter sa croissance jusqu'à que le blé atteigne une taille suffisante pour être étouffant. 

Une méthode de lutte contre le chardon et le rumex est... l'arrache manuel. Une fois par an, les associés et un employé "nettoient" les champs en arrachant les rumex et chardons installés. Cette étape représente un temps de travail d'une demie journée à 4 personnes, ce qui reste raisonnable au regard des surfaces à traiter. 

Rappelons qu'en grande culture en agriculture biologique, la propreté absolue est impossible et il s'agit de maintenir les adventices en dessous d'un seuil de nuisibilité, dont leur nombre est maîtrisable à long terme. 

\subsubsection{Cas du Ray-Grass}

Le ray-grass est souvent considéré comme une adventice en grande culture, il peut sembler étrange qu'Olivier en cultive en tant que rôle d'engrais vert. Cependant, dans sa rotation, le ray-grass est systématiquement broyé avant sa maturation, l'empêchant de répandre ses graine. D'autre part, cela ne pose pas de problème lors du tri des semence car le ray-grass se sépare très facilement des grains de blé. 

\subsection{Maladies et ravageurs du blé}

Les maladies et les ravageurs sont moins "prégnants" en grande culture qu'en maraichage. Le blé est globalement une plante rustique qui, si les rotations sont bien conduites, résiste assez bien aux maladies et aux ravageurs. Il existe tout de même quelques maladies, principalement fongiques. Il est important de les maîtriser car certaines génèrent des mycotoxines (molécules présentant un risque sanitaire avéré) rendant le blé impropre à la consommation e tà l'utilisation en boulangerie. 

\subsubsection{Carie du blé}

C'est un champignon (\textit{Tilletia caries}) qui infecte le grain à maturation et le rend impropre à la consommation et qui se propage à travers les semences. Olivier le traite en préventif en enrobant ses semences avec une eau cuivrée (à une concentration de 25g/L), d'une quantité de 4L pour 100kg de blé. Il n'a jamais eu d'apparition de la carie.

\subsubsection{Fusariose}

C'est un champignon du genre \textit{Fusarium}, courant dans le règne végétal, et qui échaude les épis après la floraison dans le cas du blé. Les grains se dessèchent et n'arrivent jamais à maturité. Olivier en remarque certaines années dans ses champs, de manière relativement limitée. La méthode de lutte est de permettre au blé de se déveopper en "bonne santé" avec un sol vivant, pas trop tassé, qui a un bon taux de matière organique. L'utilisation de variétés anciennes permet d'avoir une certaine variété génétique, et donc une meilleure résistance. D'autre part, le passage de la herse étrille oxygène le sol, active la vie biologique, dont certains microbes qui sont prédateurs des \textit{fusarium}.

\subsubsection{Ergot}

C'est (encore) un champignon parasitant le grain du blé. Il s'attaque principalement au seigle, et le blé semble présenter une certaine résistance à ce champignon. Olivier ne fait rien de préventif sur le sujet et n'en a jamais souffert sur ces parcelles. 

\subsubsection{Verse}

Ce n'est pas à proprement dit une maladie, mais peut en être une conséquence. Il s'agit du couchage à terre de la tige suite à un évènement climatique (grêle, orage) ou à une maladie de la tige (piétin-verse). Une manière de la prévenir est d'éviter les excès d'azote qui induisent une croissance disproportionnée de la tige et les semis trop dense. 

\subsubsection{Cas du charançon}

Le charançon est un insecte consommant le grain lors de son stockage en silo. La maîtrise des infestations de ce ravageur seront discutées dans la partie stockage. 

\section{Incidence du climat}

Comme toute culture, les conditions climatiques ont une incidence sur les rendements du blé. Il s'accommode très bien d'un hiver froid, d'un printemps pluvieux et d'un été sec. Globalement assez sobre en eau, il s'accommode peu au contraire des excès d'humidité, qui favorise l'apparition de maladies fongiques évoquées précédemment.  

\subsection{Conditions météorologiques}

Le climat local, du pays Voironnais, est semi-continental, similaire à celui que l'on trouve dans les "Terres froides" entre Grenoble et Vienne, caractérisé par un été chaud et un hiver rigoureux. La pluviométrie est importante (1100mm/an) à Saint-Étienne de Crossey, à cause de l'effet Foehn engendré par le massif de la Chartreuse. La plaine de Moirans, située sur les contreforts du Vercors est moins arrosée (900mm/an), mais la pluviométrie y reste plus élevée que la moyenne nationale. En conséquent, les parcelles sont particulièrement humides et n'ont pas besoin d'irrigation. Par ailleurs, le climat est suffisamment chaud l'été pour la culture du maïs et du soja. 

Néanmoins, le risque de sécheresse agricole est paradoxalement plus important sur à Saint-Étienne de Crossey, malgré la pluviométrie plus marquée, car la nappe phréatique est située à grande profondeur, contrairement aux sols de Moirans.

\subsection{Besoin en eau}

En comparaison des cultures maraichères, les besoins en eau du blé sont faibles et il peut se contenter de la pluviométrie annuelle dans la région. Olivier n'a jamais eu de problème lié à une manque d'eau pour plusieurs raisons :
\begin{itemize}
	\item[-] la sécheresse a un impact si elle intervient début juin, lors du développement des graines, au stade dit "laiteux" : les grains se remplissent de laitance. Une sécheresse risque de réduire la taille des grains (et donc le rendement final) mais risque surtout de faire coaguler les protéines de la laitance, empêchant les grains d'arriver à maturité. Cependant, il n'y a jamais eu de telles conditions à cette époque de l'année. Olivier songe à potentiellement anticiper le semis pour s'adapter à l'augmentation des températures constatées, afin de réduire ce risque.
	\item[-] Les terres de Moirans sont situées juste au-dessus de la nappe phréatique, le risque est d'autant plus faible dans ces conditions. 
\end{itemize}
Évidemment, cette culture du blé pourrait être optimisée en apportant la quantité d'eau exactement nécessaire au différents stade de développement du blé, à l'aide d'un irrigation en grande culture. Cependant, une irrigation représente un investissement certain alors que les besoins sont faibles et pour un rendement potentiellement peu augmenté ; et surtout l'irrigation doit être parfaitement maitrisée pour éviter l'apparition des maladies fongiques citées plus haut, favorisées par l'humidité. 

\subsection{Climat et maladies}

Le principal risque est le risque fongique dû à l'humidité, qui favorise leur apparition. Deux agriculteurs en conventionnels cultivant des céréales dans la plaine affirment que le blé n'est pas intéressant dans la plaine à cause de l'humidité qui y règne, limitant leur rendement à 60-70qx/ha, malgré les traitements. 

En conventionnel, l'azote n'est pas un facteur limitant grâce aux engrais synthétiques, permettant des densités de semis plus élevées, augmentant le risque de maladies fongiques. En agriculture biologique, les rendements sont principalement limités par la fourniture azotée, imposant une densité plus faible qu'en conventionnel, limitant les maladies fongiques. 

Un risque climatique est la germination sur tige : un climat trop humide sur des grains mûrs peut entraîner leur germination. Le blé est alors impropre à l'utilisation en boulangerie, car l'amidon est alors trop dégradé.

\subsection{Climat et qualité boulangère}

Le climat peut aussi présenter un risque sur la qualité boulangère du blé, c'est-à-dire la possibilité de transformer la farine avec la panification. En effet, la panification est contrôlée en partie par l'activité amylasique. L'amylase est une levure qui dégrade les macro-molécules d'amidon en glucose et fructose. Si une partie de l'amidon doit être dégradée pour rendre le pain digeste et le levain dynamique, il est indispensable à la panification : en cuisant, cette molécule "gélifie" est permet la bonne tenue du pain après cuisson. Une faible activité amylasique engendre une absence d'activité du levain (et donc pas de levée). A l'inverse, un excès dégrade une trop grande proportion d'amidon, empêchant au pain de se tenir et d'obtenir sa forme finale, et lui donne un goût sucré du fait de l'excès de glucose.

Le climat a une incidence majeure sur l'activité amylasique, à travers l'humidité lors de la moisson. Un temps excessivement sec lors de la moisson inhibe cette activité ; il faudra en tenir compte lors de la mouture en humidifiant le grain pour la stimuler ; à l'inverse un temps frais et humide la stimule. En cas d'humidité excessive, l'activité amylasique est trop forte et le grain devient inutilisable en boulangerie. Il n'y a pas de possibilité de rattraper les dommages : il faut prévoir une moisson si possible une journée sèche. 

\section{Tri et stockage}

\section{Stockage}

Le grain est stocké dans des silos métalliques pouvant contenir 100 quintaux. Il est ensuite consommé tout au long de l'année pour être transformé en farine, qui se conserve moins longtemps. La conservation du grain peut être compromise par :le compostage et le charançon. Dans ces deux cas, la solution est d'aérer le grain pour le maintenir en dessous d'une certaine température. Des ventilateurs sont pourvus en as de chaque silo de manière étanche, pour faire circuler efficacement de l'air à travers. 

\begin{itemize}

	\item[-] \textbf{le compostage du grain}. Comme toute graine, le grain de blé consomme légèrement d'oxygène et de ses réserves pour se maintenir son potentiel de germination. Cette "combustion" exothermique peut entrainer un échauffement important dans le silo où les grains sont confinés les uns sur les autres ; la réaction s'emballe et le grain se composte. Pour éviter cela, on ventile régulièrement, de préférence la nuit où lors des journées fraîches, pour maintenir la température à une valeur normale. 
	
	Par ailleurs, pour limiter l'échauffement, on peut s'assurer de la "propreté" de sa parcelle : limiter les adventices limite la présence de graines non mûres, "vertes" dans le grain, et qui se compostent d'elles mêmes rapidement. On peut aussi trier le grain après moisson pour limiter leur présence.
	
	\item[-] \textbf{le charançon du blé : \textit{Sitophilus granarius}}. C'est un petit coléoptère qui s'alimente de la farine des grains de blé. Sa prolifération peut provoquer d'énormes de dégâts dans les silos. Comme tout en agriculture biologique, le non recours au pesticide impose une approche de régulation plutôt que d'extermination des ravageurs. Le charançon cesse de se développer aux températures inférieures à 12$^\circ$, et meurt s'il est exposé à une température de moins de 5$^\circ$ sur plusieurs mois. Une ventilation régulière dès les premiers froids permet de se débarrasser des larves et des adultes (mais pas des œufs). Entre deux moissons, Olivier passe le chalumeau dans les interstices du silo pour détruire à la chaleur les œufs, d'où l'intérêt de prévoir des silos métalliques. 
	
\end{itemize}

\subsection{Tri}

Le tri est réalisé à l'aide d'un trieur à grille et d'un trieur alvéolaire, et manuellement. Le tri a plusieurs rôles : enlever les parties "vertes" dans la moisson, retirer les cailloux ou morceaux de terre du grain, enlever plus ou moins finement les graines adventices. 

Le trieur à grille permet de trier une grande quantité de grains, et fait un triage moyen, qui sépare les cailloux, la terre et certaines graines adventices. Il fonctionne en faisant passer le grain à travers des grilles de tailles différentes en vibration. A la sortie de ce trieur,  le grain a la qualité pour être utilisé directement en boulangerie. C'est ce grain qui sera stocké en silo.

Le trieur alvéolaire quant à lui permet de trier le grain de manière beaucoup plus fine et permet de préparer les semences. Il est constitué de 3 tambours alvéolés, avec des formes et des tailles différentes. Il permet de trier les graines par leur taille et leur forme. Son débit est beaucoup plus faible que le trieur à grille, il ne sert donc que pour les semences. En général, un tri manuel est encore effectué à l'issu de ce trieur. Certaines graines, comme la folle avoine, sont de taille et de forme identiques à celles du petit épeautre par exemple. Il faut les retirer manuellement. 

\subsection{Transformation et vente}

La totalité de la récolte de blé est transformée en farine à l'aide d'une meule type Astrié, puis la farine est transformée en pain.

\end{document}
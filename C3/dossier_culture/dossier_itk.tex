\documentclass{article}
 
\usepackage[utf8]{inputenc}
\usepackage[francais]{babel}
\usepackage[T1]{fontenc}      
\usepackage[top=3.5cm, bottom=3cm, left=3.0cm, right=4.0cm]{geometry}
\usepackage{graphicx}
\usepackage{eurosym}
\graphicspath{{figures/}{../figures}}
\usepackage[activate={true,nocompatibility},final,tracking=true,kerning=true,spacing=true,factor=1100,stretch=10,shrink=10]{microtype}
\title{Culture céréalière pour la transformation boulangère : exemple d'un paysan-boulanger dans le pays Voironnais}
\date{Octobre 2019}

\begin{document}

\maketitle

\section{Présentation de la culture}

\subsection{Point botanique}

On s'intéresse à la culture du blé tendre, ou \textit{Triticum aestivum}. C'est une graminée, cultivée depuis 6000 ans, à la base dans le croissant fertile (aujourd'hui de la Turquie jusqu'à l'Iraq). C'est une des plantes à la base de la civilisation indo-européenne. Son cycle est annuel : adaptée au climat chaud et sec en été et rigoureux en hiver, elle s'implante en début d'hiver, durant lequel elle pousse lentement mais prend l'avantage sur ces concurentes en étant déjà présente au printemps, puis termine son cycle (croissance végétative et épiaison) au milieu de l'été. Les pieds sèchent alors rapidement, les épis se recroqueville, pour déposer les grains naturellement au sol afin de regermer l'année suivante. 

Parler aussi : le cycle complet, les variétés d'hiver et de printemps, le tallage. 

\subsection{Pourquoi choisir cette culture}

Le blé est un symbole de l'agriculture : c'est la céréale reine pour la panification, aliment pilier de la culture Française. Avec le seigle, le blé est transformé en farine, avec laquelle on fabrique les denrées alimentaires de base. Ces céréales sont historiquement la source calorique élémentaire dans la société française. Ainsi, si dans l'imaginaire collectif, la transition agroécologique est représentée surtout sur la production maraichère, elle doit, ou devra, impérativement se focaliser sur la culture des céréales qui sont la base de notre alimentation à grande échelle. Enfin, si la culture du blé et céréalière en général n'est pas très rentable (entre 100 et 1000\euro{} par hectare en AB, à comparer aux rendements de 30000\euro{}/ha en maraichage), le blé peut être très bien valorisé à travers la panification, qui décuple la valeur ajoutée. 

En terme agronomique, l'intérêt du blé (et les grandes cultures en général), en comparaison au maraîchage, est que le temps dédié aux cultures est très faible, que les variétés sont très adaptées aux climats locaux et ne nécessitent pas systématiquement d'irrigation. En revanche, la mécanisation est indispensable sur les surfaces travaillées. 

\section{Présentation de la structure et place de la culture}

\subsection{Présentation de la ferme}

Nous nous intéressons ici à la culture céréalière réalisée dans la ferme "La poule aux fruits d'or", un GAEC de trois associés installé à Saint-Étienne de Crossey. Les céréales cultivées sont principalement le blé, le seigle et le petit épeautre, et dans une moindre mesure, le soja et du maïs. Toute la production est en agriculture biologique, et avec pour ligne directrice de réduire au minimum les intrants extérieurs. Les céréales sont directement transformées dans leur activité boulangère ou utilisées en aliment pour leurs poules pondeuses.

\subsection{Gamme}

La ferme propose à la vente au marché, en fonction de la saison :
\begin{itemize}
	\item[-] du pain de campagne, pain complet, pain aux graines, pain de petit épeautre, pain de mie et brioche ;
	\item[-] des fruits frais : pomme, poire, kiwi, fraise, framboise, noix ;
	\item[-] de la confiture, des sorbets, des jus, des coulis et des sirops ;
	\item[-] des oeufs de ponte.
\end{itemize}

La ferme produit par ailleurs du soja pour l'alimentation humaine, mais aussi du triticale, du pois, de la féverolle et du maïs pour l'alimentation des poules pondeuses.

Les produits de boulangerie sont fabriqués exclusivement à partir des céréales cultivées à la ferme\footnote{Sauf lors d'années avec une catastrophe climatique, ce qui est exceptionnel)} : blé tendre, seigle et petit épeautre. Toutes sont dévolues à la boulangerie à travers la gamme de produits suivants :
\begin{itemize}
	\item[-] pain de campagne : 78\% de farine de blé T80, 12\% de farine de seigle ;
	\item[-] pain complet : 97\% de farine de blé T130, 3\% de son ;
	\item[-] pain petit épeautre : 100\% petit épeautre. Cette céréale ne contient que très peu de gluten, elle est difficile à panifier, mais permet d'attirer une clientèle sensible à l'excès de gluten dans son alimentation.
\end{itemize}
Les pains sont préparés deux fois par semaine (le jeudi et le samedi), soit "moulés" soit en baguette, soit en "olive". 

\subsection{Caractéristiques des cultures}

Les cultures de blé se répartissent à Saint-Etienne de Crossey, à Saint-Jean de Moirans et à Moirans, en rotation avec le seigle, la prairie semée et les légumineuses. La rotation sera présentée dans la partie \ref{part:rotation}. Les céréales sont cultivées les surfaces suivantes :
\begin{itemize}
	\item[-] 5ha de blé
	\item[-] 0,5ha de seigle
	\item[-] 2,5ha de petit épeautre	
\end{itemize}
Le blé et le seigle ont des rendements similaires (aux alentours de 30 quintaux à l'hectare), les surfaces sont cultivées en proportion de l'utilisation en boulangerie : 1/10 de seigle par rapport au blé. 

\subsection{Caractéristiques pédologiques}

Le parcellaire est géographiquement assez éclaté, et les terrains sont assez différents. Tous les terrains agricoles dédiés aux céréales sont entourés de cultures en pratique conventionnelle. Olivier laisse une bande tampon permettant aussi le passage d'engins agricoles. 

\subsubsection{Plaine de Moirans} 

C'est une terre principalement limoneuse dans le lit alluvionnaire de l'Isère, avec des terrains plats. 

\textbf{Avantages :}
\begin{itemize}
	\item[-] terre facile à travailler sur terrain plat
	\item[-] terre très profonde, avec une grande réserve en nutriment et en eau
	\item[-] nappe phréatique située à quelque mètre sous le sol assurant une humidité permanente quelque soit la météo. Evite toute catastrophe liée à la sécheresse.
\end{itemize}

\textbf{Inconvénients : }
\begin{itemize}
	\item[-] une croûte de battance peut se former après une pluie et peut empêcher une bonne levée du semis. Il faut passer la herse étrille pour aérer le sol.
	\item[-] ces terrains ne souffrent jamais d'un excès de sécheresse mais l'humidité, idéale pour le maïs ou le soja, peut limiter les rendements du blé, d'après les retours d'un agriculteur en conventionnel. 
\end{itemize}

La plaine de Moirans est réputée pour avoir des terres plates et fertiles, permettant à l'agriculture conventionnelle d'obtenir des rendements exceptionnels en maïs. La culture du blé en AB est donc relativement aisée à conduire. 

\subsubsection{Saint Étienne de Crossey} 

Situés à 10 km au nord des plaines de Moirans, les terrains à Saint Étienne de Crossey sont quant à eux principalement argileux et en pente douce, au pied d'une colline. Le sol est en permanence humide, en cas de pluie, le sol est rapidement saturé en eau, au point de faire de légères résurgences.

Ainsi, à l'inverse des plaines de Moirans, ces parcelles sont plus compliquées à travailler : 
\begin{itemize}
	\item[- ] la terre étant plus lourde, elle est plus difficile à travailler et s'accumule rapidement dans une herse rotative ;
	\item[- ] les créneaux pour travailler le sol sont étroits. En effet, le temps de réssuyage est long et les jours de pluie sont nombreux, laissant peut de marge de manœuvre sur les période de travail du sol. 
	\item[- ] l'humidité est parfois telle que le risque d'embourber le matériel est réel. Aussi, de grandes flaques peuvent apparaître est empêcher tout semis, ou travail du sol.
\end{itemize}

Néanmoins, les rendements semblent similaires\footnote{Ces terrains ont été acquis récemment, il n'y a pas assez de recul avec les rotations pour avoir une idée précise du rendement.} à ceux de la plaine de Moirans. Olivier souligne qu'ils sont plus difficiles à travailler, avec des créneaux de passage au champ plus restreints et un risque plsu élevé en cas de sécheresse de printemps.

\section{Rendement attendu}

Les rendements pour le blé attendus (et obtenus en cas d'année normale, c'est-à-dire sans accident) sont de 30 à 35 quintaux par hectare. Ils se situent dans la fourchette haute de la culture du blé en agriculture biologique, qui sont plutôt autour de 25-30 quintaux. En comparaison, l'agriculture conventionnelle à un rendement de 70 quintaux dans les terres environnantes (et jusqu'à 100 dans certaines régions de France). Ces rendements, plutôt bons, semblent être dus principalement à la rotation, qui permet un apport suffisant de matière azotée et à la qualité des terres de Moirans (où se situent la majorité des cultures de blé). 

Avec ces rendements, la production totale de blé est de 5$\times$30q, soit 15 tonnes de grain. La production suffit donc à la production de pain et de réensemencement. Les 15 tonnes de grains ont le potentiel de donner 10,5 tonnes de farine (en enlevant le son), et suffisent à la boulangerie qui consomment environ 10 tonnes de farine à l'année.

\section{Calendrier de culture et rotation}
\label{part:rotation}

\subsection*{Semis}

\textbf{Le blé tendre est semé à la fin de l'automne, à la mi-novembre.} Le blé est une graminée capable de pousser dès qu'il fait 5$^\circ$C : il s'implante donc doucement durant l'hiver, où il prend l'avantage sur ces concurrentes. Au printemps, déjà implanté, il pousse fort et étouffe les concurrentes potentielles. La principale difficulté est de trouver le créneau météorologique pour semer à l'automne, où les terrains sont souvent gorgés d'eau. 

Hiver mais possible semis d eprintemps. Risque du printemps : sécheresse des plantules, terres qui se sèche ou qui gèle et qui casse les racines en se contractant. Moins de concurrence face aux adventices.

\subsection{Rotation}

En trois temps : céréales (blé d'hiver, seigle, petit épeautre), prairie semée (ray-grass et trèfle), puis légumineuse (soja, triticale-pois, ).

\begin{center}
	\includegraphics[scale=0.5]{rotation.pdf}
\end{center}

\section{Choix variétal}

\subsection{Variétés}

Variété anciennes\textbf{ (lesquelles ??)} et modernes \textbf{(lesquelles ??)}

\subsection{Tri des semences}

Réutilisation d'une partie des semences. Tri des semences avec un trieur à plaque (triage grossier) puis trieur alvéolaire (triage fin) et triage manuel (triage très fin !). 

\section{Travail du sol et préparation}

\subsection{Labour}

Labour (quelle profondeur ?) quelques jours ou semaines avant semis, cela permet la décomposition de la culture ou de l'EV détruit. 
Le labour détruit totalement les adventices.

\subsection{Préparation du lit de semence}

On utilise un combiné de semis. Affine grandement la terre labourée avant semis.

\begin{figure}[h!]
\centering
\begin{minipage}{.5\textwidth}
  \centering
  \includegraphics[width=0.9\textwidth]{combine_semis.jpeg}
  \label{fig:test1}
\end{minipage}%
\begin{minipage}{.5\textwidth}
  \centering
  \includegraphics[width=0.9\textwidth]{semoir.jpeg}
  \label{fig:test2}
\end{minipage}
\caption{\textit{(Gauche)} Combiné de semis Pichon. \textit{(Droite)} Semoir à descente Nodet}
\label{fig:test}
\end{figure}

\subsection{Semis}

\subsubsection{Calibration du semoir}

Le blé, le seigle et le petit épeautre nécessitent des densités à l'hectare différente. Il faut calibrer le semoir : décrire protocole.

\subsubsection{Semis}

Fenêtre météo. Difficultés rencontrées (bourrage, casse, sol pas ressuyé)

\subsection*{Entretien de la culture}

Passage de la herse étrille. plus de détail dans la dernière partie

\subsection{Récolte}

Période ? Sous traitée à un agriculteur du coin. 
Risques ? Trop tôt : grain trop humide, trop tard : verse, grain qui tombe. 

\section{Amendement, fertilisation et bilan humique}

Discussion potentiel agronomique Engrais vert : prairie semée. Estimation de l'azote apporté ? Satisfaction de la rotation ? Apport légumineuse ?

Pas de fertilisation, ni apport de fumier.

\subsection{Analyse de la fertilisation}

A demander, voir avec les rendements. 

\section{Gestion des adventices et entretien de la culture}

\subsection{Principale advendices}

\textbf{A compléter}

Rumex et chardon très problématiques. Gestion : herse étrille passée au printemps, désherbage manuel ponctuel (exceptionnel, compliqué à cause de la taille des parcelles).

\subsection{Méthodes de lutte}

\textbf{A compléter}

Labour et herse étrille

Parler de l'intérêt des rotation en grandes culture. Nettoyage de la prairie semée, l'évaluer. Notamment du seigle et son effet allopathique.

\section{Protection phytosanitaire}

Les maladies et les ravageurs sont moins "prégnants" en grande culture qu'en maraichage. Le blé est globalement une plante rustique qui, si les rotations sont bien conduites, résiste assez bien aux maladies et aux ravageurs.

\subsection{Ravageurs et maladie}

\textbf{A compléter}

Principale maladies ou ravageurs : carie du blé, fusariose, ergot, verse (pas une maladie mais peut être déclenchée par un champignon), mycotoxine.

\subsection{Traitement}

\textbf{A compléter}

Rotation, choix de la densité. 

\section{Incidence du climat}

\subsection{Climat sur les différentes parcelles}

Détailler à Crossey et à Moirans. 

\subsection{Conditions pédo-climatique}

Détailler Crossey et Moirans. Incidence des terres sur la rétention en eau, la possibilité de faire du soja à Crossey.

\subsection{Besoin en eau}

Les besoins en eau du blé : fort à la croissance, à l'épiaison puis besoin de sec pour la maturation

\subsection{Climat et ravageurs}

Lien entre le climat et l'apparition de maladies

\subsection{Risques climatique}

Sécheresse au printemps (rendement), humidité en été (germination sur tige), orage, grêle

\section{Récolte, tri et stockage}

\subsection{Récolte}

Batteuse, période de séchage ? Combien de temps en silo avant transformation en farine ? Combien de temps peut se stocker le grain ?

\subsection{Tri}

Tri grossier pour le grain à moudre, fin pour les semences. Cf partie tri.

\section{Stockage}

En silo. Précautions, ventilation ? Forme des silo ? Quand aère t-on, comment ? 

\subsection{Transformation et vente}

\end{document}